% Филипп Вадлерasdasd
\subsection{\texorpdfstring{Линейные типы}{Linear types}}
%Пусть у нас есть некоторые набор переменных $x$, $y$, $z$ и некоторый набор термов $s$, $t$, $u$, $v$, $w$.
Определим грамматику для нашего $\lambda$-исчисления:
\begin{bnf}
\begin{alignat*}{2}
	s ::= &\ x \\
			  | &\ \lambda\pair{x}.s | s\pair{s} \\
			  | &\ \oc s | \caseb{s}{\oc x \rightarrow s} \\
			  | &\ \pair{s,s} | \caseb{s}{\pair{x,x} \rightarrow s} \\
			  | &\ \ppair{s,s} | \fst \pair{s} | \snd \pair{s} \\
			  | &\ \inl \pair{s} | \inr \pair{s}|
                    \casec{s}{\inl \pair{x} \rightarrow s}{\inr \pair{x} \rightarrow s}
\end{alignat*}
\end{bnf}
Будем типизировать это $\lambda$-исчисление линейными высказываниями. Выпишем аксиомы:
\begin{@empty}
\inferspacing
\begin{gather*}
	\infer[]
		{[\Gamma]\vdash \oc s : \oc A}
		{[\Gamma]\vdash s : A} \qquad
	\infer[]
		{\Gamma, \Delta\vdash \caseb{s}{\oc x \rightarrow t} : B}
		{\Gamma\vdash s : \oc A && \Delta, [x : A]\vdash t : B} \\
	\infer[]
		{\Gamma, \Delta\vdash \pair{s, t} : A \otimes B}
		{\Gamma \vdash s : A && \Delta\vdash t : B} \qquad
	\infer[]
		{\Gamma, \Delta \vdash \caseb{s}{\pair{x, y} \rightarrow t} : C}
		{\Gamma\vdash s : A \otimes B && \Delta, \pair{x : A}, \pair{y : B} \vdash t : C} \\
	\infer[]{\Gamma \vdash \ppair{s, t} : A \with B}{\Gamma \vdash s : A && \Gamma \vdash t : B} \qquad
	\infer[]{\Gamma \vdash \fst \pair{s} : A}{\Gamma \vdash s : A \with B} \qquad
	\infer[]{\Gamma \vdash \snd \pair{s} : B}{\Gamma \vdash s : A \with B} \\
	\infer[]{\Gamma, \lambda\pair{x}.u : A \multimap B}{\Gamma, \pair{x : A} \vdash u : B} \qquad
	\infer[]{\Gamma, \Delta \vdash  s \pair{t} : B}{\Gamma \vdash s : A \multimap B && \Delta \vdash t : A} \\
	\infer[]{\Gamma \vdash \inl \pair{s} : A \oplus B}{\Gamma \vdash s : A} \qquad
	\infer[]{\Gamma \vdash \inr \pair{s} : A \oplus B}{\Gamma \vdash s : B} \\
	\infer[]
		{\Gamma, \Delta \vdash \casec{s}{\inl \pair{x} \rightarrow t}{\inr \pair{y} \rightarrow v} : C}
		{\Gamma \vdash s : A \oplus B && \Delta, \pair{x : A} \vdash t : C && \Delta, \pair{y : B} \vdash v : C}
\end{gather*}
\end{@empty}
Вложим просто типизируемое $\lambda$-исчисление в линейные типы.
\begin{align*}
	\lambda x.s &= \lambda\pair{x'}.\caseb{x'}{\oc x \rightarrow s} \\
	st          &= s\pair{\oc t}
\end{align*}
\begin{example}[Ломаем линейные типы]
	Пусть у нас есть выражения $f$, $g$  типа $A \multimap B$.
	Возьмем $h = \lambda x.\pair{f\pair{x}, g\pair{x}} : \oc A \multimap B \otimes B$.
	Докажем, что $h$ действительно существует в линейных типах:

	\[
		\infer[]{[f : A \multimap B], [g : A\multimap B]\vdash \lambda x.\pair{f\pair{x}, g\pair{x}} : \oc A \multimap B \otimes B}{
			\infer[]{
				[f : A \multimap B], [g : A \multimap B], \pair{x' : \oc A}
				\vdash \caseb{x'}{\oc x \rightarrow \pair{f\pair{x}, g\pair{x}} : B \otimes B}
			}{
				\infer[]{
					\pair{x' : \oc A}, [f : A \multimap B], [g : A \multimap B]
				\vdash \caseb{x'}{\oc x \rightarrow \pair{f\pair{x}, g\pair{x}} : B \otimes B}
				}{
					\pair{x' : \oc A} \vdash x' : A &&
					\infer[]{[f : A \multimap B], [g : A \multimap B], [x : A]\vdash \pair{f\pair{x}, g\pair{x}} : B \otimes B}{
						\infer[]{\vdots}{
						\infer[]{[x : A], [f : A \multimap B], [x : A], [g : A \multimap B]\vdash \pair{f\pair{x}, g\pair{x}} : B \otimes B}{
								\infer[]{[x : A], [f : A \multimap B]\vdash f\pair{x} : B}{[f : A \multimap B] \vdash f : A \multimap B}
								&&
								\infer[]{[x : A], [g : A \multimap B]\vdash g\pair{x} : B}{[g : A \multimap B] \vdash g : A \multimap B}
							}
						}
					}
				}
			}
		}
	\]

	В функции $h$ обьект $A$ был интуционистким, но функции $f$ и $g$ работают с ним, как с линейным.
	Как мы видим линейный тип не дает на размножать обьект (получить несколько ссылок на него в разных местах),
	но при этом у нас нет гарантий того, что до этого этот обьект не был размножен.
\end{example}

\subsection{\texorpdfstring{Уникальные типы}{Unique types}}
\begin{definition}[Род]
	Родом будем называть выражение, удовлетворяющие следующей граммтике:
	\begin{bnf}
	\begin{alignat*}{3}
		\kappa ::= &\ T && \qquad \text{Базовый тип} \\
				 | &\ U && \qquad \text{Аттрибут уникальности} \\
				 | &\ * && \qquad \text{Тип} \\
				 | &\ \kappa \rightarrow \kappa
	\end{alignat*}
	\end{bnf}
\end{definition}
\begin{definition}[Типовые константы]\ \\
	\begin{tabular}{llll}
		$\mathtt{Int}$, $\mathtt{Bool}$ & :: & $T$ & Базовые типы \\
		$\rightarrow$                   & :: & $* \rightarrow * \rightarrow T$ & Коструктор функций \\
		$\bullet, \times$               & :: & $U$ & Уникальный, не уникальный \\
		$\vee, \wedge$                  & :: & $U \rightarrow U \rightarrow U$ & Дизьюнкия, коньюнкция аттрибутов \\
		$\neg$                          & :: & $U \rightarrow U$ & Отрицание аттрибута \\
		$\mathtt{Attr}$                 & :: & $T \rightarrow U \rightarrow *$ & Конструктор типа
	\end{tabular}
\end{definition}
С помощью типовых констант мы можем делать различные типы разных родов. 
Тип (рода $*$) состоит из базового типа (рода $T$) и аттрибута уникальности (это тоже тип, только рода $U$). 
Аттрибут показывает обязан ли обьект этого типа быть уникальным (всего на этот обьект может существовать только однаа ссылка) "--- $\bullet$,
или это не обязательно "--- $\times$.
На аттрибутах заданы различные булевы операции, которые выполняются так, будто $\bullet \equiv true$, $\times \equiv false$.

Немного сахара:
\[
	\attr{t}{u} \equiv t^u \qquad \attr{(a \rightarrow b)}{u} \equiv a \xrightarrow{u} b
\]

Грамматика выражений будет похожа на классическую грамматику выражений, за исключением того,
что у переменных будет указываться аттрибут ($x^\odot$), если на переменную есть только 1 ссылка,
или аттрибут ($x^\otimes$), если на переменную имеется несколько ссылок.
\begin{bnf}
\[
	e ::= \lambda x.e | ee | x^\odot | x^\otimes
\]
\end{bnf}%
В доказательствах будем использовать следующую нотацию: $\Gamma \vdash e : \tau |_{\fv}$.
$\Gamma$ и $\fv$ отображают переменные в типы. 
Разница заключается в том, что $\Gamma$ отображает в типы рода $*$,
а $\fv$ отображает в типы рода $U$.

\todo пояснить про правила
\begin{@empty}
\inferspacing
\begin{gather*}
	\infer[]{\Gamma, x : t^u \vdash x^\odot : t^u |_{x:u}}{} \qquad
	\infer[]{\Gamma, x : t^\times \vdash x^\otimes : t^\times |_{x:\times}}{} \\
	\infer[]
		{\Gamma \vdash \lambda x.e : a \xrightarrow{\vee \fv'} b |_{\fv'}}
		{\Gamma, x : a \vdash e : b |_{\fv} && \fv' \equiv \fuck_x \fv} \\
	\infer[]
		{\Gamma \vdash e_1e_2 : b |_{\fv_1 \cup \fv_2}}
		{\Gamma \vdash e_1 : a \xrightarrow{u} b |_{\fv_1} \qquad \Gamma \vdash e_2 : a |_{\fv_2}}
\end{gather*}
\end{@empty}
