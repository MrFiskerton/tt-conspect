\section{$\lambda$-исчисление}

\begin{definition}[$\lambda$-выражение]
    $\lambda$-выражение "--- выражение, удовлетворяющее грамматике:
    \begin{bnf}
    \begin{alignat*}{3}
        \Lambda ::= & \lambda{}x.\Lambda{} \qquad && (абстракция) \\
                  | & \Lambda{}\Lambda{}          && (аппликация) \\
                  | & x                                           \\
                  | & \left(\Lambda\right)
    \end{alignat*}
    \end{bnf}
    \begin{enumerate}[label=(\alph*)]
        \item аппликация левоассоциативна
        \item абстракция распространяется как можно дальше вправо
        \item смысла в этом нет
    \end{enumerate}
\end{definition}

\begin{example}
    $((\lambda{} z.(z (y z))) (z x) z) = (\lambda{} z.z (y z)) (z x) z$
\end{example}
%[(а)]
%[(б)]
%[(в)]

Есть понятия связанного и свободного вхождения переменной (аналогично ИП).
$\lambda{}x.A$ связывает все свободные вхождения $x$ в $A$.
Договоримся, что:
\begin{enumerate}[label=(\alph*)]
    \item Переменные "--- $x$, $a$, $b$, $c$.
    \item Термы (части $\lambda$-выражения) "--- $X$, $A$, $B$, $C$.
    \item Фиксированные переменные обозначаются буквами из начала алфавита, метапеременные "--- из конца.
\end{enumerate}

\begin{definition}[$\alpha$-эквивалентность]
    $A$ и $B$ называются $\alpha$-эквивалентными ($A=_{\alpha}B$), если выполнено одно из следующих условий:
    \begin{enumerate}
        \item $A\equiv{}x$ и $B\equiv{}x$ .
        \item $A\equiv{}\lambda{}x.P$ и $B\equiv{}\lambda{}x.Q$. Пусть $t$ "--- новая переменная, тогда
            $P_{[x:=t]}=_{\alpha}Q_{[y:=t]}$ .
        \item $A\equiv{}PQ$, $B\equiv{}RS$, $P=_{\alpha}R$, $Q=_{\alpha}S$ .
    \end{enumerate}
\end{definition}

\begin{example}
    $\lambda{}x.\lambda{}y.xy=_{\alpha}\lambda{}y.\lambda{}x.yx$.
    \begin{proof}
        \begin{alignat*}{2}
            \lambda{}y.ty=_{\alpha}\lambda{}x.tx &\implies \lambda{}x.\lambda{}y.xy=_{\alpha}\lambda{}y.\lambda{}x.yx \\
            tz=_{\alpha}tz &\implies \lambda{}y.ty=_{\alpha}\lambda{}x.tx
        \end{alignat*}
        $tz=_{\alpha}tz$ верно по третьему условию.
    \end{proof}
\end{example}

\begin{definition}[$\beta$-редекс]
    Терм вида $\left(\lambda{}a.A\right)B$ называется $\beta$-редексом.
\end{definition}

\begin{example}
    В выражении
    $
        (
            \lambda{}f.
                \underset{A_2}{\underline{
                    (\lambda{}x.\overset{A_1}{\overline{f(xx)}})
                    \overset{B_1}{\overline{(\lambda{}x.f(xx))}}
                }}
        )\underset{B_2}{\underline{g}}
    $ два $\beta$-редекса.
\end{example}

\begin{definition}
    Множество $\lambda$-термов $\bm{\Lambda}$ назовём множеством классов эквивалентности $\Lambda$ по $(=_{\alpha})$.
\end{definition}

\begin{definition}[$\beta$-редукция]
    $A\rightarrow_{\beta}B$ (состоят в отношении $\beta$-редукции), если выполняется одно из условий:
    \begin{enumerate}
        \item $A\equiv{}PQ$, $B\equiv{}RS$ и
        \begin{alignat*}{3}
            &\text{либо } P\rightarrow_{\beta}R  &&\text{ и } Q=_{\alpha}S \\
            &\text{либо } P=_{\alpha}R           &&\text{ и } Q\rightarrow_{\beta}S
        \end{alignat*}
        \item $A\equiv{}\lambda{}x.P$, $B\equiv{}\lambda{}x.Q$, $P\rightarrow_{\beta}Q$ ($x$ из какого-то класса из $\bm{\Lambda}$).
        \item $A\equiv{}(\lambda{}x.P)Q$, $B\equiv{}P_{[x:=Q]}$, $Q$ свободно для подстановки в $P$ вместо $x$.
    \end{enumerate}
\end{definition}

\paragraph{Итак, лулзы.} Хотите знать, что такое истина?
\newcommand{\T}{\mathrm{T}}
\newcommand{\F}{\mathrm{F}}
\newcommand{\Not}{\mathrm{Not}}
\begin{alignat*}{2}
    \T   &= \lambda{}x\lambda{}y.x \\
    \F   &= \lambda{}x\lambda{}y.y \\
    \Not &= \lambda{}a.a\F\T
\end{alignat*}

Похоже на тип boolean, не правда ли?
\begin{example}
    \[
        \Not\ \T = (\lambda{}a.a\F\T)\T \rightarrow_{\beta}
            \T\F\T = (\lambda{}x.\lambda{}y.x)\F\T \rightarrow_{\beta}
            (\lambda{}y.\F)\T \rightarrow_{\beta}
            \F
    \]
\end{example}

Можно продолжить:
\begin{alignat*}{2}
    \mathrm{And} &= \lambda{}a.\lambda{}b.ab\mathrm{F} \\
    \mathrm{Or}  &= \lambda{}a.\lambda{}b.a\mathrm{T}b
\end{alignat*}

Попробуем определить числа:
\begin{definition}[Чёрчевский нумерал]
\[
    \overline{n}=\lambda{}f.\lambda{}x.f^{n}x \text{,\quadгде\quad}
    f^{n}x=
    \begin{cases}
        f\left(f^{n-1}x\right) &, n > 0 \\
        x                      &, n = 0
    \end{cases}
\]
\end{definition}

\begin{example}
\[
    \overline{3} = \lambda f . \lambda x . f (f (f x))
\]
\end{example}

Несложно определить прибавление единицы к такому нумералу:
\[
    (+1) = \lambda{}n.\lambda{}f.\lambda{}x.f(nfx) \\
\]
\begin{example}
    \[
        (+1) \overline{1} =
        (\lambda n . \lambda f . \lambda x . f (n f x)) (\lambda f . \lambda x . f x) \rightarrow_{\beta}
        \lambda f . \lambda x . f ((\lambda f . \lambda x . f x) f x) \twoheadrightarrow_{\beta}
        \lambda f . \lambda x . f (f x) =
        \overline{2}
    \]
\end{example}

\begin{definition}[$\eta{}$-эквивалентность]
    \[
        \lambda x . f x =_{\eta} f
    \]
\end{definition}
Аналог из C++: если \lstinline$int f(int x)$, то результат её вычисления равен результату вычисления
    \lstinline$[ ] (int x) { return f(x); }$ .

Арифметические операции:
\begin{alignat*}{2}
    \mathrm{IsZero} &= \lambda{}n.n(\lambda{}x.\F)\T \\
    \mathrm{IsEven} &= \lambda{}n.n\ \Not\ \T \\
    \mathrm{Add} &= \lambda{}a.\lambda{}b.\lambda{}f.\lambda{}x.a f (b f x) \\
    \mathrm{Mul} &= \lambda{}a.\lambda{}b.a (\mathrm{Add}\ b) \overline{0} \\
    \mathrm{Pow} &= \lambda{}a.\lambda{}b.b (\mathrm{Mul}\ a) \overline{1} \\
    \mathrm{Pow}^{*} &= \lambda{}a.\lambda{}b.b a
\end{alignat*}

Для того, чтобы определить $(-1)$, сначала определим "пару":
\begin{alignat*}{2}
    \left<a,b\right> &= \lambda f.f a b \\
    \mathrm{First} &= \lambda p . \T p \\
    \mathrm{Second} &= \lambda p . \F p
\end{alignat*}

$n$ раз применим функцию $f\left(\left<a,b\right>\right) = \left<b,b+1\right>$ и возьмём первый элемент пары:
\[
    (-1) = \lambda n . \mathrm{First} \left(n\ (\lambda p . \left<\left(\mathrm{Second}\ p\right), (+1)\ (\mathrm{Second}\ p)\right>)\
    \left<\overline{0},\overline{0}\right>\right)
\]

Сокращение записи:
\[
    \lambda x y . A = \lambda x . \lambda y . A
\]

\begin{definition}[Нормальная форма] \mbox{} \\
    Терм $A$ "--- нормальная форма (н.ф.), если в нём нет $\beta$-редексов. \\
    Нормальной формой $A$ называется такой $B$, что $A \twoheadrightarrow_{\beta} B$, $B$ "--- н.ф. \\
    $\twoheadrightarrow_{\beta}$ "--- транзитивно-рефлексивное замыкание $\rightarrow_{\beta}$.
\end{definition}

\begin{statement}
    Существует $\lambda$-выражение, не имеющее н.ф.
    \begin{gather*}
        \Omega = \omega \omega \\
        \omega = \lambda x . x x
    \end{gather*}
\end{statement}

\begin{definition}[Комбинатор]
    Комбинатор "--- $\lambda$-выражение без свободных переменных.
\end{definition}

Комбинатор неподвижной точки:
\[
    Y = \lambda f . (\lambda x . f (x x)) (\lambda x . f (x x))
\]

\begin{definition}[$\beta$-эквивалентность]
    $A=_{\beta}B$, если $\exists C : C \twoheadrightarrow_{\beta} A, C \twoheadrightarrow_{\beta}B$
\end{definition}

\begin{statement}
    \[
        Yf =_{\beta} f(Yf)
    \]
\end{statement}

\begin{proof} (на лекции не давалось)
    \begin{align*}
        Yf &=_{\beta} (\lambda f . (\lambda x . f (x x)) (\lambda x . f (x x))) f \\
           &=_{\beta} (\lambda x . f (x x)) (\lambda x . f (x x)) \\
           &=_{\beta} f ((\lambda x . f (x x)) (\lambda x . f (x x))) \\
           &=_{\beta} f (Y f)
    \qedhere
    \end{align*}
\end{proof}

Таким образом, с помощью $Y$-комбинатора можно определять рекурсивные функции.
\begin{example}
    \[
        \mathrm{Fact} = Y (\lambda{} f n . \mathrm{IsZero}\ n\ \overline{1}\ (\mathrm{Mul}\ n\ (f\ (-1)\ n)))
    \]
\end{example}
