\section{\texorpdfstring{Теорема Диаконеску}{}}

\begin{theorem}
    Аксиома выбора влечёт закон исключённого третьего.
\end{theorem}

\begin{proof}
    Пусть есть формула $S$. Докажем $S \vee \neg S$.
    Например,
    \begin{align*}
        A &= \left\{ x \in \left\{0;1\right\} \mid x = 0 \vee S \right\} \\
        B &= \left\{ x \in \left\{0;1\right\} \mid x = 1 \vee S \right\}
    \end{align*}
    Аксиома выбора утверждает $f : \left\{A, B\right\} \to \{0, 1\}$
    \[
        f(A) \in A \with f(B) \in B \to (f(A) = 0 \vee S) \with (f(B) = 1 \vee S) \equiv (f(A) = 0 \with f(B) = 1) \vee S
    \]
    \[
        \infer{\neg (f(A) = f(B)) \to \neg S}
        {\infer{S \to f(A) = f(B)}
        {\infer{S \to A = B}
        {\infer{\neg (f(A) = f(B)) \vee S }
        {}}}}
    \]
\todo
\end{proof}

Другая формулировка аксиомы выбора:
\[
    (\forall a \in A . \exists b \in B . Q(a, b)) \to \exists f : A \to B . \forall a \in A . Q(a, f(a))
\]
$Q(a, b) = b \in a$.

\subsection{\texorpdfstring{Сетоид}{Setoid}}

\begin{definition}
    \[
        \left<C : Type, Eq : C \to C \to Type, P : \mathrm{IsEquivalence}~Eq\right>
    \]
    $Eq$ "--- отношение эквивалентности, $P$ "--- доказательство этого факта.
\end{definition}
\begin{definition}[экстенциональность]
    \[
        f : \left< A, =_A, P_A\right> \to \left<B, =_B, P_B\right>
    \]
    $f$ экстенсиональна ($\mathrm{Ext}~f$), если из $x =_A y$ следует $f x =_B f y$
\end{definition}

Пример сетоида "--- целые числа:
\[
    \left< \left<p : \mathrm{Nat}, n : \mathrm{Nat}\right>, \overline{Eq(a+d=b+c)}, \mathrm{IsEquivalence} Eq\right>
\]
