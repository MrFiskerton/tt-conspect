\section{\texorpdfstring{Система $F$}{System F}}

Обычное $\lambda$-исчисление позволяет слишком много, просто-типизированное "--- слишком мало ($(-1)$ не выразим). Хотелось бы золотую середину.

\subsection{\texorpdfstring{Интуиционистское исчисление предикатов второго порядка}{Second order intuitionistic logic}}

\begin{definition}
    \begin{bnf}
    \[
        \Phi ::= (\Phi) | p | \Phi \rightarrow \Phi | \forall p . \Phi \color{gray}
            \underbrace{| \exists p . \Phi | \bot | \Phi \with \Phi | \Phi \vee \Phi}_{\text{не существенные}}
    \]
    \end{bnf}
    Введение кванторов:
    \begin{gather*}
        \infer[p \notin \FV(\Gamma)]{\Gamma \vdash \forall p . \varphi}{\Gamma \vdash \varphi} \qquad
        \infer{\Gamma \vdash \exists p . \varphi}{\Gamma \vdash \varphi \left[p \coloneqq \psi\right]}
    \end{gather*}
    Удаление кванторов:
    \begin{gather*}
        \infer{\Gamma \vdash \varphi \left[p \coloneqq \sigma\right]}{\Gamma \vdash \forall p . \varphi} \qquad
        \infer[p \notin \FV(\Gamma, \psi)]{\Gamma \vdash \psi}{\Gamma \vdash \exists p . \varphi && \Gamma, \varphi \vdash \psi}
    \end{gather*}
    Последние четыре связки можно выразить через первые:
    \begin{align*}
        \bot & \equiv \forall p . p \\
        \varphi \with \psi & \equiv \forall a . ((\varphi \rightarrow \psi \rightarrow a) \rightarrow a) \\
        \varphi \vee \psi & \equiv \forall a . (\varphi \rightarrow a) \rightarrow (\psi \rightarrow a) \rightarrow a \\
        \exists x . \tau & \equiv \forall a . (\forall x . \tau \rightarrow a) \rightarrow a
    \end{align*}
\end{definition}

\subsection{\texorpdfstring{Система $F$}{System F}}
\begin{definition}[Тип в системе $F$]
\[
    \tau =
    \begin{cases}
        \alpha, \beta, \gamma, \ldots & \text{(атомарный тип)} \\
        \tau \rightarrow \sigma \\
        \forall \alpha . \tau & \text{($\alpha$ "--- переменнная)}
    \end{cases}
\]
\end{definition}

\begin{definition}[Исчисление по Чёрчу в системе $F$]
    \begin{bnf}
        \begin{gather*}
            \mathbf\Lambda ::= x | \lambda p^\alpha . \mathbf\Lambda | \mathbf\Lambda \mathbf\Lambda | (\mathbf\Lambda)
            | \Lambda \alpha . \mathbf\Lambda | \mathbf\Lambda \tau
        \end{gather*}
    \end{bnf}
    $\Lambda \alpha . \mathbf\Lambda$ "--- типовая (полиморфная абстракция), $\mathbf\Lambda \tau$ "--- применение типа.

    Правила вывода:
    \begin{gather*}
        \infer[x \notin \mathrm{dom}(\Gamma)]{\Gamma, x : \sigma \vdash x : \sigma}{} \\
        \infer{\Gamma \vdash MN : \sigma}{\Gamma \vdash M : \tau \rightarrow \sigma & \Gamma \vdash N : \tau} \qquad
        \infer[(x \notin \mathrm{dom}(\Gamma))]{\Gamma \vdash \lambda x^\tau . M : \tau \rightarrow \sigma}{\Gamma, x : \tau \vdash M : \sigma} \\
        \infer[x \in \FV(\Gamma)]{\Gamma \vdash \Lambda \alpha . M : \forall \alpha : \sigma}{\Gamma \vdash M : \sigma} \qquad
        \infer[\text{(подстановка типа)}]{\Gamma \vdash M \tau : \sigma [\alpha := \tau]}{\Gamma \vdash M : \forall \alpha . \sigma} \\
    \end{gather*}
\end{definition}

\begin{example} Левая проекция: \\
    \begin{tabular}{l l l}
        & Просто типизированное $\lambda$-исчисление & Система $F$ \\
        Тип & $\pi_1:\alpha\with\beta\rightarrow\alpha$ & $\pi_1:\forall \alpha . \forall \beta . \alpha \with \beta \rightarrow \alpha$ \\
        Выражение & $\pi_1 = \lambda p . p T$ & $\pi_1 = \Lambda \varphi . \Lambda \psi . \lambda p^{\varphi\with\psi} . p T$
    \end{tabular}
\end{example}

\begin{definition}[$\beta$-редукция в $F$] \ 
    \begin{enumerate}[label=(\asbuk*)]
        \item Типовая редукция: $\left(\Lambda \alpha . M^\sigma\right) \tau \rightarrow_\beta M[\alpha:=\tau] : \sigma[\alpha := \tau]$
        \item Классическая $\beta$-редукция: $\left(\lambda x^\sigma . M\right)^{\sigma \rightarrow \tau} X \rightarrow_\beta M [x:=X] : \tau$
    \end{enumerate}
\end{definition}

\begin{theorem}[Изоморфизм Карри-Ховарда]
    $\Gamma \vdash_F M :\tau$ т.и.т.т., когда $\abs{\Gamma} \vdash \tau$ в интуиционистском исчислении предикатов второго порядка.
\end{theorem}

\begin{theorem}
    $F$ \hyperref[strong-normalization]{сильно нормализуемо}.
\end{theorem}

\todo 
