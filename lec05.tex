\section{\texorpdfstring{Логика второго порядка и полиморфизм}{Second-order logic and polymorphism}}

Обычное $\lambda$-исчисление позволяет слишком много, просто-типизированное "--- слишком мало ($(-1)$ не выразим). Хотелось бы золотую середину.

\subsection{\texorpdfstring{Интуиционистское исчисление предикатов второго порядка}{Second order intuitionistic logic}}

\begin{definition}[грамматика ИИП второго порядка]
    \begin{bnf}
    \[
        \Phi ::= (\Phi) | p | \Phi \rightarrow \Phi | \forall p . \Phi \color{gray}
            \underbrace{| \exists p . \Phi | \bot | \Phi \with \Phi | \Phi \vee \Phi}_{\text{не существенные}}
    \]
    \end{bnf}
\end{definition}

\begin{definition}[правила вывода в ИИП второго порядка]
    К правилам обычного ИИВ добавляются правила вывода для квантора всеобщности:
    \[
        \infer[(p \notin \FV(\Gamma))]{\Gamma \vdash \forall p . \varphi}{\Gamma \vdash \varphi} \qquad
        \infer{\Gamma \vdash \varphi \left[p \coloneqq \sigma\right]}{\Gamma \vdash \forall p . \varphi}
    \]
    И для квантора существования:
    \[
        \infer{\Gamma \vdash \exists p . \varphi}{\Gamma \vdash \varphi \left[p \coloneqq \psi\right]} \qquad
        \infer[(p \notin \FV(\Gamma, \psi))]{\Gamma \vdash \psi}{\Gamma \vdash \exists p . \varphi && \Gamma, \varphi \vdash \psi}
    \]
    Где $\FV(\Gamma)$ "--- множество свободных переменных всех выражений из $\Gamma$.
\end{definition}

Напомним, что в логике второго порядка переменные под кванторами могут быть любыми другими выражениями,
а не только значениями из предметного множества.

Последние четыре связки можно выразить через первые:
\begin{align*}
    \bot & \equiv \forall p . p \\
    \varphi \with \psi & \equiv \forall a . ((\varphi \rightarrow \psi \rightarrow a) \rightarrow a) \\
    \varphi \vee \psi & \equiv \forall a . (\varphi \rightarrow a) \rightarrow (\psi \rightarrow a) \rightarrow a \\
    \exists x . \tau & \equiv \forall a . (\forall x . \tau \rightarrow a) \rightarrow a
\end{align*}

$\bot$ утверждает "<любое утверждение верно">, что является критерием противоречивости.

Выражение конъюнкции можно читать как "<если всё, что выводимо из $\set{\varphi,\psi}$, выполняется,
то тогда выполняется $\varphi\with\psi$">.
Также для неё можно проверить, что правила вывода действительно выводятся через такое определение. Например:
\[
    \infer{\Gamma \vdash \varphi}
    {
        \infer{\Gamma \vdash (\varphi\rightarrow\psi\rightarrow\varphi)\rightarrow\varphi}
            {\Gamma \vdash \forall \gamma.(\varphi\rightarrow\psi\rightarrow\gamma)\rightarrow\gamma}
        &&
        \infer{\Gamma\vdash\varphi\rightarrow\psi\rightarrow\varphi}
        {\infer{\Gamma,\varphi\vdash\psi\rightarrow\varphi}
        {\infer{\Gamma,\varphi,\psi\vdash\varphi}
        {}}}
    }
\]

Формулу для дизъюнкции можно читать как "<$\varphi \vee \psi$ выполняется,
если любое $a$, которое выводимо из $\varphi$ или из $\psi$, выполняется">.
Ещё её можно понимать как следствие из формулы де Моргана $\varphi\vee\psi=\neg(\neg\varphi\with\neg\psi)$.

Формула для квантора существования получается из того, что $\exists x . \tau = \neg \forall x . \neg \tau$.

\subsection{\texorpdfstring{Система $F$}{System F}}
\begin{definition}[тип в системе $F$]
\[
    \tau =
    \begin{cases}
        \alpha, \beta, \gamma, \ldots & \text{(атомарный тип)} \\
        \tau \rightarrow \tau \\
        \forall \alpha . \tau & \text{($\alpha$ "--- переменнная)}
    \end{cases}
\]
\end{definition}

\begin{definition}[система $F$]
Грамматика выражения в системе $F$:
    \begin{bnf}
    \[
        \mathbf\Lambda ::= x | \lambda p^\alpha . \mathbf\Lambda | \mathbf\Lambda \mathbf\Lambda | (\mathbf\Lambda)
        | \Lambda \alpha . \mathbf\Lambda | \mathbf\Lambda \tau
    \]
    \end{bnf}%
    $\Lambda \alpha . \mathbf\Lambda$ "--- полиморфная абстракция, $\mathbf\Lambda \tau$ "--- применение типа.
    Правила вывода:
    \inferspacing
    \begin{gather*}
        \infer[(x \notin \mathrm{dom}(\Gamma))]{\Gamma, x : \sigma \vdash x : \sigma}{} \\
        \infer{\Gamma \vdash MN : \sigma}{\Gamma \vdash M : \tau \rightarrow \sigma && \Gamma \vdash N : \tau} \qquad
        \infer[(x \notin \mathrm{dom}(\Gamma))]
                {\Gamma \vdash \lambda x^\tau . M : \tau \rightarrow \sigma}{\Gamma, x : \tau \vdash M : \sigma} \\
        \infer[(\alpha \notin \FV(\Gamma))]{\Gamma \vdash \Lambda \alpha . M : \forall \alpha : \sigma}{\Gamma \vdash M : \sigma} \qquad
        \infer{\Gamma \vdash M \tau : \sigma [\alpha := \tau]}{\Gamma \vdash M : \forall \alpha . \sigma}
    \end{gather*}
\end{definition}

\begin{example} Левая проекция:
    \begin{center}
    \begin{tabular}{l l l} \toprule
        & Просто типизированное $\lambda$-исчисление & Система $F$ \\ \midrule
        Тип & $\pi_1:\alpha\with\beta\rightarrow\alpha$ & $\pi_1:\forall \alpha . \forall \beta . \alpha \with \beta \rightarrow \alpha$ \\
        Выражение & $\pi_1 = \lambda p . p \comb T$ & $\pi_1 = \Lambda \varphi . \Lambda \psi . \lambda p^{\varphi\with\psi} . p \comb T$
        \\ \bottomrule
    \end{tabular}
    \end{center}
В системе $F$ явно указывается, что типы элементов пары могут быть любыми, как и в типе проекции, так и в её выражении.
\end{example}

\begin{definition}[$\beta$-редукция в $F$] \ 
    \begin{enumerate}
        \item Типовая редукция: $\left(\Lambda \alpha . M^\sigma\right) \tau \rightarrow_\beta M[\alpha:=\tau] : \sigma[\alpha := \tau]$
        \item Классическая $\beta$-редукция: $\left(\lambda x^\sigma.M\right)^{\sigma\rightarrow\tau} X \rightarrow_\beta M [x:=X] : \tau$
    \end{enumerate}
\end{definition}

\begin{theorem}[изоморфизм Карри-Ховарда для системы $F$]
    $\Gamma \vdash_F M :\tau$ т.и.т.т., когда $\abs{\Gamma} \vdash \tau$ в интуиционистском исчислении предикатов второго порядка.
\end{theorem}

\begin{theorem}
    $F$ \hyperref[strong-normalization]{сильно нормализуемо}.
\end{theorem}

\todo упражнения $\alpha\with\beta$ и тип чёрчевского нумерала
$\forall\alpha.(\alpha\rightarrow\alpha)\rightarrow(\alpha\rightarrow\alpha)$
