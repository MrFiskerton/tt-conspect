\section{Связь с программированием}

\epigraph{Помните, что в $\lambda$-исчислении нет смысла? Здесь смысл отрицательный, скорее.}{Д.Г.}

3 задачи:
\begin{enumerate}[label=(\alph*)]
    \item Проверка типа: верно ли $\Gamma \vdash M : \sigma$?
    \item Вывод типа: $? \vdash M : ?$
    \item Обитаемость типа: $? \vdash ? : \sigma$
\end{enumerate}

\subsection{Вывод типа}

\begin{definition}[Алгебраический терм] \ \\
    \begin{gather*}
        \begin{bnf}
            A ::= x | f\left(A, \ldots, A\right)
        \end{bnf} \\
        (x \in X)
    \end{gather*}
\end{definition}

Уравнение в алгебраических термах: $A = A$.

\begin{definition}[$S$-подстановка]
    \[
        S : X \rightarrow A
    \]
    $S$ "--- id почти везде. (везде кроме конечного количества) \\
    $S : A \rightarrow A$ "--- естественное обобщение. $A_1, \ldots, A_n$ "--- термы.
    $S\left(f\left(A_1, \dots, A_n\right)\right) = f\left(S(f_1), \ldots, S(f_n)\right)$
\end{definition}

\begin{definition}
    $S$ "--- решение уравнения $P=Q$, если $S(P)=S(Q)$ (S "--- унификатор).
\end{definition}
\begin{definition}
    $(S \circ T)(A) = S(T(A))$
\end{definition}
Задача решения уравнение в алгебраических термах "--- унификация.
\begin{definition}
    Существует $S$: $T = S \circ \texttt{U}$, $T$ "--- частный случай $\texttt{U}$.
\end{definition}
\begin{definition}
    Наибольший общий унификатор $\texttt{U}$:
    \begin{enumerate}
        \item $\texttt{U}(A)=\texttt{U}(B)$.
        \item Если $\texttt{U}(A)=\texttt{U}(B)$, то существует $S$: $T = S \circ \texttt{U}$.
    \end{enumerate}
\end{definition}

\begin{definition}
    Назовём две системы эквивалентными, если они имеют одно решение.
\end{definition}

\begin{definition}
    Назовём систему несовместной, если выполнено одно из условий:
    \begin{enumerate}
        \item в ней есть уравнение вида $f(\ldots)=g(\ldots)$.
        \item в ней есть уравнение вида $x = \ldots x \ldots$.
    \end{enumerate}
\end{definition}

\begin{statement}
    Для любой системы
    \[
        \begin{cases} A_1 = B_1 \\ \vdots \\ A_n = B_n \end{cases}
    \]
    найдётся эквивалентная ей система из одного уравнения:
    \[
        f(A_1, \ldots, A_n) = f(B_1, \ldots, B_n)
    \]
    ($f$ "--- новый символ).
\end{statement}

\begin{definition}
    Назовём систему разрешённой, если:
    \begin{enumerate}
        \item все уравнения имеют вид $x = A$,
        \item если $x_i = A_i$, то
        \begin{enumerate}[label=(\alph*)]
            \item $x_i$ не принадлежит никакому $A_j$.
            \item $x_i \neq x_j$, если $j \neq j$.
        \end{enumerate}
        %\item все переменные в левой части встречаются однократно.
    \end{enumerate}
\end{definition}

Решение по системе в разрешённой форме строится так:
\[
    S(x_i)=A_i
\]
\todo

\begin{statement}
    $S$ "--- наибольший общий унификатор.
\end{statement}

\begin{statement}
    Несовместная система не имеет решений.
\end{statement}

Рассмотрим следующие 4 преобразования:
\begin{enumerate}[label=(\alph*)]
    \item $T=x$, где $T$ не переменная $\Rightarrow$ $x=T$
    \item $x=x$ $\Rightarrow$ $\varepsilon$
    \item $s$
\end{enumerate}

бе
