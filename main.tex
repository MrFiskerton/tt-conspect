\documentclass{article}

\usepackage{amsmath}
\usepackage{amssymb}
\usepackage{amsthm}
\usepackage{mathtext}
\usepackage[T1,T2A]{fontenc}
\usepackage[utf8]{inputenc}
\usepackage[russian]{babel}
%\usepackage{geometry}
\usepackage[left=2cm,right=2cm,top=2cm,bottom=2cm,bindingoffset=0cm]{geometry}
\usepackage[mathscr]{euscript}
\usepackage{microtype}
\usepackage{bnf}
\usepackage{enumitem}
\usepackage{bm}
\usepackage{listings}
\usepackage{cancel}
\usepackage{proof}
%\setmainfont[Ligatures=TeX,SmallCapsFont={Times New Roman}]{Palatino Linotype}

\selectlanguage{russian}

%http://tex.stackexchange.com/questions/31526/macro-for-left-and-right/58641#58641

\title{Теория типов}
\author{Человек, который поспорил на 2 торта $\heartsuit$}
%Человек, который три семестра собирался верстать конспекты \\
%Человек, который иногда вспоминает, что в ХПИ не так уж и плохо \\
%Человек, который всегда рад помочь $\heartsuit$}
\date{}

\begin{document}

\theoremstyle{definition}
\newtheorem*{definition}{Определение}
\theoremstyle{plain}
\newtheorem{theorem}{Теорема}[section]
\newtheorem{axiom}{Аксиома}
\newtheorem{lemma}[theorem]{Лемма}
\newtheorem{statement}[theorem]{Утверждение}
\newtheorem{corollary}[theorem]{Следствие}
\theoremstyle{remark}
\newtheorem*{example}{Пример}

\lstset{language=C++}

\newcommand{\todo}{\textsc{\textbf{TODO}}}

\maketitle
\tableofcontents
\newpage

\section{$\lambda$-исчисление}

\subsection{Введение}
\epigraph{Смысла в этом нет.}{Д.Г.}

\begin{definition}[$\lambda$-выражение]
    $\lambda$-выражение "--- выражение, удовлетворяющее грамматике:
    \begin{bnf}
    \begin{alignat*}{3}
        \Lambda ::= & \lambda{}x.\Lambda{} \qquad && (абстракция) \\
                  | & \Lambda{}\Lambda{}          && (аппликация) \\
                  | & x                                           \\
                  | & \left(\Lambda\right)
    \end{alignat*}
    \end{bnf}
    \begin{enumerate}[label=(\asbuk*)]
        \item аппликация левоассоциативна
        \item абстракция распространяется как можно дальше вправо
    \end{enumerate}
\end{definition}

\begin{example}
    $((\lambda{} z.(z (y z))) (z x) z) = (\lambda{} z.z (y z)) (z x) z$
\end{example}
%[(а)]
%[(б)]
%[(в)]

Есть понятия связанного и свободного вхождения переменной (аналогично ИП).
$\lambda{}x.A$ связывает все свободные вхождения $x$ в $A$.
Договоримся, что:
\begin{enumerate}[label=(\asbuk*)]
    \item Переменные "--- $x$, $a$, $b$, $c$.
    \item Термы (части $\lambda$-выражения) "--- $X$, $A$, $B$, $C$.
    \item Фиксированные переменные обозначаются буквами из начала алфавита, метапеременные "--- из конца.
\end{enumerate}

\begin{definition}[$\alpha$-эквивалентность]
    $A$ и $B$ называются $\alpha$-эквивалентными ($A=_{\alpha}B$), если выполнено одно из следующих условий:
    \begin{enumerate}
        \item $A\equiv{}x$ и $B\equiv{}x$.
        \item $A\equiv{}\lambda{}x.P$, $B\equiv{}\lambda{}y.Q$ и $P_{[x:=t]}=_{\alpha}Q_{[y:=t]}$, где $t$ "--- новая переменная.
        \item $A\equiv{}PQ$, $B\equiv{}RS$ и $P=_{\alpha}R$, $Q=_{\alpha}S$.
    \end{enumerate}
\end{definition}

\begin{example}
    $\lambda{}x.\lambda{}y.xy=_{\alpha}\lambda{}y.\lambda{}x.yx$.
    \begin{proof}
        \begin{alignat*}{2}
            \lambda{}y.ty=_{\alpha}\lambda{}x.tx &\implies \lambda{}x.\lambda{}y.xy=_{\alpha}\lambda{}y.\lambda{}x.yx \\
            tz=_{\alpha}tz &\implies \lambda{}y.ty=_{\alpha}\lambda{}x.tx
        \end{alignat*}
        $tz=_{\alpha}tz$ верно по третьему условию.
    \end{proof}
\end{example}

\begin{definition}[$\beta$-редекс]
    Терм вида $\left(\lambda{}a.A\right)B$ называется $\beta$-редексом.
\end{definition}

\begin{example}
    В выражении
    $
        (
            \lambda{}f.
                \underset{A_2}{\underline{
                    (\lambda{}x.\overset{A_1}{\overline{f(xx)}})
                    \overset{B_1}{\overline{(\lambda{}x.f(xx))}}
                }}
        )\underset{B_2}{\underline{g}}
    $ два $\beta$-редекса.
\end{example}

\begin{definition}
    Множество $\lambda$-термов $\bm{\Lambda}$ назовём множеством классов эквивалентности $\Lambda$ по $(=_{\alpha})$.
\end{definition}

\begin{definition}[$\beta$-редукция]
    $A\rightarrow_{\beta}B$ (состоят в отношении $\beta$-редукции), если выполняется одно из условий:
    \begin{enumerate}
        \item $A\equiv{}PQ$, $B\equiv{}RS$ и
        \begin{alignat*}{3}
            &\text{либо } P\rightarrow_{\beta}R  &&\text{ и } Q=_{\alpha}S \\
            &\text{либо } P=_{\alpha}R           &&\text{ и } Q\rightarrow_{\beta}S
        \end{alignat*}
        \item $A\equiv{}\lambda{}x.P$, $B\equiv{}\lambda{}x.Q$, $P\rightarrow_{\beta}Q$ ($x$ из какого-то класса из $\bm{\Lambda}$).
        \item $A\equiv{}(\lambda{}x.P)Q$, $B\equiv{}P_{[x:=Q]}$, $Q$ свободно для подстановки в $P$ вместо $x$.
    \end{enumerate}
\end{definition}

\subsection{Числа Чёрча}
\epigraph{Хотите знать, что такое истина?}{Д.Г.}
%\paragraph{Итак, лулзы.} Хотите знать, что такое истина?

\newcommand{\T}{\mathrm{T}}
\newcommand{\F}{\mathrm{F}}
\newcommand{\Not}{\mathrm{Not}}
\begin{alignat*}{2}
    \T   &= \lambda{}x\lambda{}y.x \\
    \F   &= \lambda{}x\lambda{}y.y \\
    \Not &= \lambda{}a.a\F\T
\end{alignat*}

Похоже на тип boolean, не правда ли?
\begin{example}
    \[
        \Not\ \T = (\lambda{}a.a\F\T)\T \rightarrow_{\beta}
            \T\F\T = (\lambda{}x.\lambda{}y.x)\F\T \rightarrow_{\beta}
            (\lambda{}y.\F)\T \rightarrow_{\beta}
            \F
    \]
\end{example}

Можно продолжить:
\begin{alignat*}{2}
    \mathrm{And} &= \lambda{}a.\lambda{}b.ab\mathrm{F} \\
    \mathrm{Or}  &= \lambda{}a.\lambda{}b.a\mathrm{T}b
\end{alignat*}

Попробуем определить числа:
\begin{definition}[Чёрчевский нумерал]
\[
    \overline{n}=\lambda{}f.\lambda{}x.f^{n}x \text{,\quadгде\quad}
    f^{n}x=
    \begin{cases}
        f\left(f^{n-1}x\right) &, n > 0 \\
        x                      &, n = 0
    \end{cases}
\]
\end{definition}

\begin{example}
\[
    \overline{3} = \lambda f . \lambda x . f (f (f x))
\]
\end{example}

Несложно определить прибавление единицы к такому нумералу:
\[
    (+1) = \lambda{}n.\lambda{}f.\lambda{}x.f(nfx) \\
\]
\begin{example}
    \[
        (+1) \overline{1} =
        (\lambda n . \lambda f . \lambda x . f (n f x)) (\lambda f . \lambda x . f x) \rightarrow_{\beta}
        \lambda f . \lambda x . f ((\lambda f . \lambda x . f x) f x) \twoheadrightarrow_{\beta}
        \lambda f . \lambda x . f (f x) =
        \overline{2}
    \]
\end{example}

\begin{definition}[$\eta{}$-эквивалентность]
    \[
        \lambda x . f x =_{\eta} f
    \]
\end{definition}
Аналог из C++: если \lstinline$int f(int x)$, то результат её вычисления равен результату вычисления\\
    \lstinline$[ ] (int x) { return f(x); }$ .

Арифметические операции:
\begin{alignat*}{2}
    \mathrm{IsZero} &= \lambda{}n.n(\lambda{}x.\F)\T \\
    \mathrm{IsEven} &= \lambda{}n.n\ \Not\ \T \\
    \mathrm{Add} &= \lambda{}a.\lambda{}b.\lambda{}f.\lambda{}x.a f (b f x) \\
    \mathrm{Mul} &= \lambda{}a.\lambda{}b.a (\mathrm{Add}\ b) \overline{0} \\
    \mathrm{Pow} &= \lambda{}a.\lambda{}b.b (\mathrm{Mul}\ a) \overline{1} \\
    \mathrm{Pow}^{*} &= \lambda{}a.\lambda{}b.b a
\end{alignat*}

Для того, чтобы определить $(-1)$, сначала определим "пару":
\begin{alignat*}{2}
    \left<a,b\right> &= \lambda f.f a b \\
    \mathrm{First} &= \lambda p . \T p \\
    \mathrm{Second} &= \lambda p . \F p
\end{alignat*}

$n$ раз применим функцию $f\left(\left<a,b\right>\right) = \left<b,b+1\right>$ и возьмём первый элемент пары:
\[
    (-1) = \lambda n . \mathrm{First} \left(n\ (\lambda p . \left<\left(\mathrm{Second}\ p\right), (+1)\ (\mathrm{Second}\ p)\right>)\
    \left<\overline{0},\overline{0}\right>\right)
\]

Введём сокращение записи:
\[
    \lambda x y . A = \lambda x . \lambda y . A
\]

\begin{definition}[Нормальная форма] \mbox{} \\
    Терм $A$ "--- нормальная форма (н.ф.), если в нём нет $\beta$-редексов. \\
    Нормальной формой $A$ называется такой $B$, что $A \twoheadrightarrow_{\beta} B$, $B$ "--- н.ф. \\
    $\twoheadrightarrow_{\beta}$ "--- транзитивно-рефлексивное замыкание $\rightarrow_{\beta}$.
\end{definition}

\begin{statement}
    Существует $\lambda$-выражение, не имеющее н.ф.
\end{statement}

\begin{definition}[Комбинатор]
    Комбинатор "--- $\lambda$-выражение без свободных переменных.
\end{definition}

\begin{definition}
    \begin{gather*}
        \Omega = \omega \omega \\
        \omega = \lambda x . x x
    \end{gather*}
\end{definition}

$\Omega$ не имеет нормальной формы.

\begin{definition}[Комбинатор неподвижной точки]
    \[
        Y = \lambda f . (\lambda x . f (x x)) (\lambda x . f (x x))
    \]
\end{definition}

\begin{definition}[$\beta$-эквивалентность]
    $A=_{\beta}B$, если $\exists C : C \twoheadrightarrow_{\beta} A, C \twoheadrightarrow_{\beta}B$
\end{definition}

\begin{statement}
    \[
        Yf =_{\beta} f(Yf)
    \]
\end{statement}

\begin{proof} (на лекции не давалось)
    \begin{align*}
        Yf &=_{\beta} (\lambda f . (\lambda x . f (x x)) (\lambda x . f (x x))) f \\
           &=_{\beta} (\lambda x . f (x x)) (\lambda x . f (x x)) \\
           &=_{\beta} f ((\lambda x . f (x x)) (\lambda x . f (x x))) \\
           &=_{\beta} f (Y f)
    \qedhere
    \end{align*}
\end{proof}

Таким образом, с помощью $Y$-комбинатора можно определять рекурсивные функции.
\begin{example}
    \[
        \mathrm{Fact} = Y (\lambda{} f n . \mathrm{IsZero}\ n\ \overline{1}\ (\mathrm{Mul}\ n\ (f\ (-1)\ n)))
    \]
\end{example}

\todo % пояснение
 \newpage
\subsection{\texorpdfstring{Ромбовидное свойство и параллельная редукция}{Diamond property and parallel reduction}}

\begin{definition}[Ромбовидное свойство (diamond)]
    $G$ обладает ромбовидным свойством, если какие бы ни были $a$, $b$, $c$, что $aGb$, $aGc$, $b \ne c$, найдётся такое $d$, что $bGd$ и $cGd$.
\end{definition}

\begin{example}
    $(<)$ на натуральных числах обладает ромбовидным свойством.
    $(>)$ на натуральных числах не обладает ромбовидным свойством.
\end{example}

$\beta$-редукция не обладает ромбовидным свойством.
\begin{example} % TODO запилить картинку
    \begin{gather*}
        a = (\lambda x . x x)(Ia) \\
        a \rightarrow_{\beta} (Ia)(Ia) = b\\
        a \rightarrow_{\beta} (\lambda x . x x) a = c \\
        b \rightarrow_{\beta} (Ia)a \rightarrow_{\beta} aa \\
        b \rightarrow_{\beta} a(Ia) \rightarrow_{\beta} aa \\
        c \rightarrow_{\beta} aa
    \end{gather*}
    Нет $d$, что $b \rightarrow_{\beta} d$ и $c \rightarrow_{\beta} d$.
\end{example}

\begin{theorem}[Чёрча-Россера] \label{church-rosser}
    $\beta$-редуцируемость обладает ромбовидным свойством.
\end{theorem}

\begin{lemma}
    Если $R$ обладает ромбовидным свойством, то $R^{*}$ обладает ромбовидным свойством.
\end{lemma}

\begin{proof} (Упражнение) \todo % TODO
    \begin{enumerate}
        \item $M_{1}RN_{1}$ и $M_{1}RM_{2}...M_{n-1}RM_{n}$ $\Rightarrow$ есть $N_{2}...N_{n}$: \\
            $N_{1}RN_{2}...N_{n-1}RN_{n}$ и $M_{n}RN_{n}$.
        \item Покажем ромбовидное свойство.
        \qedhere
    \end{enumerate}
\end{proof}

\begin{definition}[Параллельная $\beta$-редукция]
    $A \rightrightarrows_{\beta} B$
    \begin{enumerate}
        \item $A =_{\beta} B$, то $A \rightrightarrows_{\beta}B$
        \item $A \rightrightarrows_{\beta} B$, то $\lambda x.A \rightrightarrows_{\beta} \lambda x . B$
        \item $P \rightrightarrows_{\beta} Q$ и $R \rightrightarrows_{\beta} S$, то $PR \rightrightarrows_{\beta} QS$
        \item $(\lambda x . P) Q \rightrightarrows_{\beta} R_{[x:=S]}$, если 
            $P \rightrightarrows_{\beta}R$ и $Q \rightrightarrows_{\beta} S$.
    \end{enumerate}
\end{definition}

\begin{statement} \label{st-star}
    $(\rightrightarrows_{\beta})$ обладает ромбовидным свойством.
\end{statement}

\begin{proof}
    (Упражнение) \todo % TODO
\end{proof}

\begin{statement} \label{st-A}
    Если $A \rightarrow_{\beta} B$, то $A \rightrightarrows_{\beta} B$.
\end{statement}

\begin{statement} \label{st-B}
    Если $A \rightrightarrows_{\beta} B$, то $A \twoheadrightarrow_{\beta} B$.
\end{statement}

\begin{proof}
    (Упражнение) \todo % TODO
\end{proof}

При этом, обратное не всегда верно.

\begin{example}
    \begin{gather*}
        (\lambda x . x x) (\lambda x . x x x) \twoheadrightarrow_{\beta} (\lambda x . x x x)(\lambda x . x x x)(\lambda x . x x x) \\
        (\lambda x . x x) (\lambda x . x x x) \cancel{\rightrightarrows_{\beta}} (\lambda x . x x x)(\lambda x . x x x)(\lambda x . x x x)
    \end{gather*}
\end{example}

\begin{statement} \label{st-C}
    Из \ref{st-A} и \ref{st-B} следует, что $(\rightarrow_{\beta})^{*} = (\rightrightarrows_{\beta})^{*}$.
\end{statement}

\begin{proof}
    Теорема \nameref{church-rosser} следует из \ref{st-star} и \ref{st-C}.
\end{proof}

\begin{corollary}
    Нормальная форма для $\lambda$-выражения единственна, если существует.
\end{corollary}

\begin{theorem}[Тезис Чёрча]
    Если функция вычислима с помощью механического аппарата, то она вычислима с помощью $\lambda$-выражения.
\end{theorem}

\subsection{\texorpdfstring{Порядок редукции}{Order of reduction}}
\epigraph{"<Завтра! Завтра! Не сегодня!"> "--- так ленивцы говорят.}{Das deutsches Sprichwort}

\begin{definition}
    \begin{align*}
        K &= \lambda x \lambda y . x \\
        I &= \lambda x . x \\
        S &= \lambda x y z . x z (y z)
    \end{align*}
\end{definition}
$I$ выражается через $S$ и $K$: $I = S K K$.

\begin{statement}
    Пусть $A$ "--- замкнутое $\lambda$-выражение. Тогда найдётся выражение $T$, состоящее только из $S$, $K$, что $A =_{\beta}T$.
\end{statement}

\begin{example}
    тут какой-то пример с омегой, подскажите чё там было, \todo % TODO
\end{example}

\begin{definition}[Нормальный порядок редукции]
    Нормальным порядком редукции называется редукция самого левого $\beta$-редекса.
\end{definition}
"<Ленивые вычисления"> (ну, почти, в ленивых ещё есть меморизация)

\begin{definition}[Аппликативный порядок редукции]
    Самый левый из самых вложенных.
\end{definition}
"<Энергичные вычисления">

\begin{statement}
    Если нормальная форма существует, она может быть достигнута нормальным порядком редукции.
\end{statement}

\subsection{\texorpdfstring{Парадокс Карри}{Curry's paradox}}

\epigraph{Если это утверждение верно, то русалки существуют.}

Попробуем построить логику на основе $\lambda$-исчисления.
Введём комбнатор-импликацию, обозначим $(\supset)$. Введём M.P. и правила:
\begin{enumerate}
    \item $A \supset A$
    \item $(A \supset (A \supset B)) \supset (A \supset B)$
    \item $A =_{\beta} B$, тогда $A \supset B$
\end{enumerate}

Покажем, как в полученной логике можно доказать любое утверждение.
Введём обозначение: $Y_{\supset a} \equiv Y (\lambda t . t \supset a) =_{\beta} Y (\lambda t . t \supset a) \supset a$.

\begin{tabular}{lll}
    1) & $Y_{\supset a} \supset Y_{\supset a}$ & (схема аксиом) \\
    2) & $Y_{\supset a} \supset (Y_{\supset a} \supset a)$ & (можно доказать) \\
    3) & $(Y_{\supset a} \supset Y_{\supset a} \supset a) \supset (Y_{\supset a} \supset a)$ & (схема аксиом) \\
    4) & $Y_{\supset a} \supset a$ & (M.P.) \\
    5) & $(Y_{\supset a} \supset a) \supset Y_{\supset a}$ & (третье правило) \\
    6) & $Y_{\supset a}$ & (M.P.) \\
    7) & $a$ & (M.P.)
\end{tabular}

Получается, что данная логика противоречива.

\subsection{\texorpdfstring{Импликационный фрагмент ИИВ}{Implication fragment of intuitionistic logic}}

\begin{definition}[импликационный фрагмент ИИВ]
    Рассмотрим интуиционистское исчисление высказываний.
    \begin{enumerate}
        \item Введём схему аксиом:
        \[
            \infer{\Gamma, \varphi \vdash \varphi}{}
        \]
        \item Правило введения импликации:
        \[
            \infer{\Gamma \vdash \varphi \rightarrow \psi}{\Gamma, \varphi \vdash \psi}
        \]
        \item И правило удаления импликации:
        \[
            \infer{\Gamma \vdash \psi}{\Gamma \vdash \varphi \rightarrow \psi & \Gamma \vdash \varphi}
        \]
    \end{enumerate}

    Мы построили импликационный фрагмент ИИВ (и.ф.и.и.в).
\end{definition}

\begin{example} Докажем $\varphi \rightarrow \psi \rightarrow \varphi$:
\[
    \infer[(2)]
        { \vdash \varphi \rightarrow (\psi \rightarrow \varphi) }
        { \infer[(2)]
            { \varphi \vdash \psi \rightarrow \varphi }
            { \infer[(1)]
                { \varphi, \psi \vdash \varphi}
                {}
            }
        }
\]
\end{example}

\begin{theorem}
    И.ф.и.и.в полон в моделях Крипке, то есть $\Gamma \vdash \varphi$ т.и.т.т.,
    когда для любой модели крипке $C$ из $\Vdash_C \Gamma$ следует $\Vdash_C \varphi$.
\end{theorem}

\begin{proof}
    Рассмотрим модель Крипке вида $W = \left\{\Delta : \Gamma \subseteq \Delta, \Delta\text{ замкнуто относительно }\vdash\right\}$,
    $\Gamma \leq \Delta$ если $\Gamma \subseteq \Delta$.
    Индукцией по структуре $\varphi$ покажем, что $\Delta \Vdash \varphi$ т.и.т.т., когда $\Delta \vdash \varphi$:
    \begin{enumerate}
        \item Пусть $\varphi \equiv x$ "--- переменная. Тогда $\Gamma \vdash \varphi$ эквивалентно $x \in \Gamma$, что эквивалентно $\Vdash x$ (по определению).
        \item Пусть $\varphi \equiv \alpha \rightarrow \beta$.
        \begin{enumerate}[label=(\asbuk*)]
            \item Пусть $\Delta \vdash \varphi$.
            Рассмотрим такое $\Delta'$, что $\Delta \leq \Delta'$ и $\Delta' \Vdash \alpha$.
        \end{enumerate}
        ой всё \todo
    \end{enumerate}
\end{proof}

\begin{corollary}
    И.ф.и.и.в замкнут относительно выводимости.
\end{corollary}
Если некоторое утверждение выводится в ИИВ ($\vdash_{и} \varphi$) и содержит только импликации,
то оно выводится и в и.ф.и.и.в. ($\vdash_{и \rightarrow} \varphi$).
 \newpage

\end{document}
