\documentclass{article}

\usepackage{amsmath}
\usepackage{amssymb}
\usepackage{amsthm}
\usepackage{mathtext}
\usepackage{mathtools}
\usepackage[T1,T2A]{fontenc}
\usepackage[utf8]{inputenc}
%\usepackage{geometry}
\usepackage[left=2cm,right=2cm,top=2cm,bottom=2cm,bindingoffset=0cm]{geometry}
\usepackage{microtype}
\usepackage{bnf}
\usepackage{enumitem}
\usepackage{bm}
%\usepackage{listings}
\usepackage{minted}
\usepackage{cancel}
\usepackage{proof}
\usepackage{epigraph}
\usepackage{titlesec}
\usepackage[dvipsnames]{xcolor}
\usepackage{stmaryrd}
\usepackage{cellspace}
\usepackage{cmll}
\usepackage{multirow}
\usepackage{booktabs}
\usepackage{tikz}
\usepackage{caption}
\usepackage{wrapfig}

\usepackage[hidelinks]{hyperref}

%\setmainfont[Ligatures=TeX,SmallCapsFont={Times New Roman}]{Palatino Linotype}

\usepackage[russian]{babel}
\selectlanguage{russian}

\hypersetup{%
    colorlinks=true,
    linkcolor=blue
}

\AddEnumerateCounter{\Asbuk}{\@Asbuk}{\CYRM}
\AddEnumerateCounter{\asbuk}{\@asbuk}{\cyrm}

%http://tex.stackexchange.com/questions/31526/macro-for-left-and-right/58641#58641

\DeclareMathOperator{\rank}{rk}
\DeclareMathOperator{\dom}{dom}
\DeclareMathOperator{\inj}{inj}
\DeclareMathOperator{\FV}{FV}

\DeclareMathAlphabet{\mathpzc}{OT1}{pzc}{m}{it}

\setlength\heavyrulewidth{1pt}

\title{\textsc{Теория типов}}
\author{\url{https://github.com/artemohanjanyan/tt-conspect}}
\date{}

\begin{document}

\theoremstyle{definition}
\newtheorem*{definition}{Определение}
\theoremstyle{plain}
\newtheorem{theorem}{Теорема}[section]
\newtheorem{axiom}{Аксиома}
\newtheorem{lemma}[theorem]{Лемма}
\newtheorem{statement}[theorem]{Утверждение}
\newtheorem{corollary}[theorem]{Следствие}
\theoremstyle{remark}
\newtheorem*{example}{Пример}
\newtheorem{property}[theorem]{Свойство}

\usetikzlibrary{arrows.meta}

%\lstset{language=C++}

\newcommand{\todo}{\textsc{\textbf{TODO}}\ }
%\newcommand{\comb}[1]{\mathit{#1}}
\newcommand{\comb}[1]{\operatorname{\mathpzc{#1}}}
\newcommand{\combl}[1]{\operatorname{#1}}
\newcommand{\abs}[1]{\left|#1\right|}
\newcommand{\set}[1]{\left\{#1\right\}}
\newcommand{\xx}{ч}
\newcommand{\rr}{к}
\newcommand{\case}[3]{\mathinner{\mathtt{case}}~#1~#2~#3}
\newcommand{\pack}[2]{\mathinner{\mathtt{pack}}~#1~\mathinner{\mathtt{to}}~#2}
\newcommand{\abstype}[4]{\mathinner{\mathtt{abstype}}~#1~\mathinner{\mathtt{with}}~#2~\mathinner{\mathtt{in}}~#3~\mathinner{\mathtt{is}}~#4}
\newcommand{\lett}[2]{\mathinner{\mathtt{let}}~#1~\mathinner{\mathtt{in}}~#2}
\newcommand{\pair}[1]{\left<#1\right>}
\newcommand{\ppair}[1]{\left<\!\left<#1\right>\!\right>}
\newcommand{\caseb}[2]{\mathinner{\mathrm{case}}~#1~\mathinner{\mathrm{of}}~#2}
\newcommand{\casec}[3]{\mathinner{\mathrm{case}}~#1~\mathinner{\mathrm{of}}~#2;~#3}

\newcommand{\reduction}[1]{{\color{OrangeRed}#1}}

\newcommand{\inferspacing}{\setlength{\jot}{1em}}

\newcommand{\inl}[0]{\mathinner\mathrm{inl}}
\newcommand{\inr}[0]{\mathinner\mathrm{inr}}
\newcommand{\fst}[0]{\mathinner\mathrm{fst}}
\newcommand{\snd}[0]{\mathinner\mathrm{snd}}

\newcommand{\fuck}[0]{%
\begin{tikzpicture}%
\draw (0,0) -- (1ex,1ex);%
\draw (0,0) -- (0,2ex);%
\draw (0,2ex) -- (1ex,1ex);%
\draw (0,1ex) -- (1ex,1ex);%
\end{tikzpicture}%
}

\fuck

\maketitle
\tableofcontents
%\newpage
\newcommand{\sectionbreak}{\clearpage}

%\renewcommand{\arraystretch}{1.0}
\setlength\cellspacetoplimit{5pt}
\setlength\cellspacebottomlimit{5pt}
\setlength{\inferLineSkip}{4pt}

\section{\texorpdfstring{$\lambda$-исчисление}{Lambda calculus}}

\subsection{\texorpdfstring{Введение}{Introduction}}
\epigraph{Смысла в этом нет.}{Д.Г.}

\begin{definition}[$\lambda$-выражение]
    $\lambda$-выражение "--- выражение, удовлетворяющее грамматике:
    \begin{bnf}
    \begin{alignat*}{3}
        \Lambda ::= & \lambda{}x.\Lambda{} \qquad && (абстракция) \\
                  | & \Lambda{}\Lambda{}          && (аппликация) \\
                  | & x                                           \\
                  | & \left(\Lambda\right)
    \end{alignat*}
    \end{bnf}
    \begin{enumerate}[label=(\asbuk*)]
        \item Аппликация левоассоциативна.
        \item Абстракция распространяется как можно дальше вправо.
    \end{enumerate}
\end{definition}

\begin{example}
    $((\lambda{} z.(z (y z))) (z x) z) = (\lambda{} z.z (y z)) (z x) z$
\end{example}

Есть понятия связанного и свободного вхождения переменной (аналогично ИП).
$\lambda{}x.A$ связывает все свободные вхождения $x$ в $A$.
Договоримся, что:
\begin{enumerate}[label=(\asbuk*)]
    \item Переменные "--- $x$, $a$, $b$, $c$.
    \item Термы (части $\lambda$-выражения) "--- $X$, $A$, $B$, $C$.
    \item Фиксированные переменные обозначаются буквами из начала алфавита, метапеременные "--- из конца.
\end{enumerate}

На самом деле, смысл в этом есть, $\lambda$-выражение можно понимать как функцию.
Абстракция "--- это функция с аргументом, аппликация "--- это передача аргумента.

\begin{definition}[$\alpha$-эквивалентность]
    $A$ и $B$ называются $\alpha$-эквивалентными ($A=_{\alpha}B$), если выполнено одно из следующих условий:
    \begin{enumerate}
        \item $A\equiv{}x$ и $B\equiv{}x$.
        \item $A\equiv{}\lambda{}x.P$, $B\equiv{}\lambda{}y.Q$ и $P [x\coloneqq{}t] =_{\alpha}Q [y\coloneqq{}t]$, где $t$ "--- новая переменная.
        \item $A\equiv{}PQ$, $B\equiv{}RS$ и $P=_{\alpha}R$, $Q=_{\alpha}S$.
    \end{enumerate}
\end{definition}

То есть два выражения $\alpha$-эквивалентны, если они равны с точностью до переименования абстракций.
В C++ между функциями \mintinline{C++}{[](int x){return f(x);}} и \mintinline{C++}{[](int y){return f(y);}} нет разницы.

\begin{example}
    $\lambda{}x.\lambda{}y.xy=_{\alpha}\lambda{}y.\lambda{}x.yx$.
    \begin{proof} Согласно второму правилу следующие утверждения верны:
        \begin{alignat*}{2}
            \lambda{}y.ty=_{\alpha}\lambda{}x.tx &\implies \lambda{}x.\lambda{}y.xy=_{\alpha}\lambda{}y.\lambda{}x.yx \\
            tz=_{\alpha}tz &\implies \lambda{}y.ty=_{\alpha}\lambda{}x.tx
        \end{alignat*}%
        $tz=_{\alpha}tz$ верно по третьему условию.
    \end{proof}
\end{example}

\begin{definition}[$\beta$-редекс]
    Терм вида $\left(\lambda{}a.A\right)B$ называется $\beta$-редексом.
\end{definition}

\begin{example}
    В выражении
    $
        (
            \lambda{}f.
                \underset{A_2}{\underline{
                    (\lambda{}x.\overset{A_1}{\overline{f(xx)}})
                    \overset{B_1}{\overline{(\lambda{}x.f(xx))}}
                }}
        )\underset{B_2}{\underline{g}}
    $ два $\beta$-редекса.
\end{example}

\begin{definition}
    Множество $\lambda$-термов $\boldsymbol{\Lambda}$ назовём множеством классов эквивалентности $\Lambda$ по $(=_{\alpha})$.
\end{definition}

\begin{definition}[$\beta$-редукция]
    $A\rightarrow_{\beta}B$ (состоят в отношении $\beta$-редукции), если выполняется одно из условий:
    \begin{enumerate}
        \item $A\equiv{}PQ$, $B\equiv{}RS$ и либо $P\rightarrow_{\beta}R$ и $Q=_{\alpha}S$,
            либо $P=_{\alpha}R$ и $Q\rightarrow_{\beta}S$.
        \item $A\equiv{}\lambda{}x.P$, $B\equiv{}\lambda x.Q$, $P\rightarrow_{\beta}Q$ ($x$ из какого-то класса из $\boldsymbol{\Lambda}$).
        \item $A\equiv{}(\lambda{}x.P)Q$, $B\equiv{}P [x\coloneqq{}Q]$, $Q$ свободно для подстановки в $P$ вместо $x$.
    \end{enumerate}
\end{definition}

$\beta$-редукция это вычисление функции, подстановка её аргумента.

\begin{example} $(\lambda x . x y) a \rightarrow_\beta a y$
\end{example}

\subsection{\texorpdfstring{Числа Чёрча}{Church numerals}}
\epigraph{Хотите знать, что такое истина?}{Д.Г.}

\newcommand{\T}{\comb{T}}
\newcommand{\F}{\comb{F}}
\newcommand{\Not}{\comb{Not}}
\begin{gather*}
    \T   = \lambda{}x\lambda{}y.x \qquad
    \F   = \lambda{}x\lambda{}y.y \\
    \Not = \lambda{}a.a\F\T
\end{gather*}

Похоже на тип \mintinline{Pascal}{boolean}, не правда ли?
\begin{example}
    \[
        \Not \T = (\reduction{\lambda{}a}.a\F\T)\reduction\T \rightarrow_{\beta}
            \T\F\T = (\reduction{\lambda{}x}.\lambda{}y.x)\reduction\F\T \rightarrow_{\beta}
            (\reduction{\lambda{}y}.\F)\reduction\T \rightarrow_{\beta}
            \F
    \]
\end{example}
Истина это функция, которая принимает два аргумента, и возвращает первый аргумент.
Ложь принимает два аргумента и возвращает второй.
Если в выражении $a \F \T$ терм $a$ является истиной, то результатом будет первый аргумент, $\F$. Если ложью "--- второй аргумент, $\T$.

Можно продолжить:
\[
    \comb{And} = \lambda{}a.\lambda{}b.ab\F \qquad
    \comb{Or}  = \lambda{}a.\lambda{}b.a\T b
\]
Разберём $\comb{And}$. Он принимает два аргумента.
Если первый аргумент это истина, то тогда результат $\comb{And}$ равен второму аргументу.
Если первый аргумент это ложь, то тогда результат $\comb{And}$ это ложь, и от второго аргумента не зависит.
Это и записано в выражении. $\comb{Or}$ разбирается аналогично.

Попробуем определить числа:
\begin{definition}[чёрчевский нумерал]
\[
    \overline{n}=\lambda{}f.\lambda{}x.f^{n}x \text{,\quadгде\quad}
    f^{n}x=
    \begin{cases}
        f\left(f^{n-1}x\right) &, n > 0 \\
        x                      &, n = 0
    \end{cases}
\]
\end{definition}

\begin{example}
\[
    \overline{3} = \lambda f . \lambda x . f (f (f x))
\]
\end{example}

Несложно определить прибавление единицы к такому нумералу:
\[
    (+1) = \lambda{}n.\lambda{}f.\lambda{}x.f(nfx) \\
\]
\begin{example}
    \[
        (+1) \overline{1} =
        (\reduction{\lambda n} . \lambda f . \lambda x . f (n f x)) \reduction{(\lambda f . \lambda x . f x)} \rightarrow_{\beta}
        \lambda f . \lambda x . f ((\reduction{\lambda f . \lambda x} . f x) \reduction{f x}) \twoheadrightarrow_{\beta}
        \lambda f . \lambda x . f (f x) =
        \overline{2}
    \]
\end{example}

\begin{definition}[$\eta{}$-эквивалентность]
    \[
        \lambda x . f x =_{\eta} f
    \]
\end{definition}
Точно так же результаты вычисления \mintinline{C++}{int f(int x)} и \mintinline{C++}{[](int x){return f(x);}} равны.

Арифметические операции:
\begin{align*}
    \comb{IsZero} &= \lambda{}n.n\mathinner{(\lambda{}x.\F)}\T &\qquad
    \comb{Add}    &= \lambda{}a.\lambda{}b.\lambda{}f.\lambda{}x.a \mathinner{f} (b \mathinner{f} x) &\qquad
    \comb{Pow}    &= \lambda{}a.\lambda{}b.b \mathinner{(\comb{Mul} a)} \mathinner{\overline{1}} \\
    \comb{IsEven} &= \lambda{}n.n \Not \T &\qquad
    \comb{Mul}    &= \lambda{}a.\lambda{}b.a \mathinner{(\comb{Add} b)} \mathinner{\overline{0}} &\qquad
    \comb{Pow'}   &= \lambda{}a.\lambda{}b.b a
\end{align*}

Для того, чтобы определить $(-1)$, сначала определим "пару":
\[
    \left<a,b\right> = \lambda f.f \mathinner{a} b \qquad
    \comb{First} = \lambda p . p \T \qquad
    \comb{Second} = \lambda p . p \F
\]%
затем $n$ раз применим функцию $f\left(\left<a,b\right>\right) = \left<b,b+1\right>$ и возьмём первый элемент пары:
\[
    (-1) = \lambda n . \comb{First}
        \left(n \mathinner{(\lambda p . \left<\left(\comb{Second} p\right), \mathinner{(+1)} (\comb{Second} p)\right>)}
        \left<\overline{0},\overline{0}\right>\right)
\]

Введём сокращение записи:
\[
    \lambda x y . A = \lambda x . \lambda y . A
\]

\begin{definition}[Нормальная форма] \mbox{} \\
    Терм $A$ "--- нормальная форма (н.ф.), если в нём нет $\beta$-редексов. \\
    Нормальной формой $A$ называется такой $B$, что $A \twoheadrightarrow_{\beta} B$, $B$ "--- н.ф. \\
    $\twoheadrightarrow_{\beta}$ "--- транзитивно-рефлексивное замыкание $\rightarrow_{\beta}$.
\end{definition}

\begin{statement}
    Существует $\lambda$-выражение, не имеющее н.ф.
\end{statement}

\begin{definition}[комбинатор]
    Комбинатор "--- $\lambda$-выражение без свободных переменных.
\end{definition}

\begin{definition}
\[
    \combl\Omega = \combl\omega \combl\omega \qquad
    \combl\omega = \lambda x . x x
\]
\end{definition}

$\combl\Omega$ не имеет нормальной формы.

\begin{definition}[комбинатор неподвижной точки]
    \[
        \comb Y = \lambda f . (\lambda x . f (x x)) (\lambda x . f (x x))
    \]
\end{definition}

\begin{definition}[$\beta$-эквивалентность]
    $A=_{\beta}B$, если $\exists C : C \twoheadrightarrow_{\beta} A, C \twoheadrightarrow_{\beta}B$
\end{definition}

\begin{statement}
    \[
        \comb Yf =_{\beta} f(\comb Yf)
    \]
\end{statement}

Из-за этого свойста комбинатор $\comb Y$ и получил своё название.

\begin{proof} Наивное $\beta$-редуцирование:
    \begin{align*}
        \comb Yf &=_{\beta} (\reduction{\lambda f} . (\lambda x . f (x x)) (\lambda x . f (x x))) \reduction f \\
                 &=_{\beta} (\reduction{\lambda x} . f (x x)) \reduction{(\lambda x . f (x x))} \\
                 &=_{\beta} f ((\lambda x . f (x x)) (\lambda x . f (x x))) \\
                 &=_{\beta} f (\comb Y f)
    \qedhere
    \end{align*}
\end{proof}

Таким образом, с помощью $\comb Y$-комбинатора можно определять рекурсивные функции.
\begin{example} Определим факториал.
\begin{align*}
    \comb{Fact'} &= \lambda{} f n . \comb{IsZero} n \mathinner{\overline{1}}
                    (\comb{Mul} n \mathinner{(f \mathinner{(\mathinner{(-1)} n))}}) \\
    \comb{Fact} &= \comb Y \comb{Fact'} \\
    \comb{Fact} \overline 3 &=_\beta \comb Y \comb{Fact'} \overline 3 =_\beta \comb{Fact'} \mathinner{(\comb Y \comb{Fact'})} \overline 3
                \\&=_\beta \comb{IsZero} \overline 3 \mathinner{\overline{1}}
                    (\comb{Mul} \overline 3 \mathinner{(\comb Y \comb{Fact'} \mathinner{(\mathinner{(-1)} \overline 3)})})
                =_\beta \comb{Mul} \overline 3 \mathinner{(\comb Y \comb{Fact'} \mathinner{(\mathinner{(-1)} \overline 3)})}
                =_\beta \comb{Mul} \overline 3 \mathinner{(\comb Y \comb{Fact'} \overline 2)} \\&=_\beta \ldots
                =_\beta \comb{Mul} \overline 3 \mathinner{(\comb{Mul} \overline 2 \mathinner{(\comb Y \comb{Fact'} \overline 1)})}
                =_\beta \ldots =_\beta \comb{Mul} \overline 3 \mathinner{(\comb{Mul} \overline 2 \mathinner{(
                    \comb{Mul} \overline 1 \mathinner{(\comb Y \comb{Fact'} \overline 0)})})} =_\beta \ldots \\
                &=_\beta \comb{Mul} \overline 3 \mathinner{(\comb{Mul} \overline 2 \mathinner{( 
                    \comb{Mul} \mathinner{\overline 1} \overline 1)})}
                =_\beta \overline 6
\end{align*}
\end{example}

\subsection{\texorpdfstring{Ромбовидное свойство и параллельная редукция}{Diamond property and parallel reduction}}

\begin{definition}[ромбовидное свойство]
    $G$ обладает ромбовидным свойством, если какие бы ни были $a$, $b$, $c$, что $aGb$, $aGc$, $b \ne c$, найдётся такое $d$, что $bGd$ и $cGd$.
\end{definition}

\begin{example}
    $(<)$ на натуральных числах обладает ромбовидным свойством.
    $(>)$ на натуральных числах не обладает ромбовидным свойством.

$\beta$-редукция не обладает ромбовидным свойством.
\begin{figure}[h]
    \centering
    \begin{tikzpicture}[->,>={Stealth[black]},
                every edge/.style={draw=black,thick}]
        \node[label={\scriptsize\tikz\node[circle,draw]{$a$};}]     at (0,   0) (A)  {$(\lambda x . x x)(\comb Ia)$};
        \node[label={135:\scriptsize\tikz\node[circle,draw]{$b$};}] at (-2, -1) (B)  {$(\comb Ia)(\comb Ia)$};
        \node[label={45:\scriptsize\tikz\node[circle,draw]{$c$};}]  at (2,  -1) (C)  {$(\lambda x . x x) a$};
        \node                                                       at (-3, -2) (B1) {$(\comb Ia)a$};
        \node                                                       at (-1, -2) (B2) {$a(\comb Ia)$};
        \node                                                       at (0,  -3) (D)  {$aa$};

        \path (A)  edge (B)
                   edge (C)
              (B)  edge (B1)
                   edge (B2)
              (B1) edge (D)
              (B2) edge (D)
              (C)  edge (D);
    \end{tikzpicture}
    \captionsetup{labelformat=empty}
    \caption{Нет такого $d$, что $b \rightarrow_{\beta} d$ и $c \rightarrow_{\beta} d$.}
\end{figure}
\end{example}

\begin{theorem}[Чёрча-Россера] \label{church-rosser}
    $\beta$-редуцируемость обладает ромбовидным свойством.
\end{theorem}

\begin{lemma}
    Если $R$ обладает ромбовидным свойством, то $R^{*}$ обладает ромбовидным свойством.
\end{lemma}

\begin{proof}
    Пусть $M_1 R^* M_n$ и $M_1 R N_1$. Тогда существуют такие $M_2 \ldots M_{n-1}$, что $M_1 R M_2$ \ldots $M_{n-1} R M_n$.
    Так как $R$ обладает ромбовидным свойством, $M_1 R M_2$ и $M_1 R N_1$, то существует такое $N_2$,
    что $N_1 R N_2$ и $M_2 R N_2$. Аналогично, существуют такие $N_3 \ldots N_n$, что $N_{i-1} R N_{i}$ и $M_i R N_i$.
    Мы получили такое $N_n$, что $N_1 R^* N_n$ и $M_n R^* N_n$.

    Пусть теперь $M_{1,1}R^*M_{1,n}$ и $M_{1,1}R^*M_{m,1}$, то есть имеются $M_{1,2}$\ldots$M_{1,n-1}$ и $M_{2,1}$\ldots$M_{m-1,1}$,
    что $M_{1,i-1} R M_{1,i}$ и $M_{i-1, 1} R M_{i, 1}$.
    Тогда существует такое $M_{2,n}$, что $M_{2,1} R^* M_{2,n}$ и $M_{1,n} R^* M_{2,n}$.
    Аналогично, существуют такие $M_{3,n}\ldots M_{m,n}$, что $M_{i,1} R^* M_{i,n}$ и $M_{1,n} R^* M_{i,n}$.
    Тогда $M_{1,n} R^* M_{m,n}$ и $M_{m,1} R^* M_{m,n}$.
\end{proof}

\begin{definition}[параллельная $\beta$-редукция]
    $A \rightrightarrows_{\beta} B$
    \begin{enumerate}
        \item Если $A =_\alpha B$, то $A \rightrightarrows_{\beta}B$
        \item Если $A \rightrightarrows_{\beta} B$, то $\lambda x.A \rightrightarrows_{\beta} \lambda x . B$
        \item Если $P \rightrightarrows_{\beta} Q$ и $R \rightrightarrows_{\beta} S$, то $PR \rightrightarrows_{\beta} QS$
        \item Если $P \rightrightarrows_{\beta}R$ и $Q \rightrightarrows_{\beta} S$,
            то $(\lambda x . P) Q \rightrightarrows_{\beta} R_{[x\coloneqq{}S]}$.
    \end{enumerate}
\end{definition}

\begin{statement} \label{st-star}
    $(\rightrightarrows_{\beta})$ обладает ромбовидным свойством.
\end{statement}

\begin{proof}
    \todo % TODO
\end{proof}

\begin{statement} \label{st-A}
    Если $A \rightarrow_{\beta} B$, то $A \rightrightarrows_{\beta} B$.
\end{statement}

\begin{statement} \label{st-B}
    Если $A \rightrightarrows_{\beta} B$, то $A \twoheadrightarrow_{\beta} B$.
\end{statement}

\begin{proof}
    \todo % TODO
\end{proof}

При этом, обратное не всегда верно.
\begin{gather*}
    (\lambda x . x x) (\lambda x . x x x) \twoheadrightarrow_{\beta} (\lambda x . x x x)(\lambda x . x x x)(\lambda x . x x x) \\
    (\lambda x . x x) (\lambda x . x x x) \cancel{\rightrightarrows_{\beta}} (\lambda x . x x x)(\lambda x . x x x)(\lambda x . x x x)
\end{gather*}

\begin{statement} \label{st-C}
    Из \ref{st-A} и \ref{st-B} следует, что $(\rightarrow_{\beta})^{*} = (\rightrightarrows_{\beta})^{*}$.
\end{statement}

\begin{proof}
    Теорема \nameref{church-rosser} следует из \ref{st-star} и \ref{st-C}.
\end{proof}

\begin{corollary}
    Нормальная форма для $\lambda$-выражения единственна, если существует.
\end{corollary}

\begin{theorem}[тезис Чёрча]
    Если функция вычислима с помощью механического аппарата, то она вычислима с помощью $\lambda$-выражения.
\end{theorem}

\subsection{\texorpdfstring{Порядок редукции}{Order of reduction}}
\epigraph{"<Завтра! Завтра! Не сегодня!"> "--- так ленивцы говорят.}{Das deutsches Sprichwort}

Допустим, мы действительно хотим с помощью $\lambda$-исчисления что-то посчитать.
Нам надо определиться со стратегией редукции. Если она будет выбрана неудачно, то мы рискуем не дождаться ожидаемого результата.

\begin{definition}
\[
    \comb K = \lambda x \lambda y . x \qquad
    \comb I = \lambda x . x \qquad
    \comb S = \lambda x y z . x z (y z)
\]
\end{definition}
$\comb I$ выражается через $\comb S$ и $\comb K$: $\comb I = \comb S \comb K \comb K$.

\begin{statement} \label{SK-basis}
    Пусть $A$ "--- замкнутое $\lambda$-выражение.
    Тогда найдётся выражение $T$, состоящее только из $\comb S$ и $\comb K$, что $A =_{\beta}T$.
\end{statement}

\begin{wrapfigure}[2]{r}{0pt}
\centering
\begin{tikzpicture}[->,>={Stealth[black]},
            every edge/.style={draw=black,thick}]
    \node at (0, 0) (A)  {$\comb K a \combl \Omega$};
    \node at (2, 0) (B)  {$a$};

    \path (A) edge [loop left] ()
              edge             (B);
\end{tikzpicture}
\caption*{}
\end{wrapfigure}
Нормальная форма некоторых $\lambda$-выражений может не достигаться при неудачном выборе порядка редукции.

\begin{definition}[нормальный порядок редукции]
    Редукция самого левого $\beta$-редекса.
\end{definition}
"<Ленивые вычисления"> (ну, почти, в ленивых ещё есть меморизация)

\begin{definition}[аппликативный порядок редукции]
    Редукция самого левого $\beta$-редекса из самых вложенных.
\end{definition}
"<Энергичные вычисления">

\begin{statement}
    Если нормальная форма существует, она может быть достигнута нормальным порядком редукции.
\end{statement}

\subsection{\texorpdfstring{Парадокс Карри}{Curry's paradox}}

Попробуем построить логику на основе $\lambda$-исчисления.
Введём комбнатор-импликацию, обозначим $(\supset)$. Введём M.P. и правила:
\begin{enumerate}
    \item $A \supset A$
    \item $(A \supset (A \supset B)) \supset (A \supset B)$
    \item $A =_{\beta} B$, тогда $A \supset B$
\end{enumerate}

Покажем, как в полученной логике можно доказать любое утверждение.
Введём обозначение: $Y_{\supset a} \equiv Y (\lambda t . t \supset a) =_{\beta} Y (\lambda t . t \supset a) \supset a$.

\begin{tabular}{lll}
    1) & $Y_{\supset a} \supset Y_{\supset a}$ & (схема аксиом) \\
    2) & $Y_{\supset a} \supset (Y_{\supset a} \supset a)$ & (можно доказать) \\
    3) & $(Y_{\supset a} \supset Y_{\supset a} \supset a) \supset (Y_{\supset a} \supset a)$ & (схема аксиом) \\
    4) & $Y_{\supset a} \supset a$ & (M.P.) \\
    5) & $(Y_{\supset a} \supset a) \supset Y_{\supset a}$ & (третье правило) \\
    6) & $Y_{\supset a}$ & (M.P.) \\
    7) & $a$ & (M.P.)
\end{tabular}

Получается, что данная логика противоречива. Показать это нам дал возможность тот факт,
что в нашей логике с помощью $Y$-комбинатора мы можем ссылаться в утверждении на само себя.
Аналогично можно прийти к парадоксальному выводу из высказывания "<если это утверждение верно, то русалки существуют"> на нашем мета-языке.

\subsection{\texorpdfstring{Импликационный фрагмент ИИВ}{Implication fragment of intuitionistic logic}}

\begin{definition}[импликационный фрагмент ИИВ]
    Рассмотрим интуиционистское исчисление высказываний.
    \begin{enumerate}
        \item Введём схему аксиом:
        \[
            \infer{\Gamma, \varphi \vdash \varphi}{}
        \]
        \item Правило введения импликации:
        \[
            \infer{\Gamma \vdash \varphi \rightarrow \psi}{\Gamma, \varphi \vdash \psi}
        \]
        \item И правило удаления импликации:
        \[
            \infer{\Gamma \vdash \psi}{\Gamma \vdash \varphi \rightarrow \psi && \Gamma \vdash \varphi}
        \]
    \end{enumerate}

    Мы построили импликационный фрагмент ИИВ (и.ф.и.и.в).
\end{definition}

\begin{example} Докажем $\varphi \rightarrow \psi \rightarrow \varphi$:
\[
    \infer[(2)]
        { \vdash \varphi \rightarrow (\psi \rightarrow \varphi) }
        { \infer[(2)]
            { \varphi \vdash \psi \rightarrow \varphi }
            { \infer[(1)]
                { \varphi, \psi \vdash \varphi}
                {}
            }
        }
\]
\end{example}

\begin{theorem}
    И.ф.и.и.в полон в моделях Крипке, то есть $\Gamma \vdash \varphi$ т.и.т.т.,
    когда для любой модели крипке $C$ из $\Vdash_C \Gamma$ следует $\Vdash_C \varphi$.
\end{theorem}

\begin{proof}
    Рассмотрим модель Крипке вида $W = \left\{\Delta \mid \Gamma \subseteq \Delta, \Delta\text{ замкнуто относительно }\vdash\right\}$,
    $\Gamma \leq \Delta$ если $\Gamma \subseteq \Delta$.
    Индукцией по структуре $\varphi$ покажем, что $\Delta \Vdash \varphi$ т.и.т.т., когда $\Delta \vdash \varphi$.
    \begin{enumerate}
        \item Пусть $\varphi \equiv x$ "--- переменная. Тогда $\Gamma \vdash \varphi$ эквивалентно $x \in \Gamma$, что эквивалентно $\Gamma\Vdash x$ (по определению).
        \item Пусть $\varphi \equiv \alpha \rightarrow \beta$.
        \begin{enumerate}[label=(\asbuk*)]
            \item Пусть $\Delta \vdash \alpha\rightarrow\beta$.
                Рассмотрим такое $\Delta'$, что $\Delta \leq \Delta'$ и $\Delta' \Vdash \alpha$.
                Так как $\Delta \vdash \alpha\rightarrow\beta$, то $\Delta' \vdash \alpha\rightarrow\beta$.
                Так как $\Delta' \Vdash \alpha$, то по индукционному предположению $\Delta' \vdash \alpha$.
                \[
                    \infer{\Delta' \vdash \beta}{\Delta' \vdash \alpha \rightarrow \beta && \Delta' \vdash \alpha}
                \]
                Значит, $\Delta' \Vdash \beta$, по индуционному предположению. Тогда и $\Delta' \Vdash \alpha\rightarrow\beta$.
            \item Пусть $\Delta \Vdash \alpha\rightarrow\beta$.
                Пусть $\Delta'$ "--- замыкание относительно выводимости $\Delta \cup \set \alpha$.
                Тогда $\Delta' \Vdash \beta$. Тогда $\Delta' \vdash \beta$ по предположению индукции.
                \[
                    \infer{\Delta' \vdash \alpha \rightarrow \beta}{\Delta', \alpha \vdash \beta} 
                    \qedhere
                \]
        \end{enumerate}
    \end{enumerate}
\end{proof}

\begin{corollary}
    И.ф.и.и.в замкнут относительно выводимости.
\end{corollary}
Если некоторое утверждение выводится в ИИВ ($\vdash_{и} \varphi$) и содержит только импликации,
то оно выводится и в и.ф.и.и.в. ($\vdash_{и \rightarrow} \varphi$).

\subsection{\texorpdfstring{Исчисление по Карри}{Curry-style}}

\begin{definition}[тип]
    $T = \{\alpha, \beta, \gamma, \ldots\}$ "--- множество типов.
    $\sigma$, $\tau$ "--- метапеременные для типов.
    Если $\sigma$, $\tau$ "--- типы, то $\sigma \rightarrow \tau$ "--- тип.
    \begin{bnf}
    \[
        \Pi ::= T | \Pi \rightarrow \Pi | (\Pi)
    \]
    \end{bnf}
    $\left(\rightarrow\right)$ правоассоциативна.
\end{definition}

\begin{definition}[контекст] Контекст "--- $\Gamma$.
\begin{gather*}
    \Gamma = \{ \Lambda_{1} : \sigma_{1},\ \Lambda_{2} : \sigma_{2}\ \ldots\ \Lambda_{n} : \sigma_{n} \} \\
    \abs{\Gamma} = \{ \sigma_{1},\ \sigma_{2}\ \ldots\ \sigma_{n} \} \\
    \dom \Gamma = \{ \Lambda_{1},\ \Lambda_{2}\ \ldots\ \Lambda_{n} \}
\end{gather*}
\end{definition}

\begin{definition}[типизируемость по Карри]
    Рассмотрим исчисление со следующими правилами:
    \begin{enumerate}
        \item $\infer[(x \notin \dom(\Gamma))]
            {\Gamma, x:\sigma \vdash x:\sigma}
            {}$
        \item $\infer[]
            {\Gamma \vdash M N : \tau}
            {\Gamma \vdash M:\sigma \rightarrow \tau && \Gamma \vdash N:\sigma}$
        \item $\infer[(x \notin \dom(\Gamma))]
            {\Gamma \vdash \lambda x . M : \sigma \rightarrow \tau}
            {\Gamma, x : \sigma \vdash M : \tau}$
    \end{enumerate}
    Если $\lambda$-выражение типизируется этими трёмя правилами, то говорят, что оно типизируется по Карри.
\end{definition}

\begin{lemma}[subject reduction]
    Если $\Gamma \vdash M : \sigma$ и $M \rightarrow_{\beta}N$, то $\Gamma \vdash N : \sigma$.
\end{lemma}

\begin{corollary}
    Если $\Gamma \vdash M : \sigma$ и $M \twoheadrightarrow_{\beta}N$, то $\Gamma \vdash N : \sigma$.
\end{corollary}

\begin{theorem}[Чёрча-Россера]
    Если $\Gamma \vdash M : \sigma$, $M \twoheadrightarrow_{\beta} N$ и $M \twoheadrightarrow_{\beta} P$, то найдётся такое $Q$, что
    $N \twoheadrightarrow_{\beta} Q$, $P \twoheadrightarrow_{\beta} Q$ и $\Gamma \vdash Q : \sigma$.
\end{theorem}

\begin{example} Несколько доказательств:
    \begin{enumerate}
        \item $\lambda x . x : \alpha \rightarrow \alpha$:
        \[
            \infer[(3)]
                {\vdash \lambda x . x : \alpha \rightarrow \alpha}
                { \infer[(1)]
                    {x : \alpha \vdash x : \alpha}
                    {}
                }
        \]

        \item $\lambda f . \lambda x . f x : (\alpha \rightarrow \beta) \rightarrow \alpha \rightarrow \beta$:
        \[
            \infer[(3)]
                { \vdash \lambda f . \lambda x . f x : (\alpha \rightarrow \beta) \rightarrow (\alpha \rightarrow \beta) }
                { \infer[(3)]
                    { f : \alpha \rightarrow \beta \vdash \lambda x . f x : \alpha \rightarrow \beta }
                    { \infer[(2)]
                        {f : \alpha \rightarrow \beta; x : \alpha \vdash f x : \beta}
                        {
                            \infer[(1)]{ \Gamma \vdash f : \alpha \rightarrow \beta }{} &&
                            \infer[(1)]{ \Gamma \vdash x : \alpha }{}
                        }
                    }
                }
        \]

        \item \begin{proof}[$\combl \Omega$ и $\comb Y$ не типизируемы] Допустим обратное.
        Тогда в выводе должен будет присутствовать вывод подвыражения $xx$:
        \[
            \infer{\Gamma, x : \sigma \rightarrow \tau, x : \sigma \vdash x x : \tau}
            {  \infer{\Gamma, x : \sigma \rightarrow \tau, x : \sigma \vdash x : \sigma \rightarrow \tau}{}
            && \infer{\Gamma, x : \sigma \rightarrow \tau, x : \sigma \vdash x : \sigma}{}
            }
        \]
        Однако первое правило вывода запрещает повторение переменных в контексте. Значит, такой вывод не может быть корректным.
        \end{proof}

    \end{enumerate}
\end{example}

Если мы знаем тип выражения, то построить вывод этого типа нам не составит труда.
По виду $\lambda$-терма можно однозначно сказать, каким правилом был выведен его тип.
Правилом 1 выводится тип обособленных переменных, правилом 2 выводится тип аппликаций, правилом 3 "--- абстракций.

\begin{lemma}[отсутствие subject expansion]
    Неверно, что если $M \rightarrow_{\beta} N$, $\Gamma \vdash N : \sigma$, то $\Gamma \vdash M : \sigma$.
\end{lemma}
Например, для $\comb K a \combl\Omega$.

В общем случае тип не уникален. Например, одновременно $\vdash \lambda x . x : \alpha \rightarrow \alpha$ и $\vdash \lambda x . x : (\beta \rightarrow \beta) \rightarrow (\beta \rightarrow \beta)$.

\begin{definition}[сильная нормализация] \label{strong-normalization}
    Назовём исчисление сильно-нормализуемым, если любая последовательность $\beta$-редукций неизбежно приводит к нормальной форме.
\end{definition}
Или, иными словами, если не существует бесконечной последовательности $\beta$-редукций.

\begin{definition}[слабая нормализация]
    Назовём исчисление слабо-нормализуемым, если для любого терма существует последовательность $\beta$-редукций, приводящая его к нормальной форме.
\end{definition}

\begin{theorem}[о сильной нормализации]
    Просто типизируемое $\lambda$-исчисление сильно нормализуемо.
    Любое просто типизируемое $\lambda$-выражение сильно нормализуемо.
\end{theorem}

\subsection{\texorpdfstring{Исчисление по Чёрчу}{Church-style}}

\begin{definition}[типизация по Чёрчу]
    \begin{bnf}
    \[
        \Lambda_{\xx} ::= x | \lambda x^{\sigma}.\Lambda_{\xx} | (\Lambda_{\xx}) | \Lambda_{\xx} \Lambda_{\xx}
    \]
    \end{bnf}
    Правила:
    \begin{enumerate}
        \item $\infer[(x \notin \dom(\Gamma))]
            {\Gamma, x:\sigma \vdash_{\xx} x:\sigma}
            {}$
        \item $\infer[]
            {\Gamma \vdash_{\xx} M N : \tau}
            {\Gamma \vdash_{\xx} M:\sigma \rightarrow \tau && \Gamma \vdash_{\xx} N:\sigma}$
        \item $\infer[(x \notin \dom(\Gamma))]
            {\Gamma \vdash_{\xx} \lambda x^{\sigma} . M : \sigma \rightarrow \tau}
            {\Gamma, x : \sigma \vdash_{\xx} M : \tau}$
    \end{enumerate}

\end{definition}

\begin{definition}
\[
    \abs{\Lambda_{\xx}} =
    \begin{cases}
        x                                   & \Lambda_{\xx} \equiv x \\
        \abs{\Lambda_{1}} \abs{\Lambda_{2}} & \Lambda_{\xx} \equiv \Lambda_{1} \Lambda_{2} \\
        \lambda x . \abs{\Lambda}           & \Lambda_{\xx} \equiv \lambda x^{\sigma} . \Lambda
    \end{cases}
\]
\end{definition}

\begin{lemma}[subject reduction по Чёрчу]
    Пусть $\Gamma \vdash_{\xx} M : \sigma$ и $\abs{M} \rightarrow_{\beta} N$.
    Тогда найдётся такое $H$, что $\abs{H} = N$, $\Gamma \vdash_{\xx} H:\sigma$.
\end{lemma}

\begin{theorem}[Чёрча-Россера]
    Пусть $\Gamma \vdash_{\xx} M : \sigma$, $\abs{M} \twoheadrightarrow_{\beta} N$, $\abs{M} \twoheadrightarrow_{\beta} T$.
    Тогда найдётся такое $P$, что $\Gamma \vdash_{\xx} P : \sigma$,
            $N \twoheadrightarrow_{\beta} \abs{P}$ и $T \twoheadrightarrow_{\beta} \abs{P}$.
\end{theorem}

\begin{lemma}[уникальность типов] \label{uniqueness}
    Если $\Gamma \vdash_\xx M : \gamma$ и $\Gamma \vdash_\xx M : \tau$, то $\sigma = \tau$.
\end{lemma}

Лемма \ref{uniqueness} показывает, чем исчисление по Чёрчу отличается исчислением по Карри.

\begin{theorem}[о стирании] \ 
    \begin{enumerate}
        \item Если $M \rightarrow_{\beta} N$ и $\Gamma \vdash_{\xx} M : \sigma$, то $\abs{M} \rightarrow_{\beta} \abs{N}$.
        \item Если $\Gamma \vdash_{\xx} M : \sigma$, то $\Gamma \vdash_{к} \abs{M} : \sigma$.
    \end{enumerate}
\end{theorem}

\begin{theorem}[о поднятии]
    Пусть $P \in \Lambda_{\xx}$, $M, N \in \Lambda_{\rr}$.
    \begin{enumerate}
        \item Если $M \rightarrow_{\beta} N$, $\abs{P} = M$, то найдётся такое $Q$, что $\abs{Q} = N$, $P \rightarrow_{\beta} Q$.
        \item Если $\Gamma \vdash_{\rr} M : \sigma$, то найдётся такое $P \in \Lambda_{\xx}$, что
                $\Gamma \vdash_{\xx} P : \sigma$, $\abs{P} = M$.
    \end{enumerate}
\end{theorem}

\subsection{\texorpdfstring{Изоморфизм Карри-Ховарда}{Curry-Howard correspondence}}

Можно отчётливо проследить связь между аксиомами типизированного $\lambda$-исчисления и аксиомами импликационного фрагмента ИИВ.
Сейчас мы введём два новых типа данных, правила для которых будут соответствовать связкам $\with$ и $\lor$ из ИИВ.

Первый "--- тип "<пары">. Пара хранит в себе два значения. Пусть $A : \sigma$ и $B : \tau$, тогда:
\begin{gather*}
    \pair{A,B} : \sigma \with \tau \\
    \pi_1 : \sigma\with\tau\rightarrow\sigma \qquad \pi_2 : \sigma\with\tau\rightarrow\tau
\end{gather*}
$\pi_1$ и $\pi_2$ это функции для доступа к элементам пары, левая и правая проекции.
Например, $\pi_2 \pair{\overline 7,\overline 5} =_\beta \overline 5$.

Второй "--- алгебраический тип данных. Пусть $T : \sigma\vee\tau$, $L : \sigma\rightarrow\pi$, $R : \tau\rightarrow\pi$. Тогда
\[
    \case{T}{L}{R} : \pi \qquad \inj_1 : \sigma\rightarrow\sigma\vee\tau \qquad \inj_2 : \tau\rightarrow\sigma\vee\tau
\]
Если $A : \sigma$ и $B : \tau$, то $\inj_1 A$ и $\inj_2 B$ имеют тип $\sigma\vee\tau$.
$\inj_1$ и $\inj_2$ это левая и правая инъекции.
Тип $\sigma \vee \tau$ означает, что это либо $\sigma$, либо $\tau$.
Его можно понимать как \mintinline{C++}{union}, который дополнительно хранит в себе информацию о типе, который в нём записан.

Рассмотрим алгебраический тип данных на Haskell: \mintinline{Haskell}{data Fruit = Orange Int | Banana Int}. Левая инъекция для него "---
\mintinline{Haskell}{Orange :: Int -> Fruit}, правая "--- \mintinline{Haskell}{Banana :: Int -> Fruit}.
На нашем языке можно сказать, что $\mathinner\mathtt{Fruit} = \mathinner{\mathtt{Int}} \vee \mathinner{\mathtt{Int}}$.
Тогда если $T \mathinner{:} \mathtt{Fruit}$, то $\case{T}{(+1)}{(-1)} \mathinner{:} \mathrel{\mathtt{Int}}$ это число,
хранящееся в фрукте, увеличенное на единицу, если это апельсин, и уменьшенное на единицу, если это банан.

Соответствие описано в таблицах \ref{correspondence-table} и \ref{correspondence-terms-table}.
Формально изоморфизм описывается следующей теоремой:

\begin{table}[hp]
\centering
\begin{tabular}{Sc@{\hspace{1.5cm}} Sc} \toprule
    ИИВ & Типизированное $\lambda$-исчисление \\ \midrule

    $\infer{\Gamma \vdash \psi}{\Gamma \vdash \varphi \rightarrow \psi && \Gamma \vdash \varphi}$ &
    $\infer{\Gamma \vdash AB : \psi}{\Gamma \vdash A : \varphi \rightarrow \psi && \Gamma \vdash B : \varphi}$ \\ \addlinespace

    $\infer{\Gamma \vdash \varphi \rightarrow \psi}{\Gamma, \varphi \vdash \psi}$ &
    $\infer{\Gamma \vdash \lambda x^\varphi . A : \varphi \rightarrow \psi}{\Gamma, x : \varphi \vdash A : \psi}$ \\ \midrule

    $\infer{\Gamma \vdash \varphi \with \psi}{\Gamma \vdash \varphi && \Gamma \vdash \psi}$ &
    $\infer{\Gamma \vdash \left<A,B\right> : \varphi \with \psi}{\Gamma \vdash A : \varphi && \Gamma \vdash B : \psi}$ \\ \addlinespace

    $\infer{\Gamma \vdash \varphi}{\Gamma \vdash \varphi \with \psi}$ &
    $\infer{\Gamma \vdash \pi_1 R : \varphi}{\Gamma \vdash R : \varphi \with \psi}$ \\ \addlinespace

    $\infer{\Gamma \vdash \psi}{\Gamma \vdash \varphi \with \psi}$ &
    $\infer{\Gamma \vdash \pi_2 R : \psi}{\Gamma \vdash R : \varphi \with \psi}$ \\ \midrule

    $\infer{\Gamma \vdash \varphi \vee \psi}{\Gamma \vdash \varphi}$ &
    $\infer{\Gamma \vdash \inj_1 A : \varphi \vee \psi}{\Gamma \vdash A : \varphi}$ \\ \addlinespace

    $\infer{\Gamma \vdash \varphi \vee \psi}{\Gamma \vdash \psi}$ &
    $\infer{\Gamma \vdash \inj_2 A : \varphi \vee \psi}{\Gamma \vdash A : \psi}$ \\ \addlinespace

    $\infer{\Gamma \vdash \varphi \vee \psi \rightarrow \pi}
        {\Gamma \vdash \varphi \rightarrow \pi && \Gamma \vdash \psi \rightarrow \pi}$ &
    $\infer{\Gamma \vdash \case{T}{A}{B} : \pi}{\Gamma \vdash T : \varphi \vee \psi &&
        \Gamma \vdash A : \varphi \rightarrow \pi && \Gamma \vdash B : \psi \rightarrow \pi}$ \\ \bottomrule
\end{tabular}
\caption{Соответствие правил вывода}
\label{correspondence-table}
\end{table}

\begin{table}[hp]
\centering
\begin{tabular}{Sl Sl} \toprule
    Интуиционистская логика & $\lambda$-исчисление \\ \midrule
    выражение & тип \\
    доказательство & терм (программа) \\
    предположение & свободная переменная \\
    импликация & абстракция (функция) \\ \bottomrule
\end{tabular}
\caption{Соответствие сущностей}
\label{correspondence-terms-table}
\end{table}

\begin{theorem}[об изоморфизме Карри-Ховарда] \ 
    \begin{enumerate}
        \item Пусть $\Delta \vdash_{\xx} M : \sigma$, тогда $\abs{\Delta} \vdash \sigma$.
        \item Пусть $\Gamma \vdash \sigma$ в и.ф.и.и.в., тогда найдётся такой терм M,
            что $\Delta \vdash_{\xx} M : \sigma$, где $\Delta=\left\{ \left(x : \varphi \right) \mid \varphi \in \Gamma \right\}$.
    \end{enumerate}
\end{theorem}

%% долг с предыдущей лекции
\begin{proof}
\begin{enumerate}
    \item Индукция по выводу $\Delta \vdash_\xx M : \sigma$. Заменяем каждое правило в выводе соответсвующим правилом из ИИВ и получаем доказательство $\abs\Delta \vdash \sigma$.

    \item Индукция по структуре вывода $\Gamma \vdash \sigma$. Пусть $\Gamma = \{\sigma_{1}, \sigma_{2} \ldots\}$,
        $\Delta = \{x_{1}:\sigma_{1}, x_{2}:\sigma_{2}, \ldots \}$.
    \begin{enumerate}[label=(\asbuk*)]
        \item Вывод имеет вид
        \[
            \infer{\Gamma, \varphi \vdash \varphi}{}
        \]
        \begin{enumerate}[label=\roman*.]
            \item Если $\varphi \in \Gamma$, то $\infer{\Delta \vdash x : \varphi}{}$.
            \item Если $\varphi \notin \Gamma$, то $\infer{\Delta, x : \varphi \vdash x : \varphi}{}$.
        \end{enumerate}

        \item Вывод заканчивается правилом
        \[
            \infer{\Gamma \vdash \psi}{\Gamma \vdash \varphi \rightarrow \psi && \Gamma \vdash \varphi}
        \]
        По индукционному предположению $\Delta \vdash M : \varphi \rightarrow \psi$ и $\Delta \vdash N : \varphi$. Тогда
        \[
            \infer{\Delta \vdash MN : \psi}{\Delta \vdash A : \varphi \rightarrow \psi && \Delta \vdash B : \varphi}
        \]

        \item Вывод заканчивается правилом
        \[
            \infer{\Gamma \vdash \varphi \rightarrow \psi}{\Gamma,\varphi \vdash \psi}
        \]
        \begin{enumerate}[label=\roman*.]
            \item Пусть $\varphi \in \Gamma$. Тогда по индукционному предположению $\Delta \vdash M : \psi$.
            Возьмём новую переменную $x \notin \dom(\Delta)$.
            Тогда можно изменить построенное доказательство и получить $\Delta, x : \varphi \vdash M : \psi$.

            \item Пусть $\varphi \notin \Gamma$. Тогда по индукционному предположению $\Delta, x : \varphi \vdash M : \psi$.
        \end{enumerate}
        Тогда
        \[
            \infer{\Delta \vdash \lambda x^\varphi . M : \varphi \rightarrow \psi}{\Delta, x : \varphi \vdash M : \psi}
            \qedhere
        \]
    \end{enumerate}
\end{enumerate} %есть в Curry-Howard Isomorphism, стр 75
\end{proof}

\subsection{\texorpdfstring{Вывод типа}{Type deduction}}
\epigraph{Помните, что в $\lambda$-исчислении нет смысла? Здесь смысл отрицательный, скорее.}{Д.Г.}

В $\lambda$-исчислении выделяют 3 задачи:
\begin{enumerate}[label=(\asbuk*)]
    \item Проверка типа: верно ли $\Gamma \vdash M : \sigma$?
    \item Вывод типа: $? \vdash M : \mathinner{?}$
    \item Обитаемость типа: $? \vdash ? : \sigma$
\end{enumerate}

В этом разделе мы будем рассматривать задачу вывода типа.

\begin{definition}[алгебраический терм]
    \begin{bnf}
    \[
        A ::= x | f\left(A, \ldots, A\right)
    \]
    \end{bnf}
    Где $x \in X$.
\end{definition}

Уравнение в алгебраических термах: $A = A$.

\begin{definition}[$S$-подстановка]
    \[
        S : A \rightarrow A
    \]
    Причём $S$ "--- id почти везде (везде кроме конечного количества).
\end{definition}

\begin{definition}[естественное обобщение]
    Естественное обобщение "--- такая подстановка $S : A \rightarrow A$, получаемая из $S : X \rightarrow A$, что
    $S\left(f\left(A_1, \dots, A_n\right)\right) = f\left(S(f_1), \ldots, S(f_n)\right)$
\end{definition}

\begin{definition}[унификатор] \label{unificator}
    $S$ "--- унификатор (решение уравнения) $P=Q$, если $S(P)=S(Q)$.
\end{definition}
\begin{example}
    Пусть
    \[
        \sigma = \beta\rightarrow\alpha\rightarrow\beta \qquad \tau = (\gamma\rightarrow\gamma)\rightarrow\delta
    \]
    Их унификатор это
    \[
        S = [\beta \coloneqq \gamma\rightarrow\gamma, \delta \coloneqq \alpha\rightarrow\gamma\rightarrow\gamma]
        \qquad S(\sigma) \equiv S(\tau) = (\gamma\rightarrow\gamma)\rightarrow\alpha\rightarrow\gamma\rightarrow\gamma
    \]
\end{example}
Задача решения уравнение в алгебраических термах "--- унификация.

\begin{definition}[композиция]
    $(S \circ T)(A) = S(T(A))$
\end{definition}

\begin{definition}[частный случай]
    $T$ "--- частный случай $U$, если существует такое $S$, что $T = S \circ U$.
\end{definition}

\begin{definition}[наибольший общий унификатор]
    Наибольший общий унификатор $U$ для уравнения $A=B$ "--- такой унификатор, что:
    \begin{enumerate}
        \item $U(A)=U(B)$.
        \item Любой другой унификатор "--- частный случай $U$.
    \end{enumerate}
\end{definition}

\begin{definition}[несовместная система]
    Назовём систему несовместной, если выполнено одно из условий:
    \begin{enumerate}
        \item в ней есть уравнение вида $f(\ldots)=g(\ldots)$
        \item в ней есть уравнение вида $x = \ldots x \ldots$
    \end{enumerate}
\end{definition}

\begin{definition}[эквивалентные системы]
    Назовём две системы эквивалентными, если они имеют одинаковые решения.
\end{definition}

\begin{statement}
    Для любой системы
    \[
        \begin{cases} A_1 = B_1 \\ \vdots \\ A_n = B_n \end{cases}
    \]
    найдётся эквивалентная ей система из одного уравнения:
    \[
        f(A_1, \ldots, A_n) = f(B_1, \ldots, B_n)\text{,}
    \]
    где $f$ "--- новый символ.
\end{statement}

\begin{definition}[разрешённая система]
    Назовём систему разрешённой, если:
    \begin{enumerate}
        \item все уравнения имеют вид $x = A$;
        \item все переменные в левой части встречаются однократно.
    \end{enumerate}
\end{definition}

По системе в разрешённой форме мы можем построить решение $S$, определив $S(x_i) = A_i$ для каждого $i$.

\begin{statement}
    Построенный по системе в разрешённой форме унификатор $S$ "--- наибольший общий унификатор.
\end{statement}

\begin{statement}
    Несовместная система не имеет решений.
\end{statement}

Рассотрим следующие преобразования, которые не меняют свойства системы:
\begin{center}
\begin{tabular}{l l l} \toprule
    Выражения                         & Условия             & Новые выражения \\ \midrule
    $T=x$, $T$ не переменная          &                     & $x=T$ \\ \midrule
    $T=T$                             &                     & убрать это уравнение \\ \midrule
    \multirow{2}{*}[-\aboverulesep]{$f(A_1, \ldots A_n) = g(B_1, \ldots B_n)$}
                                      & $f=g$               & $ A_1 = B_1 \ldots A_n = B_n$ \\ \cmidrule{2-3}
                                      & $f \neq g$          & система несовместна \\ \midrule
    \multirow{2}{*}[-\aboverulesep]{$x=T$, $R=S$, $x$ входит в $S$ или $R$}
                                      & $T$ не содержит $x$ & $x=T$,$R\left[x\coloneqq T\right]=S\left[x\coloneqq T\right]$\\ \cmidrule{2-3}
                                      & $T$ содержит $x$    & система несовместна \\ \bottomrule
\end{tabular}
\end{center}

\begin{statement}
    Последовательное применение правил либо за конечное число шагов приведёт систему в разрешённый вид, либо сделает её несовместной.
\end{statement}

\begin{proof}
    Пусть $(n_1, n_2, n_3)$ "--- характеристика системы, где
    $n_1$ "--- количество переменных не входящих систему только слева от знака равенства один раз,
    $n_2$ "--- общее количество вхождений функциональных символов в $S$,
    $n_3$ "--- количество выражений вида $T=x$ или $T=T$.
    Каждое преобразование уменьшает эту тройку (если сравнивать лексикографически).
\end{proof}

\begin{theorem}
    Задача вывода типа в $\lambda$-исчислении разрешима.
\end{theorem}

\begin{proof}
    Опишем алгоритм. \\
    Пусть нам дан $\lambda$-терм $M$. Рекурсивно построим по нему систему уравнений $E_m$:
    \begin{center}
    \begin{tabular}{l l l} \toprule
        $M \equiv x$ & $E_m=\set{}$ & $\tau_m=\alpha$ "--- новая переменная. \\ \midrule
        $M \equiv PR$ & $E_m=E_p \cup E_r \cup \set{\tau_p=\tau_r\rightarrow\pi}$ & $\tau_m=\pi$ "--- новая переменная \\ \midrule
        $M \equiv \lambda x . P$ & $E_m=E_p$ & $\tau_m=\tau_x\rightarrow\tau_p$ \\ \bottomrule
    \end{tabular}
    \end{center}
    Решим построенную систему уравнений.

    Можно показать, что алгоритм корректный.
\end{proof}

\begin{example}
    \todo %todo
\end{example}

%\subsection{\texorpdfstring{Про ложь}{About false}}
%
%\todo\ послушать запись. %todo

\section{\texorpdfstring{Система $F$}{System F}}

Обычное $\lambda$-исчисление позволяет слишком много, просто-типизированное "--- слишком мало ($(-1)$ не выразим). Хотелось бы золотую середину.

\subsection{\texorpdfstring{Интуиционистское исчисление предикатов второго порядка}{Second order intuitionistic logic}}

\begin{definition}
    \begin{bnf}
    \[
        \Phi ::= (\Phi) | p | \Phi \rightarrow \Phi | \forall p . \Phi \color{gray}
            \underbrace{| \exists p . \Phi | \bot | \Phi \with \Phi | \Phi \vee \Phi}_{\text{не существенные}}
    \]
    \end{bnf}
    Введение кванторов:
    \begin{gather*}
        \infer[p \notin \FV(\Gamma)]{\Gamma \vdash \forall p . \varphi}{\Gamma \vdash \varphi} \qquad
        \infer{\Gamma \vdash \exists p . \varphi}{\Gamma \vdash \varphi \left[p \coloneqq \psi\right]}
    \end{gather*}
    Удаление кванторов:
    \begin{gather*}
        \infer{\Gamma \vdash \varphi \left[p \coloneqq \sigma\right]}{\Gamma \vdash \forall p . \varphi} \qquad
        \infer[p \notin \FV(\Gamma, \psi)]{\Gamma \vdash \psi}{\Gamma \vdash \exists p . \varphi && \Gamma, \varphi \vdash \psi}
    \end{gather*}
    Последние четыре связки можно выразить через первые:
    \begin{align*}
        \bot & \equiv \forall p . p \\
        \varphi \with \psi & \equiv \forall a . ((\varphi \rightarrow \psi \rightarrow a) \rightarrow a) \\
        \varphi \vee \psi & \equiv \forall a . (\varphi \rightarrow a) \rightarrow (\psi \rightarrow a) \rightarrow a \\
        \exists x . \tau & \equiv \forall a . (\forall x . \tau \rightarrow a) \rightarrow a
    \end{align*}
\end{definition}

\subsection{\texorpdfstring{Система $F$}{System F}}
\begin{definition}[Тип в системе $F$]
\[
    \tau =
    \begin{cases}
        \alpha, \beta, \gamma, \ldots & \text{(атомарный тип)} \\
        \tau \rightarrow \sigma \\
        \forall \alpha . \tau & \text{($\alpha$ "--- переменнная)}
    \end{cases}
\]
\end{definition}

\begin{definition}[Исчисление по Чёрчу в системе $F$]
    \begin{bnf}
        \begin{gather*}
            \mathbf\Lambda ::= x | \lambda p^\alpha . \mathbf\Lambda | \mathbf\Lambda \mathbf\Lambda | (\mathbf\Lambda)
            | \Lambda \alpha . \mathbf\Lambda | \mathbf\Lambda \tau
        \end{gather*}
    \end{bnf}
    $\Lambda \alpha . \mathbf\Lambda$ "--- типовая (полиморфная абстракция), $\mathbf\Lambda \tau$ "--- применение типа.

    Правила вывода:
    \begin{gather*}
        \infer[x \notin \mathrm{dom}(\Gamma)]{\Gamma, x : \sigma \vdash x : \sigma}{} \\
        \infer{\Gamma \vdash MN : \sigma}{\Gamma \vdash M : \tau \rightarrow \sigma & \Gamma \vdash N : \tau} \qquad
        \infer[(x \notin \mathrm{dom}(\Gamma))]{\Gamma \vdash \lambda x^\tau . M : \tau \rightarrow \sigma}{\Gamma, x : \tau \vdash M : \sigma} \\
        \infer[x \in \FV(\Gamma)]{\Gamma \vdash \Lambda \alpha . M : \forall \alpha : \sigma}{\Gamma \vdash M : \sigma} \qquad
        \infer[\text{(подстановка типа)}]{\Gamma \vdash M \tau : \sigma [\alpha := \tau]}{\Gamma \vdash M : \forall \alpha . \sigma} \\
    \end{gather*}
\end{definition}

\begin{example} Левая проекция: \\
    \begin{tabular}{l l l}
        & Просто типизированное $\lambda$-исчисление & Система $F$ \\
        Тип & $\pi_1:\alpha\with\beta\rightarrow\alpha$ & $\pi_1:\forall \alpha . \forall \beta . \alpha \with \beta \rightarrow \alpha$ \\
        Выражение & $\pi_1 = \lambda p . p T$ & $\pi_1 = \Lambda \varphi . \Lambda \psi . \lambda p^{\varphi\with\psi} . p T$
    \end{tabular}
\end{example}

\begin{definition}[$\beta$-редукция в $F$] \ 
    \begin{enumerate}[label=(\asbuk*)]
        \item Типовая редукция: $\left(\Lambda \alpha . M^\sigma\right) \tau \rightarrow_\beta M[\alpha:=\tau] : \sigma[\alpha := \tau]$
        \item Классическая $\beta$-редукция: $\left(\lambda x^\sigma . M\right)^{\sigma \rightarrow \tau} X \rightarrow_\beta M [x:=X] : \tau$
    \end{enumerate}
\end{definition}

\begin{theorem}[Изоморфизм Карри-Ховарда]
    $\Gamma \vdash_F M :\tau$ т.и.т.т., когда $\abs{\Gamma} \vdash \tau$ в интуиционистском исчислении предикатов второго порядка.
\end{theorem}

\begin{theorem}
    $F$ \hyperref[strong-normalization]{сильно нормализуемо}.
\end{theorem}

\todo 

\subsection*{\texorpdfstring{Экзистенциальные типы}{Existential types}}

\todo
Допустим, у нас есть абстрактный тип данных "<Стек">: \\
\begin{tabular}{l l}
    $\comb{empty}$ & $: \alpha$ \\
    $\comb{push}$  & $: \alpha\with\nu\rightarrow\alpha$ \\
    $\comb{pop}$   & $: \alpha\rightarrow\alpha\with\nu$ \\
\end{tabular} \\
Можно попробовать сказать это так: $\mathtt{stack} =
    \alpha \with (\alpha\with\nu\rightarrow\alpha) \with (\alpha\rightarrow\alpha\with\nu)$.
Но проблема в том, что у нас есть только интерфейс стека, а не его реализация. Поэтому лучше будет сказать так:
    $\exists \alpha . \alpha \with (\alpha\with\nu\rightarrow\alpha) \with (\alpha\rightarrow\alpha\with\nu)$.
То есть существует какое-то $\alpha$, реазизовывающее требуемый интерфейс.
\todo 

\ 

По аналогии с правилом удаления квантора существования, можно определить правила вывода для выражений экзистенциальных типов:
\[
    \infer{\Gamma\vdash (\pack{M, \theta}{\exists \alpha . \varphi}) : \exists \alpha.\varphi}
        {\Gamma \vdash M : \varphi[\alpha := \theta]} \qquad
    \infer[(\alpha \notin \FV(\Gamma, \psi))]
        {\Gamma \vdash \abstype{\alpha}{x:\varphi}{M}{N:\psi}}
        {\Gamma \vdash M : \exists \alpha . \varphi && \Gamma, x : \varphi \vdash N : \psi}
\]
Если вспомнить, что квантор существования выразим через квантор всеобщности, то можно попытаться записать типы выражений
\texttt{pack} и \texttt{abstype} через квантор существования и выразить их без расширения языка.
\begin{gather*}
    \pack{M, \theta}{\exists \alpha . \varphi} =
        \Lambda \beta . \lambda x^{\forall \alpha . \varphi \rightarrow \beta} . x \theta M \\
    \abstype{\alpha}{x : \varphi}{M}{N}:\psi =
        M \psi (\Lambda \alpha . \lambda x ^ \varphi . N)
\end{gather*}
Можно показать, что $\abstype{\alpha}{x:\varphi}{(\pack{M,\theta}{\exists\alpha .\varphi})}{N}$
        редуцируется в $N[\alpha\coloneqq\theta][x\coloneqq M]$:
\begin{align*}
    \abstype{\alpha}{x:\varphi}{(\pack{M,\theta}{\exists\alpha .\varphi})}{N}
    &= (\reduction{\Lambda \beta} . \lambda x^{\forall \alpha . \varphi \rightarrow \beta} . x \theta M)
        \reduction{\psi} (\Lambda \alpha . \lambda x ^ \varphi . N) \\
    &\rightarrow_\beta (\reduction{\lambda x^{\forall \alpha . \varphi \rightarrow \psi}} . x \theta M)
        \reduction{(\Lambda \alpha . \lambda x ^ \varphi . N)} \\
    &\rightarrow_\beta (\reduction{\Lambda \alpha} . \lambda x^\varphi . N) \reduction\theta M \\
    &\rightarrow_\beta (\reduction{\lambda x^\varphi} . N)[\alpha\coloneqq\theta] \reduction M \\
    &\rightarrow_\beta N[\alpha\coloneqq\theta][x\coloneqq M]
\end{align*}

\begin{example} \todo там чёрт ногу сломит \textbf{:/}
\end{example}

\begin{statement}
    $F$ сильно нормализуемо.
\end{statement}

\begin{statement}
    $F$ неразрешима.
\end{statement}
Ни одна из задач $\lambda$-исчисления в системе $F$ не разрешима, даже задача проверки типизации.
Доказать это можно через сведение к проблеме останова.

Итак, мы попытались добавить к типизированному $\lambda$-исчислению полиморфизм и абстрактные типы данных и получили слишком сложный язык.
Давайте попробуем немного его упростить, чтобы с ним можно было работать.

\subsection{\texorpdfstring{Типовая система Хиндли-Милнера}{Hindley and Milner’s type system}}

\begin{definition}[Ранг типа]
\[
    \rank(\tau) =
    \begin{cases}
        \max(\rank(\sigma)+1, \rank(\rho)) & \tau \equiv \sigma \rightarrow \rho\text{, если }\sigma\text{ содержит }\forall \\
        \rank(\rho) & \tau \equiv \sigma \rightarrow \rho\text{, если }\sigma \text{ не содержит } \forall \\
        0 & \tau \equiv \alpha \\
        \max(\rank(\rho), 1) & \tau \equiv \forall \alpha . \rho
    \end{cases}
\]
\end{definition}

\begin{example} Ранг экзистенциального типа:
\begin{align*}
    \rank(\exists \alpha . \beta) &= \rank(\forall \gamma . (\forall \alpha . \beta \rightarrow \gamma) \rightarrow \gamma) \\
    &= \max(\rank((\forall \alpha . \beta \rightarrow \gamma) \rightarrow \gamma), 1) \\
    &= \max(\max(\rank(\forall \alpha . \beta \rightarrow \gamma) + 1, \rank(\gamma)), 1) \\
    &= \max(\max(2, 0), 1) = 2
\end{align*}
\end{example}

\begin{definition}[грамматика типа в системе Хиндли-Милнера] \ \\
    Тип (монотип) "--- выражение в грамматике $ \begin{bnf} \tau ::= \alpha | \tau \rightarrow \tau | (\tau) \end{bnf} $. \\
    Типовая схема (политип) "--- выражение в грамматике $ \begin{bnf} \sigma ::= \tau | \forall \alpha . \sigma \end{bnf} $.
\end{definition}

$\forall\alpha.\alpha\rightarrow\alpha$ "--- политип, $\forall\alpha.\alpha\rightarrow\forall\beta.\beta$ "--- некорректный в системе Хиндли-Милнера тип.

\begin{statement}
    $\rank(\tau) = 0$, $\rank(\sigma) = 1$.
\end{statement}

\begin{definition}
    $\sigma_1$ "--- подтип $\sigma_2$, если существует такая подстановка
            $[\alpha_1 \coloneqq \theta_1, \alpha_2 \coloneqq \theta_2 \ldots \alpha_n \coloneqq \theta_n]$, что:
    \begin{enumerate}
        \item $\sigma_1 = \forall \beta_1 \ldots \forall \beta_k . \tau [\alpha_1 \coloneqq \theta_1 \ldots \alpha_n := \theta_n]$,
            $\alpha_i$ не входит свободно в $\theta_j$
        \item $\sigma_2 = \forall \alpha_1 \ldots \forall \alpha_n \tau$
    \end{enumerate}
\end{definition}


\begin{definition}[грамматика выражения в системе Хиндли-Милнера]
\[
\begin{bnf}
    \Lambda ::= x | \lambda x . \Lambda | \Lambda \Lambda | (\Lambda) | \lett{x=\Lambda}{\Lambda}
\end{bnf}
\]
\end{definition}

\begin{definition}[правила вывода в системе Хиндли-Милнера]
\inferspacing
\begin{gather*}
    \infer[\text{(Тавтология)}]{\Gamma, x : \sigma \vdash x : \sigma}{} \qquad
    \infer[\text{(Уточнение, $\sigma'$ "--- подтип $\sigma$)}]{\Gamma \vdash e : \sigma'}{\Gamma \vdash e : \sigma} \\
    \infer[\text{(Обобщение, $\alpha\notin\FV(\Gamma)$)}]{\Gamma \vdash e : \forall \alpha . \sigma}{\Gamma \vdash e : \sigma} \qquad
    \infer[\text{(Абстракция)}]{\Gamma \vdash \lambda x . e : \sigma \rightarrow \tau}{\Gamma, x : \sigma \vdash e : \tau} \\
    \infer[\text{(Применение)}]
        {\Gamma \vdash e e' : \tau}{\Gamma \vdash e : \sigma \rightarrow \tau && \Gamma \vdash e' : \sigma} \qquad
    \infer[\text{(Let)}]
        {\Gamma \vdash \lett{x=e}{e'=\tau}}
        {\Gamma \vdash e : \sigma && \Gamma, x : \sigma \vdash e' : \tau}
\end{gather*}
\end{definition}

Мы существенно ограничили набор возможных типов в нашем языке, однако, он всё ещё вполне сильный.
Например, в нём можно определить тип $\comb Y$: $\forall \alpha . (\alpha\rightarrow\alpha)\rightarrow\alpha$,
и явно определить оператор фиксированной точки: $\mathtt{fix~f~=~f~(fix~f)}$.
Через него можно определять рекурсивные функции и они будут типизироваться.

\subsection{\texorpdfstring{Вывод типа}{Type inference}}
\begin{statement}
    Задача вывода типа в системе Хиндли-Милнера разрешима.
\end{statement}
Приведём алгоритм, решающий эту задачу.
Он будет принимать выражение $e$ в контексте $\Gamma$ и возвращать такую подстановку $S$ и тип $\tau$, что
\[
    S(\Gamma) \vdash e : \tau
\]

В алгоритме будем пользоваться \hyperref[unificator]{унификацией} ($U(\tau_1, \tau_2)$ "--- унификатор $\tau_1$ и $\tau_2$),
будем обозначать контекст $\Gamma$ без типа $x$ как $\Gamma_x$
и определим замыкание всех несвязанных типовых переменных в контексте:
\[
    \overline{\Gamma}(\tau) = \forall \alpha_1 \ldots \forall \alpha_n . \tau
\]
где $\alpha_i \in \FV(\tau)$ и $\alpha_i \notin \FV(\Gamma)$.

Алгоритм описан в таблице \ref{algorithm-w}.
Если какие-то условия не могут быть соблюдены, то тип выражения не может быть выведен.

%\begin{center}
\begin{table}[ht]
\centering
\begin{tabular}{Sl Sl Sl} \toprule
    Вид $e$ & Условия & $W(\Gamma, e)$ \\ \midrule
    $x$
        & $x : \forall \alpha_1 \ldots \alpha_k . \tau' \in \Gamma$ & $S =\mathrm{Id}$ \\
        & $\beta_i$ "--- новые переменные                           & $\tau = \tau'[\alpha_i \coloneqq \beta_i]$ \\
        \midrule
    $e_1 e_2$
            & $W(\Gamma, e_1) = (S_1, \tau_1)$                                       & $S = V \circ S_1 \circ S_2$ \\
            & $W(S_1(\Gamma), e_2) = (S_2, \tau_2)$                                  & $\tau = S(\beta)$ \\
            & $U(S_2(\tau_1), \tau_2 \rightarrow \beta) = V$, $\beta$ "--- новый тип & \\ \midrule
    $\lambda x . e$
        & $W(\Gamma_x \cup \set{x : \beta}, e) = (S_1, \tau_1)$ & $S = S_1$  \\
        & $\beta$ "--- новый тип                                & $\tau = S(\beta) \rightarrow \tau_1$ \\ \midrule
    $\lett{x=e_1}{e_2}$
        & $W(\Gamma, e_1) = (S_1, \tau_1)$                                                     & $S = S_2 \circ S_1$ \\
        & $W(S_1 \Gamma_x \cup \set{x : \overline{S_1 \Gamma} (\tau_1)}, e_2) = (S_2, \tau_2)$ & $\tau = \tau_2$ \\ \bottomrule
\end{tabular}
\caption{Алгоритм $W$.}
\label{algorithm-w}
\end{table}
%\end{center}

\begin{example}
\todo
\end{example}

\section{\texorpdfstring{Линейные и уникальные типы}{Linear and unique types}}

Пусть $A \to_\beta A'$.
С одной стороны, порядок редукции не важен, % TODO картинка
$(\lambda x . x x) A \to_\beta (\lambda x . x x) A' \to_\beta A' A'$
и $(\lambda x . x x) A \to_\beta A A \to_\beta A' A \to_\beta A' A'$.
По теореме \nameref{church-rosser} нормальная форма единственна, если существует.
С другой стороны, реальный мир на самом деле не таков, в нём есть побочные эффекты.
% A(b(), unique_ptr<C>(new C))

\subsection{\texorpdfstring{Комбинаторная логика}{Combinatory logic}}

Историческая справка. Известные нам комбинаторы $\comb S$ и $\comb K$ придумал Моисей Шейнфинкель,
ещё до Карри с его $\lambda$-исчислением.
У него было своё каррирование и комбинаторная логика.
Хаскелл Карри придумал свою систему позже, однако независимо от Шейнфинкеля.

\begin{center} \newcommand{\eemph}[1]{\underline{\textbf{#1}}}
\begin{tabular}{l l l l l} \toprule
    \multicolumn{3}{l}{Моисей Шейнфинкель} & \multicolumn{2}{l}{Хаскелл Карри} \\ \cmidrule(lr){1-3} \cmidrule(lr){4-5}
    $\comb I$ & (\eemph{I}dentität)     & $\lambda x . x$             & $\comb B$ & $\lambda x y z . x (y z)$ \\
    $\comb K$ & (\eemph{K}onstanz)      & $\lambda x y . x$           & $\comb C$ & $\lambda x y z . x z y$   \\
    $\comb S$ & (Ver\eemph{s}chmelzung) & $\lambda x y z . x z (y z)$ & $\comb K$ & $\lambda x y . x$         \\
    $\comb T$ & (Ver\eemph{t}auschung)  & $\lambda x y z . x z y$     & $\comb W$ & $\lambda x y . x y y$ \\ \bottomrule
\end{tabular} % Вот тут про названия. Больше не нашёл :( http://www.johndcook.com/blog/2014/02/06/schonfinkel-combinators/
\end{center}

С помощью каждой из этих двух систем можно выразить любое $\lambda$-выражение без свободных переменных.
Пусть
\[
    \Lambda_{\comb{SK}} ::= x \mid \comb S \mid \comb K \mid \comb I \mid (\Lambda_{\comb{SK}} \Lambda_{\comb{SK}})
\]

\begin{proof}[Докажем \ref{SK-basis}] \newcommand{\opop}{\operatorname}
    Определим $\opop T : \Lambda \to \Lambda_{\comb{SK}}$:
    \begin{align*}
        \opop T[x]                         &= x \\
        \opop T[A B]                       &= \opop{T}[A]\opop{T}[B]  \\
        \opop T[\lambda x . P]             &= \comb K \opop{T}[P] \text{, если $x \notin \FV(P)$} \\
        \opop T[\lambda x . x]             &= \comb I \\
        \opop T[\lambda x . A B]           &= \comb S \opop{T}[\lambda x . A]\opop{T}[\lambda x . B] \\
        \opop T[\lambda x . \lambda y . A] &= \opop T[\lambda x . \opop T[\lambda y . A]]
    \end{align*}
    $\opop T[\lambda\text{-выражение}]$ завершается и не содержит абстракций.
    Можно проверить, что $\opop T[A] =_\beta A$.
\end{proof}

Например, $\overline 4$ после преобразования приведённым алгоритмом, очевидно, будет выглядеть так:
\[
    \overline 4 =_\beta \comb S (\comb S (\comb K \comb S) (\comb S (\comb K \comb K) \comb I)) (\comb S (\comb S (\comb K \comb S) (\comb S (\comb K \comb K) \comb I)) (\comb S (\comb S (\comb K \comb S) (\comb S (\comb K \comb K) \comb I)) (\comb S (\comb S (\comb K \comb S) (\comb S (\comb K \comb K) \comb I)) (\comb K \comb I))))
\]

Для доказательства аналогичной теоремы для базиса $\comb{BCKW}$ достаточно выразить через них комбинаторы $\comb S$ и $\comb K$:
\[
    \comb S = \comb B (\comb B \comb W) (\comb B \comb B \comb C) \qquad \comb I = \comb C \comb K \comb K
\]

Давайте теперь проследим связь комбинаторов с логикой. Выведем типы у $\comb S$ и $\comb K$:
\begin{align*}
    \comb S &= \lambda x y z . x z (y z) : (\alpha \to \beta \to \gamma) \to
        (\alpha \to \beta) \to (\alpha \to \gamma) \\
    \comb K &= \lambda x y . x : \alpha \to \beta \to \alpha
\end{align*}
Это похоже на вторую и первую схемы аксиом в ИИВ.
Понятно, что изоморфизм Карри-Ховарда переносится и на комбинаторы, так как это сокращения для $\lambda$-выражений.
Роль modus ponens будет исполнять редукция.
Посмотрим на другие комбинаторы:
\begin{alignat*}{4}
    \comb I&=\lambda x.x         &&: \alpha \to \alpha \\
    \comb B&=\lambda x y z.x(yz) &&:(\alpha\to\beta)\to(\beta \to\gamma) \to (\alpha \to \gamma) \\
    \comb C&=\lambda x y z.x z y &&:(\alpha \to\beta\to \gamma)\to (\beta \to \alpha \to \gamma) \\
    \comb K&=\lambda x y . x     &&: \alpha \to \beta \to \alpha \\
    \comb W&=\lambda x y . x y y &&: (\alpha \to \alpha \to \beta) \to (\alpha \to \beta)
\end{alignat*}

У последних двух комбинаторов есть отличительная особенность. $\comb K$ "<убивает"> один аргумент, а $\comb W$ "--- дублирует.
В логике это не даёт ничего плохого, однако в $\lambda$-исчислении такой эффект может быть нежелателен.
Ведь не всегда в реальных программах мы можем произвольным образом дублировать какие-то объекты.
Базис $\comb{BCKWI}$ по изоморфизму Карри-Ховарда порождает интуиционистскую логику.
Можно рассматривать исчисления, прорджённые базисами $\comb{BCKI}$ и $\comb{BCI}$.

Прежде чем описывать новую логику, мы вспомним классическое ИИВ, однако представим правила вывода в немного другом виде:
\begin{@empty} \inferspacing
\begin{gather*}
    \infer{A \vdash A}{} \qquad
    \infer{\Delta, \Gamma \vdash A}{\Gamma, \Delta \vdash A} \qquad
    \infer{\Gamma, A \vdash B}{\Gamma, A, A \vdash B} \qquad
    \infer{\Gamma, A \vdash B}{\Gamma \vdash B} \\
    \infer{\Gamma \vdash A \to B}{\Gamma, A \vdash B} \qquad
    \infer{\Gamma, \Delta \vdash B}{\Gamma \vdash A \to B && \Delta \vdash A} \\
    \infer{\Gamma, \Delta \vdash A \times B}{\Gamma \vdash A && \Delta \vdash B} \qquad
    \infer{\Gamma, \Delta \vdash C}{\Gamma \vdash A \times B && \Delta, A, B \vdash C}\\
    \infer{\Gamma \vdash A + B}{\Gamma \vdash A} \qquad
    \infer{\Gamma \vdash A + B}{\Gamma \vdash B} \qquad
    \infer{\Gamma, \Delta \vdash C}{\Gamma \vdash A + B && \Delta, A \vdash C && \Delta, B \vdash C}
\end{gather*}
\end{@empty}%

\subsection{\texorpdfstring{Линейные высказывания}{Linear statements}}
\begin{definition}[Грамматика линейных высказываний]
    \begin{bnf}
    \[
        T ::= x | T \multimap T | T \otimes T | T \with T | T \oplus T | \oc T
    \]
    \end{bnf}
\end{definition}
Неформально, этим значкам можно предать следующий смысл (формально "--- смысла в этом нет):
\begin{enumerate}
	\item $A \multimap B$  "--- Имеем возможноть превратить $A$ в $B$
	\item $A \otimes B$ "--- Можем получить и $A$, и $B$, одновременно
	\item $A \with B$ "--- Можем получить либо $A$, либо $B$, по своему усмотрению
	\item $A \oplus B$ "--- Можем получить либо $A$, либо $B$, но не по своему усмотрению
	\item $\oc A$ "--- Имеем фабрику, производящая  неограниченное количество $A$
\end{enumerate}

Расширим язык новыми правилами работы с контекстом "--- заведем два вида контекстов: $\pair{A}$ "--- линейный, $[A]$ "--- интуиционистский.
Если у мета-переменной контекста нет каких-либо скобок, то вид контекста нам не важен и может быть любым.
Разница заключается в том, что на выражанения в линейном контексте налагаются новые правила,
говорящие что мы не можем раздвоить или убрать в контексте эти выражения. Формально это отображается в следующих аксиомах:
\[
	\infer{\pair{A} \vdash A}{} \qquad
    \infer{[A] \vdash A}{} \qquad
    \infer{\Delta, \Gamma \vdash A}{\Gamma, \Delta \vdash A} \qquad
    \infer{\Gamma, [A] \vdash B}{\Gamma, [A], [A] \vdash B} \qquad
    \infer{\Gamma, [A] \vdash B}{\Gamma \vdash B} \\
\]

Заведем также аксиомы для работы с самими выражениями:
\begin{@empty}
\inferspacing
\begin{gather*}
	\infer[]{[\Gamma] \vdash \oc A}{[\Gamma] \vdash A} \qquad
	\infer[]{\Gamma, \Delta \vdash B}{\Gamma \vdash \oc A && \Delta, [A] \vdash B} \qquad
	\infer[]{\Gamma \vdash A \multimap B}{\Gamma, \pair{A} \vdash B} \qquad
	\infer[]{\Gamma, \Delta \vdash B}{\Gamma \vdash A \multimap B && \Delta \vdash A} \\
	\infer[]{\Gamma, \Delta \vdash A \otimes B}{\Gamma \vdash A && \Delta \vdash B} \qquad
	\infer[]{\Gamma, \Delta \vdash C}{\Gamma \vdash A \otimes B && \Delta, \pair{A}, \pair{B} \vdash C} \\
	\infer[]{\Gamma \vdash A \with B}{\Gamma \vdash A && \Gamma \vdash B} \qquad
	\infer[]{\Gamma \vdash A}{\Gamma \vdash A \with B} \qquad
	\infer[]{\Gamma \vdash B}{\Gamma \vdash A \with B} \\
	\infer[]{\Gamma \vdash A \oplus B}{\Gamma \vdash A} \qquad
	\infer[]{\Gamma \vdash A \oplus B}{\Gamma \vdash B} \qquad
	\infer[]{\Gamma, \Delta \vdash C}{\Gamma \vdash A \oplus B && \Delta, \pair{A} \vdash C && \Delta, \pair{B} \vdash C}
\end{gather*}
\end{@empty}
\begin{example}[законы Де-Моргана]
В линейных высказываниях также верны законы Де-Моргана :
\[
	\pair{\oc (A \with B)} \vdash \oc A \otimes \oc B \qquad
	\pair{\oc A \otimes \oc B} \vdash \oc (A \with B)
\]
Докажем один из них:
\[
	\infer{\pair{\oc (A \with B)} \vdash \oc A \otimes \oc B} {
		\infer{\pair{\oc (A \with B)} \vdash \oc (A \with B)}{}
        &&
        \infer{[A \with B] \vdash \oc A \otimes \oc B} {
            \infer{[A \with B, A \with B] \vdash \oc A \otimes \oc B}{
                \infer{[A \with B] \vdash \oc A} {
                    \infer{[A \with B] \vdash A} {
                        \infer{[A \with B] \vdash A \with B}{}
                    }
                }
                &&
                \infer{[A \with B] \vdash \oc B} {
                    \infer{[A \with B] \vdash B} {
                        \infer{[A \with B] \vdash A \with B}{}
                    }
                }
            }
        }
    }
\]
\end{example}

Интуционисткую логику можно вложить в линейную логику путем определения интуционистких связок через линейные:
\begin{align*}
    A \to B &= \oc A \multimap B \\
    A \times B      &= A \with B \\
    A + B           &= \oc A \oplus \oc B
\end{align*}
В качестве альтернативного варианта, $A \times B$ можно вложить как $\oc A \otimes \oc B$.
Доказательства из ИИВ могут быть переписаны на язык линейной логики путем замены аксиомы ИИВ их аналогами из линенйной логики.
\begin{table}
\centering
\begin{tabular}{Sc Sc} \toprule
	Схема аксиом в ИИВ & Представление в линейной логике \\ \midrule
	$\infer[]{\Gamma \vdash A \to B}{\Gamma, A \vdash B}$ &
	$\infer[]{\Gamma \vdash \oc A \multimap B}{
		\infer[]{\Gamma, \pair{\oc A} \vdash B}{
			\infer[]{\pair{\oc A} \vdash \oc A}{} && \Gamma, [A] \vdash B
		}
	}$ \\ \midrule
	$\infer[]{\Gamma, \Delta \vdash B}{\Gamma \vdash A \to B && \Delta \vdash A}$ &
	$\infer[]{\Gamma, [\Delta] \vdash B}{
		\Gamma \vdash \oc A \multimap B && \infer[]{[\Delta] \vdash \oc A}{[\Delta] \vdash A}
	}$ \\ \bottomrule
\end{tabular}
\caption{Примеры перевода аксиом ИИВ в линейную логику.}
\label{intutionistic-axioms-to-linear-samples}
\end{table}

% Филипп Вадлерasdasd
\subsection{\texorpdfstring{Новое $\lambda$-исчисление}{}}
\begin{gather*}
    \lambda
        \left<x\right>.u
        | s\left<t\right>
        | \oc t
        | case~s~of~\oc x \rightarrow u
        | \left<\left<t,u\right>\right> | fst\left<p\right> | snd \left<p\right> \\
        | inl \left<t\right> | inr \left<t\right> | case~s~of~inl\left<x\right> \rightarrow v; inr\left<y\right>\rightarrow w
\end{gather*}

\subsection{\texorpdfstring{Уникальные типы}{Unique types}}

\begin{bnf}
\[
    \kappa ::= \tau | U | * | \kappa \rightarrow \kappa
\]
\end{bnf}

Артибуты типа.
Типы: Int, Bool, \ldots $: \tau$ "--- базовые типы, $\rightarrow : * \rightarrow * \rightarrow \tau$. \\
$\cdot, \times : U$, $\vee \with : U \rightarrow U \rightarrow U$, $\oc : U \rightarrow U$. \\
$Attr : \tau \rightarrow U \rightarrow *$

\begin{bnf}
\[
    e ::= x
\]
\end{bnf}

ай

\section{\texorpdfstring{Теорема Диаконеску}{}}

\begin{theorem}
    Аксиома выбора влечёт закон исключённого третьего.
\end{theorem}

\begin{proof}
    Пусть есть формула $S$. Докажем $S \vee \neg S$.
    Например,
    \begin{align*}
        A &= \left\{ x \in \left\{0;1\right\} \mid x = 0 \vee S \right\} \\
        B &= \left\{ x \in \left\{0;1\right\} \mid x = 1 \vee S \right\}
    \end{align*}
    Аксиома выбора утверждает $f : \left\{A, B\right\} \rightarrow \{0, 1\}$
    \[
        f(A) \in A \with f(B) \in B \rightarrow (f(A) = 0 \vee S) \with (f(B) = 1 \vee S) \equiv (f(A) = 0 \with f(B) = 1) \vee S
    \]
    \[
        \infer{\neg (f(A) = f(B)) \rightarrow \neg S}
        {\infer{S \rightarrow f(A) = f(B)}
        {\infer{S \rightarrow A = B}
        {\infer{\neg (f(A) = f(B)) \vee S }
        {}}}}
    \]
\todo
\end{proof}

Другая формулировка аксиомы выбора:
\[
    (\forall a \in A . \exists b \in B . Q(a, b)) \rightarrow \exists f : A \rightarrow B . \forall a \in A . Q(a, f(a))
\]
$Q(a, b) = b \in a$.

\subsection{\texorpdfstring{Сетоид}{Setoid}}

\begin{definition}
    \[
        \left<C : Type, Eq : C \rightarrow C \rightarrow Type, P : \mathrm{IsEquivalence}~Eq\right>
    \]
    $Eq$ "--- отношение эквивалентности, $P$ "--- доказательство этого факта.
\end{definition}
\begin{definition}[экстенциональность]
    \[
        f : \left< A, =_A, P_A\right> \rightarrow \left<B, =_B, P_B\right>
    \]
    $f$ экстенсиональна ($\mathrm{Ext}~f$), если из $x =_A y$ следует $f x =_B f y$
\end{definition}

Пример сетоида "--- целые числа:
\[
    \left< \left<p : \mathrm{Nat}, n : \mathrm{Nat}\right>, \overline{Eq(a+d=b+c)}, \mathrm{IsEquivalence} Eq\right>
\]


\end{document}
