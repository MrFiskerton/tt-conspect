\section{\texorpdfstring{Линейные и уникальные типы}{Linear and unique types}}

Пусть $A \rightarrow_\beta A'$.
С одной стороны, порядок редукции не важен, % TODO картинка
$(\lambda x . x x) A \rightarrow_\beta (\lambda x . x x) A' \rightarrow_\beta A' A'$
и $(\lambda x . x x) A \rightarrow_\beta A A \rightarrow_\beta A' A \rightarrow_\beta A' A'$.
По теореме \nameref{church-rosser} нормальная форма единственна, если существует.
С другой стороны, реальный мир на самом деле не таков, в нём есть побочные эффекты.
% A(b(), unique_ptr<C>(new C))

\subsection{\texorpdfstring{Комбинаторы}{Combinators}}

Рассмотрим комбинаторные логики Моисея Шейнфинкеля и Хаскелла Карри:

Моисей Шейнфинкель:
\begin{center}
\begin{tabular}{Sc Sl}
    $I$ & Identität \\
    $K$ & Konstanz \\
    $S$ & VerSchmelzung \\
    $T$ & VerTauschung
\end{tabular}
\end{center}

Хаскел Карри:
\begin{align*}
    B &= \lambda x y z . x (y z) \\
    C &= \lambda x y z . x z y \\
    K &= \lambda x y . x \\
    W &= \lambda x y . x y y
\end{align*}


\begin{proof}[Докажем \ref{SK-basis}]
    Определим $T : \Lambda \rightarrow \Lambda_{SK} \cup \left\{ \text{свободные переменные} \right\}$:
    \begin{align*}
        T[x] &= x \\
        T[A B] &= T[A]\ T[B]  \\
        T[\lambda x . P] &= K\ T[P] \text{, если $x \notin FV(P)$} \\
        T[\lambda x . x] &= I =_\beta SKK \\
        T[\lambda x . A B] &= S\ T[\lambda x . A]\ T[\lambda x . B] \\
        T[\lambda x . \lambda y . A] &= T[\lambda x . T[\lambda y . A]]
    \end{align*}

    $T[\lambda\text{-выражение}]$ завершается и не содержит абстракций.
    Можно показать, что $T[A] =_\beta A$.
\end{proof}

Можно доказать аналогичную теорему для комбинаторов $B$, $C$, $K$, $W$, выразив через них $S$ и $K$:
$K = K$, $S = B (BW) (BBC)$, $I = CKK$.

А теперь выведем типы у $S$ и $K$:
\begin{align*}
    S &= \lambda x y z . x z (y z) : (\alpha \rightarrow \beta \rightarrow \gamma) \rightarrow
        (\alpha \rightarrow \beta) \rightarrow (\alpha \rightarrow \gamma) \\
    K &= \lambda x y . x : \alpha \rightarrow \beta \rightarrow \alpha
\end{align*}
Это похоже на вторую и первую схемы аксиом в ИИВ.

Базис $BCKWI$ по изоморфизму Карри-Ховарда порождает интуиционистскую логику.
Можно рассматривать исчисления, прорджённые базисами $BCKI$ и $BCI$.
\begin{align*}
    I &: \alpha \rightarrow \alpha \\
    B &: (\alpha \rightarrow \beta) \rightarrow (\beta \rightarrow \gamma) \rightarrow (\alpha \rightarrow \gamma) \\
    C &: (\alpha \rightarrow \beta \rightarrow \gamma) \rightarrow (\beta \rightarrow \alpha \rightarrow \gamma) \\
    K &: \alpha \rightarrow \beta \rightarrow \alpha \\
    W &: (\alpha \rightarrow \alpha \rightarrow \beta) \rightarrow (\alpha \rightarrow \beta)
\end{align*}

Сейчас мы выпишем какое-то исчисление.
\[
    \infer{\Gamma \vdash A \times B}{\Gamma \vdash A && \Gamma \vdash B} \qquad
    \infer{\Gamma \vdash A}{\Gamma \vdash A \times B} \qquad
    \infer{\Gamma \vdash B}{\Gamma \vdash A \times B}
\]

\todo{}шенька

\subsection{\texorpdfstring{Линейные высказывания}{Linear statements}}
\begin{definition}[Грамматика линейных высказываний]
    \begin{bnf}
    \[
        T ::= x | T \multimap T | T \otimes T | T \with T | T \oplus T | \oc T
    \]
    \end{bnf}
\end{definition}
Неформально, этим значкам можно предать следующий смысл(формально - смысла в этом нет):
\begin{enumerate}
	\item $A \multimap B$  "--- Имеем возможноть превратить $A$ в $B$
	\item $A \otimes B$ "--- Можем получить и $A$, и $B$, одновременно
	\item $A \with B$ "--- Можем получить либо $A$, либо $B$, по своему усмотрению
	\item $A \oplus B$ "--- Можем получить либо $A$, либо $B$, но не по своему усмотрению
	\item $\oc A$ "--- Имеем фабрику, производящая  неограниченное количество $A$
\end{enumerate}

Расширим язык новыми правилами работы с контекстом "--- заведем два вида контекстов: $\pair{A}$ "--- линейный, $[A]$ "--- интуиционистский.
Если у мета-переменной контекста нет каких-либо скобок, то вид контекста нам не важен и может быть любым.
Разница заключается в том, что на выражанения в линейном контексте налагаются новые правила,
говорящие что мы не можем раздвоить или убрать в контексте эти выражения. Формально это отображается в следующих аксиомах:
\begin{gather*}
	\infer{\pair{A} \vdash A}{} \qquad
    \infer{[A] \vdash A}{} \qquad
    \infer{\Delta, \Gamma \vdash A}{\Gamma, \Delta \vdash A} \qquad
    \infer{\Gamma, [A] \vdash B}{\Gamma, [A], [A] \vdash B} \qquad
    \infer{\Gamma, [A] \vdash B}{\Gamma \vdash B} \\
\end{gather*}

Заведем также аксиомы для работы с самими выражениями:
\begin{gather*}
	\infer[]{[\Gamma] \vdash \oc A}{[\Gamma] \vdash A} \qquad
	\infer[]{\Gamma, \Delta \vdash B}{\Gamma \vdash \oc A && \Delta, [A] \vdash B} \qquad
	\infer[]{\Gamma \vdash A \multimap B}{\Gamma, \pair{A} \vdash B} \qquad
	\infer[]{\Gamma, \Delta \vdash B}{\Gamma \vdash A \multimap B && \Delta \vdash A} \\
	\infer[]{\Gamma, \Delta \vdash A \otimes B}{\Gamma \vdash A && \Delta \vdash B} \qquad
	\infer[]{\Gamma, \Delta \vdash C}{\Gamma \vdash A \otimes B && \Delta, \pair{A}, \pair{B} \vdash C} \\
	\infer[]{\Gamma \vdash A \with B}{\Gamma \vdash A && \Gamma \vdash B} \qquad
	\infer[]{\Gamma \vdash A}{\Gamma \vdash A \with B} \qquad
	\infer[]{\Gamma \vdash B}{\Gamma \vdash A \with B} \\
	\infer[]{\Gamma \vdash A \oplus B}{\Gamma \vdash A} \qquad
	\infer[]{\Gamma \vdash A \oplus B}{\Gamma \vdash B} \qquad
	\infer[]{\Gamma, \Delta \vdash C}{\Gamma \vdash A \oplus B && \Delta, \pair{A} \vdash C && \Delta, \pair{B} \vdash C}
\end{gather*}
\begin{example}[закон Де-Моргана]
\[
	\infer{\pair{\oc (A \with B)} \vdash \oc A \otimes \oc B} {
		\infer{\pair{\oc (A \with B)} \vdash \oc (A \with B)}{}
        &&
        \infer{[A \with B] \vdash \oc A \otimes \oc B} {
            \infer{[A \with B, A \with B] \vdash \oc A \otimes \oc B}{
                \infer{[A \with B] \vdash \oc A} {
                    \infer{[A \with B] \vdash A} {
                        \infer{[A \with B] \vdash A \with B}{}
                    }
                }
                &&
                \infer{[A \with B] \vdash \oc B} {
                    \infer{[A \with B] \vdash B} {
                        \infer{[A \with B] \vdash A \with B}{}
                    }
                }
            }
        }
    }
\]
\end{example}

Интуционисткую логику можно вложить в линейную логику путем определения интуционистких связок через линейные:
\begin{align*}
    A \rightarrow B &= \oc A \multimap B \\
    A \times B      &= A \with B \\
    A + B           &= \oc A \oplus \oc B
\end{align*}
В качестве альтернативного варианта, $A \times B$ можно вложить как $\oc A \otimes \oc B$.
Доказательства из ИИВ могут быть переписаны на язык линейной логики путем замены аксиомы ИИВ их аналогами из линенйной логики.
\begin{table}
\centering
\begin{tabular}{Sc Sc} \toprule
	Схема аксиом в ИИВ & Представление в линейной логике \\ \toprule
	$\infer[]{\Gamma \vdash A \rightarrow B}{\Gamma, A \vdash B}$ &
	$\infer[]{\Gamma \vdash \oc A \multimap B}{
		\infer[]{\Gamma, \pair{\oc A} \vdash B}{
			\infer[]{\pair{\oc A} \vdash \oc A}{} && \Gamma, [A] \vdash B
		}
	}$ \\ \midrule
	$\infer[]{\Gamma, \Delta \vdash B}{\Gamma \vdash A \rightarrow B && \Delta \vdash A}$ &
	$\infer[]{\Gamma, [\Delta] \vdash B}{
		\Gamma \vdash \oc A \multimap B && \infer[]{[\Delta] \vdash \oc A}{[\Delta] \vdash A}
	}$ \\ \bottomrule
\end{tabular}
\caption{Примеры перевода аксиом ИИВ в линейную логику}
\label{intutionistic-axioms-to-linear-samples}
\end{table}
