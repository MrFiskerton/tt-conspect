\section{\texorpdfstring{Линейные и уникальные типы}{Linear and unique types}}

Пусть $A \to_\beta A'$.
С одной стороны, порядок редукции не важен, % TODO картинка
$(\lambda x . x x) A \to_\beta (\lambda x . x x) A' \to_\beta A' A'$
и $(\lambda x . x x) A \to_\beta A A \to_\beta A' A \to_\beta A' A'$.
По теореме \nameref{church-rosser} нормальная форма единственна, если существует.
С другой стороны, реальный мир на самом деле не таков, в нём есть побочные эффекты.
% A(b(), unique_ptr<C>(new C))

\subsection{\texorpdfstring{Комбинаторная логика}{Combinatory logic}}

Историческая справка. Известные нам комбинаторы $\comb S$ и $\comb K$ придумал Моисей Шейнфинкель,
ещё до Карри с его $\lambda$-исчислением.
У него было своё каррирование и комбинаторная логика.
Хаскелл Карри придумал свою систему позже, однако независимо от Шейнфинкеля.

\begin{center} \newcommand{\eemph}[1]{\underline{\textbf{#1}}}
\begin{tabular}{l l l l l} \toprule
    \multicolumn{3}{l}{Моисей Шейнфинкель} & \multicolumn{2}{l}{Хаскелл Карри} \\ \cmidrule(lr){1-3} \cmidrule(lr){4-5}
    $\comb I$ & (\eemph{I}dentität)     & $\lambda x . x$             & $\comb B$ & $\lambda x y z . x (y z)$ \\
    $\comb K$ & (\eemph{K}onstanz)      & $\lambda x y . x$           & $\comb C$ & $\lambda x y z . x z y$   \\
    $\comb S$ & (Ver\eemph{s}chmelzung) & $\lambda x y z . x z (y z)$ & $\comb K$ & $\lambda x y . x$         \\
    $\comb T$ & (Ver\eemph{t}auschung)  & $\lambda x y z . x z y$     & $\comb W$ & $\lambda x y . x y y$ \\ \bottomrule
\end{tabular} % Вот тут про названия. Больше не нашёл :( http://www.johndcook.com/blog/2014/02/06/schonfinkel-combinators/
\end{center}

С помощью каждой из этих двух систем можно выразить любое $\lambda$-выражение без свободных переменных.
Пусть
\[
    \Lambda_{\comb{SK}} ::= x \mid \comb S \mid \comb K \mid \comb I \mid (\Lambda_{\comb{SK}} \Lambda_{\comb{SK}})
\]

\begin{proof}[Докажем \ref{SK-basis}] \newcommand{\opop}{\operatorname}
    Определим $\opop T : \Lambda \to \Lambda_{\comb{SK}}$:
    \begin{align*}
        \opop T[x]                         &= x \\
        \opop T[A B]                       &= \opop{T}[A]\opop{T}[B]  \\
        \opop T[\lambda x . P]             &= \comb K \opop{T}[P] \text{, если $x \notin \FV(P)$} \\
        \opop T[\lambda x . x]             &= \comb I \\
        \opop T[\lambda x . A B]           &= \comb S \opop{T}[\lambda x . A]\opop{T}[\lambda x . B] \\
        \opop T[\lambda x . \lambda y . A] &= \opop T[\lambda x . \opop T[\lambda y . A]]
    \end{align*}

    $\opop T[\lambda\text{-выражение}]$ завершается и не содержит абстракций.
    Можно проверить, что $\opop T[A] =_\beta A$.
\end{proof}

Например, $\overline 4$ после преобразования приведённым алгоритмом, очевидно, будет выглядеть так:
\[
    \overline 4 =_\beta \comb S (\comb S (\comb K \comb S) (\comb S (\comb K \comb K) \comb I)) (\comb S (\comb S (\comb K \comb S) (\comb S (\comb K \comb K) \comb I)) (\comb S (\comb S (\comb K \comb S) (\comb S (\comb K \comb K) \comb I)) (\comb S (\comb S (\comb K \comb S) (\comb S (\comb K \comb K) \comb I)) (\comb K \comb I))))
\]

Для доказательства аналогичной теоремы для базиса $\comb{BCKW}$ достаточно выразить через них комбинаторы $\comb S$ и $\comb K$:
\[
    \comb S = \comb B (\comb B \comb W) (\comb B \comb B \comb C) \qquad \comb I = \comb C \comb K \comb K
\]

Давайте теперь проследим связь комбинаторов с логикой. Выведем типы у $\comb S$ и $\comb K$:
\begin{align*}
    \comb S &= \lambda x y z . x z (y z) : (\alpha \to \beta \to \gamma) \to
        (\alpha \to \beta) \to (\alpha \to \gamma) \\
    \comb K &= \lambda x y . x : \alpha \to \beta \to \alpha
\end{align*}
Это похоже на вторую и первую схемы аксиом в ИИВ.
Понятно, что изоморфизм Карри-Ховарда переносится и на комбинаторы, так как это сокращения для $\lambda$-выражений.
Роль modus ponens будет исполнять редукция.
Посмотрим на другие комбинаторы:
\begin{alignat*}{4}
    \comb I&=\lambda x.x         &&: \alpha \to \alpha \\
    \comb B&=\lambda x y z.x(yz) &&:(\alpha\to\beta)\to(\beta \to\gamma) \to (\alpha \to \gamma) \\
    \comb C&=\lambda x y z.x z y &&:(\alpha \to\beta\to \gamma)\to (\beta \to \alpha \to \gamma) \\
    \comb K&=\lambda x y . x     &&: \alpha \to \beta \to \alpha \\
    \comb W&=\lambda x y . x y y &&: (\alpha \to \alpha \to \beta) \to (\alpha \to \beta)
\end{alignat*}

У последних двух комбинаторов есть отличительная особенность. $\comb K$ "<убивает"> один аргумент, а $\comb W$ "--- дублирует.
В логике это не даёт ничего плохого, однако в $\lambda$-исчислении такой эффект может быть нежелателен.
Ведь не всегда в реальных программах мы можем произвольным образом дублировать какие-то объекты.
Базис $\comb{BCKWI}$ по изоморфизму Карри-Ховарда порождает интуиционистскую логику.
Можно рассматривать исчисления, прорджённые базисами $\comb{BCKI}$ и $\comb{BCI}$.

Прежде чем описывать новую логику, мы вспомним классическое ИИВ, однако представим правила вывода в немного другом виде:
\begin{@empty} \inferspacing
\begin{gather*}
    \infer{A \vdash A}{} \qquad
    \infer{\Delta, \Gamma \vdash A}{\Gamma, \Delta \vdash A} \qquad
    \infer{\Gamma, A \vdash B}{\Gamma, A, A \vdash B} \qquad
    \infer{\Gamma, A \vdash B}{\Gamma \vdash B} \\
    \infer{\Gamma \vdash A \to B}{\Gamma, A \vdash B} \qquad
    \infer{\Gamma, \Delta \vdash B}{\Gamma \vdash A \to B && \Delta \vdash A} \\
    \infer{\Gamma, \Delta \vdash A \times B}{\Gamma \vdash A && \Delta \vdash B} \qquad
    \infer{\Gamma, \Delta \vdash C}{\Gamma \vdash A \times B && \Delta, A, B \vdash C}\\
    \infer{\Gamma \vdash A + B}{\Gamma \vdash A} \qquad
    \infer{\Gamma \vdash A + B}{\Gamma \vdash B} \qquad
    \infer{\Gamma, \Delta \vdash C}{\Gamma \vdash A + B && \Delta, A \vdash C && \Delta, B \vdash C}
\end{gather*}
\end{@empty}%

\subsection{\texorpdfstring{Линейные высказывания}{Linear statements}}
\begin{definition}[Грамматика линейных высказываний]
    \begin{bnf}
    \[
        T ::= x | T \multimap T | T \otimes T | T \with T | T \oplus T | \oc T
    \]
    \end{bnf}
\end{definition}
Неформально, этим значкам можно предать следующий смысл (формально "--- смысла в этом нет):
\begin{enumerate}
	\item $A \multimap B$  "--- Имеем возможноть превратить $A$ в $B$
	\item $A \otimes B$ "--- Можем получить и $A$, и $B$, одновременно
	\item $A \with B$ "--- Можем получить либо $A$, либо $B$, по своему усмотрению
	\item $A \oplus B$ "--- Можем получить либо $A$, либо $B$, но не по своему усмотрению
	\item $\oc A$ "--- Имеем фабрику, производящая  неограниченное количество $A$
\end{enumerate}

Расширим язык новыми правилами работы с контекстом "--- заведем два вида контекстов: $\pair{A}$ "--- линейный, $[A]$ "--- интуиционистский.
Если у мета-переменной контекста нет каких-либо скобок, то вид контекста нам не важен и может быть любым.
Разница заключается в том, что на выражанения в линейном контексте налагаются новые правила,
говорящие что мы не можем раздвоить или убрать в контексте эти выражения. Формально это отображается в следующих аксиомах:
\[
	\infer{\pair{A} \vdash A}{} \qquad
    \infer{[A] \vdash A}{} \qquad
    \infer{\Delta, \Gamma \vdash A}{\Gamma, \Delta \vdash A} \qquad
    \infer{\Gamma, [A] \vdash B}{\Gamma, [A], [A] \vdash B} \qquad
    \infer{\Gamma, [A] \vdash B}{\Gamma \vdash B} \\
\]

Заведем также аксиомы для работы с самими выражениями:
\begin{@empty}
\inferspacing
\begin{gather*}
	\infer[]{[\Gamma] \vdash \oc A}{[\Gamma] \vdash A} \qquad
	\infer[]{\Gamma, \Delta \vdash B}{\Gamma \vdash \oc A && \Delta, [A] \vdash B} \qquad
	\infer[]{\Gamma \vdash A \multimap B}{\Gamma, \pair{A} \vdash B} \qquad
	\infer[]{\Gamma, \Delta \vdash B}{\Gamma \vdash A \multimap B && \Delta \vdash A} \\
	\infer[]{\Gamma, \Delta \vdash A \otimes B}{\Gamma \vdash A && \Delta \vdash B} \qquad
	\infer[]{\Gamma, \Delta \vdash C}{\Gamma \vdash A \otimes B && \Delta, \pair{A}, \pair{B} \vdash C} \\
	\infer[]{\Gamma \vdash A \with B}{\Gamma \vdash A && \Gamma \vdash B} \qquad
	\infer[]{\Gamma \vdash A}{\Gamma \vdash A \with B} \qquad
	\infer[]{\Gamma \vdash B}{\Gamma \vdash A \with B} \\
	\infer[]{\Gamma \vdash A \oplus B}{\Gamma \vdash A} \qquad
	\infer[]{\Gamma \vdash A \oplus B}{\Gamma \vdash B} \qquad
	\infer[]{\Gamma, \Delta \vdash C}{\Gamma \vdash A \oplus B && \Delta, \pair{A} \vdash C && \Delta, \pair{B} \vdash C}
\end{gather*}
\end{@empty}
\begin{example}[законы Де-Моргана]
В линейных высказываниях также верны законы Де-Моргана :
\[
	\pair{\oc (A \with B)} \vdash \oc A \otimes \oc B \qquad
	\pair{\oc A \otimes \oc B} \vdash \oc (A \with B)
\]
Докажем один из них:
\[
	\infer{\pair{\oc (A \with B)} \vdash \oc A \otimes \oc B} {
		\infer{\pair{\oc (A \with B)} \vdash \oc (A \with B)}{}
        &&
        \infer{[A \with B] \vdash \oc A \otimes \oc B} {
            \infer{[A \with B, A \with B] \vdash \oc A \otimes \oc B}{
                \infer{[A \with B] \vdash \oc A} {
                    \infer{[A \with B] \vdash A} {
                        \infer{[A \with B] \vdash A \with B}{}
                    }
                }
                &&
                \infer{[A \with B] \vdash \oc B} {
                    \infer{[A \with B] \vdash B} {
                        \infer{[A \with B] \vdash A \with B}{}
                    }
                }
            }
        }
    }
\]
\end{example}

Интуционисткую логику можно вложить в линейную логику путем определения интуционистких связок через линейные:
\begin{align*}
    A \to B &= \oc A \multimap B \\
    A \times B      &= A \with B \\
    A + B           &= \oc A \oplus \oc B
\end{align*}
В качестве альтернативного варианта, $A \times B$ можно вложить как $\oc A \otimes \oc B$.
Доказательства из ИИВ могут быть переписаны на язык линейной логики путем замены аксиомы ИИВ их аналогами из линенйной логики.
\begin{table}
\centering
\begin{tabular}{Sc Sc} \toprule
	Схема аксиом в ИИВ & Представление в линейной логике \\ \midrule
	$\infer[]{\Gamma \vdash A \to B}{\Gamma, A \vdash B}$ &
	$\infer[]{\Gamma \vdash \oc A \multimap B}{
		\infer[]{\Gamma, \pair{\oc A} \vdash B}{
			\infer[]{\pair{\oc A} \vdash \oc A}{} && \Gamma, [A] \vdash B
		}
	}$ \\ \midrule
	$\infer[]{\Gamma, \Delta \vdash B}{\Gamma \vdash A \to B && \Delta \vdash A}$ &
	$\infer[]{\Gamma, [\Delta] \vdash B}{
		\Gamma \vdash \oc A \multimap B && \infer[]{[\Delta] \vdash \oc A}{[\Delta] \vdash A}
	}$ \\ \bottomrule
\end{tabular}
\caption{Примеры перевода аксиом ИИВ в линейную логику.}
\label{intutionistic-axioms-to-linear-samples}
\end{table}
