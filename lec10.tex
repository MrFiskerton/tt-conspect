\section{\texorpdfstring{Линейные и уникальные типы}{Linear and unique types}}

Пусть $A \rightarrow_\beta A'$.
С одной стороны, порядок редукции не важен, % TODO картинка
$(\lambda x . x x) A \rightarrow_\beta (\lambda x . x x) A' \rightarrow_\beta A' A'$
и $(\lambda x . x x) A \rightarrow_\beta A A \rightarrow_\beta A' A \rightarrow_\beta A' A'$.
По теореме \nameref{church-rosser} нормальная форма единственна, если существует.
С другой стороны, реальный мир на самом деле не таков, в нём есть побочные эффекты.
% A(b(), unique_ptr<C>(new C))

\subsection{\texorpdfstring{Комбинаторы}{Combinators}}

Рассмотрим комбинаторные логики Моисея Шейнфинкеля и Хаскелла Карри:

Моисей Шейнфинкель:
\begin{center}
\begin{tabular}{Sc Sl}
    $I$ & Identität \\
    $K$ & Konstanz \\
    $S$ & VerSchmelzung \\
    $T$ & VerTauschung
\end{tabular}
\end{center}

Хаскел Карри:
\begin{align*}
    B &= \lambda x y z . x (y z) \\
    C &= \lambda x y z . x z y \\
    K &= \lambda x y . x \\
    W &= \lambda x y . x y y
\end{align*}


\begin{proof}[Докажем \ref{SK-basis}]
    Определим $T : \Lambda \rightarrow \Lambda_{SK} \cup \left\{ \text{свободные переменные} \right\}$:
    \begin{align*}
        T[x] &= x \\
        T[A B] &= T[A]\ T[B]  \\
        T[\lambda x . P] &= K\ T[P] \text{, если $x \notin FV(P)$} \\
        T[\lambda x . x] &= I =_\beta SKK \\
        T[\lambda x . A B] &= S\ T[\lambda x . A]\ T[\lambda x . B] \\
        T[\lambda x . \lambda y . A] &= T[\lambda x . T[\lambda y . A]]
    \end{align*}

    $T[\lambda\text{-выражение}]$ завершается и не содержит абстракций.
    Можно показать, что $T[A] =_\beta A$.
\end{proof}

Можно доказать аналогичную теорему для комбинаторов $B$, $C$, $K$, $W$, выразив через них $S$ и $K$:
$K = K$, $S = B (BW) (BBC)$, $I = CKK$.

А теперь выведем типы у $S$ и $K$:
\begin{align*}
    S &= \lambda x y z . x z (y z) : (\alpha \rightarrow \beta \rightarrow \gamma) \rightarrow
        (\alpha \rightarrow \beta) \rightarrow (\alpha \rightarrow \gamma) \\
    K &= \lambda x y . x : \alpha \rightarrow \beta \rightarrow \alpha
\end{align*}
Это похоже на вторую и первую схемы аксиом в ИИВ.

Базис $BCKWI$ по изоморфизму Карри-Ховарда порождает интуиционистскую логику.
Можно рассматривать исчисления, прорджённые базисами $BCKI$ и $BCI$.
\begin{align*}
    I &: \alpha \rightarrow \alpha \\
    B &: (\alpha \rightarrow \beta) \rightarrow (\beta \rightarrow \gamma) \rightarrow (\alpha \rightarrow \gamma) \\
    C &: (\alpha \rightarrow \beta \rightarrow \gamma) \rightarrow (\beta \rightarrow \alpha \rightarrow \gamma) \\
    K &: \alpha \rightarrow \beta \rightarrow \alpha \\
    W &: (\alpha \rightarrow \alpha \rightarrow \beta) \rightarrow (\alpha \rightarrow \beta)
\end{align*}

Сейчас мы выпишем какое-то исчисление.
\[
    \infer{\Gamma \vdash A \times B}{\Gamma \vdash A && \Gamma \vdash B} \qquad
    \infer{\Gamma \vdash A}{\Gamma \vdash A \times B} \qquad
    \infer{\Gamma \vdash B}{\Gamma \vdash A \times B}
\]

\todo{}шенька

\subsection{\texorpdfstring{Линейные высказывания}{Linear statements}}

\begin{definition}
    \begin{bnf}
    \[
        T ::= x | T \multimap T | T \otimes T | T \with T | T \oplus T | \oc T
    \]
    \end{bnf}
\end{definition}

Контексты двух сортов. $\left<A\right>$ "--- линейный, $\left[ A \right]$ "--- интуиционистский.

\begin{gather*}
    \infer{\left<A\right> \vdash A}{} \qquad
    \infer{\left[A\right] \vdash A}{} \qquad
    \infer{\Delta, \Gamma \vdash A}{\Gamma, \Delta \vdash A} \qquad
    \infer{\Gamma, \left[A\right] \vdash B}{\Gamma, \left[A\right], \left[A\right] \vdash B} \qquad
    \infer{\Gamma, \left[A\right] \vdash B}{\Gamma \vdash B} \\
\end{gather*}

Докажем закон Де-Моргана:
\[
    \infer{\left<\oc (A \with B)\right> \vdash \oc A \otimes \oc B} {
        \infer{\left<\oc (A \with B)\right> \vdash \oc (A \with B)}{}
        &&
        \infer{\left[A \with B\right] \vdash \oc A \otimes \oc B} {
            \infer{\left[A \with B, A \with B\right] \vdash \oc A \otimes \oc B}{
                \infer{\left[A \with B\right] \vdash \oc A} {
                    \infer{\left[A \with B\right] \vdash A} {
                        \infer{\left[A \with B \right] \vdash A \with B}{}
                    }
                }
                &&
                \infer{\left[A \with B\right] \vdash \oc B} {
                    \infer{\left[A \with B\right] \vdash B} {
                        \infer{\left[A \with B \right] \vdash A \with B}{}
                    }
                }
            }
        }
    }
\]

Вложение интуиционистских связок в нашу логику:
\begin{align*}
    A \rightarrow B &= \oc A \multimap B \\
    A \times B      &= A \with B \\
    A + B           &= \oc A \oplus \oc B
\end{align*}

Также можно вкладывать $A \times B$ как $\oc A \otimes \oc B$.
