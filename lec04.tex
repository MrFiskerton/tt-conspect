\subsection{\texorpdfstring{Вывод типа}{Type deduction}}
\epigraph{Помните, что в $\lambda$-исчислении нет смысла? Здесь смысл отрицательный, скорее.}{Д.Г.}

В $\lambda$-исчислении выделяют 3 задачи:
\begin{enumerate}[label=(\asbuk*)]
    \item Проверка типа: верно ли $\Gamma \vdash M : \sigma$?
    \item Вывод типа: $? \vdash M : \mathinner{?}$
    \item Обитаемость типа: $? \vdash ? : \sigma$
\end{enumerate}

В этом разделе мы будем рассматривать задачу вывода типа.

\begin{definition}[алгебраический терм]
    \begin{bnf}
    \[
        A ::= x | f\left(A, \ldots, A\right)
    \]
    \end{bnf}
    Где $x \in X$.
\end{definition}

Уравнение в алгебраических термах: $A = A$.

\begin{definition}[$S$-подстановка]
    \[
        S : A \rightarrow A
    \]
    Причём $S$ "--- id почти везде (везде кроме конечного количества).
\end{definition}

\begin{definition}[естественное обобщение]
    Естественное обобщение "--- такая подстановка $S : A \rightarrow A$, получаемая из $S : X \rightarrow A$, что
    $S\left(f\left(A_1, \dots, A_n\right)\right) = f\left(S(f_1), \ldots, S(f_n)\right)$
\end{definition}

\begin{definition}[унификатор] \label{unificator}
    $S$ "--- унификатор (решение уравнения) $P=Q$, если $S(P)=S(Q)$.
\end{definition}
\begin{example}
    Пусть
    \[
        \sigma = \beta\rightarrow\alpha\rightarrow\beta \qquad \tau = (\gamma\rightarrow\gamma)\rightarrow\delta
    \]
    Их унификатор это
    \[
        S = [\beta \coloneqq \gamma\rightarrow\gamma, \delta \coloneqq \alpha\rightarrow\gamma\rightarrow\gamma]
        \qquad S(\sigma) \equiv S(\tau) = (\gamma\rightarrow\gamma)\rightarrow\alpha\rightarrow\gamma\rightarrow\gamma
    \]
\end{example}
Задача решения уравнение в алгебраических термах "--- унификация.

\begin{definition}[композиция]
    $(S \circ T)(A) = S(T(A))$
\end{definition}

\begin{definition}[частный случай]
    $T$ "--- частный случай $U$, если существует такое $S$, что $T = S \circ U$.
\end{definition}

\begin{definition}[наибольший общий унификатор]
    Наибольший общий унификатор $U$ для уравнения $A=B$ "--- такой унификатор, что:
    \begin{enumerate}
        \item $U(A)=U(B)$.
        \item Любой другой унификатор "--- частный случай $U$.
    \end{enumerate}
\end{definition}

\begin{definition}[несовместная система]
    Назовём систему несовместной, если выполнено одно из условий:
    \begin{enumerate}
        \item в ней есть уравнение вида $f(\ldots)=g(\ldots)$
        \item в ней есть уравнение вида $x = \ldots x \ldots$
    \end{enumerate}
\end{definition}

\begin{definition}[эквивалентные системы]
    Назовём две системы эквивалентными, если они имеют одинаковые решения.
\end{definition}

\begin{statement}
    Для любой системы
    \[
        \begin{cases} A_1 = B_1 \\ \vdots \\ A_n = B_n \end{cases}
    \]
    найдётся эквивалентная ей система из одного уравнения:
    \[
        f(A_1, \ldots, A_n) = f(B_1, \ldots, B_n)\text{,}
    \]
    где $f$ "--- новый символ.
\end{statement}

\begin{definition}[разрешённая система]
    Назовём систему разрешённой, если:
    \begin{enumerate}
        \item все уравнения имеют вид $x = A$;
        \item все переменные в левой части встречаются однократно.
    \end{enumerate}
\end{definition}

По системе в разрешённой форме мы можем построить решение $S$, определив $S(x_i) = A_i$ для каждого $i$.

\begin{statement}
    Построенный по системе в разрешённой форме унификатор $S$ "--- наибольший общий унификатор.
\end{statement}

\begin{statement}
    Несовместная система не имеет решений.
\end{statement}

Рассотрим следующие преобразования, которые не меняют свойства системы:
\begin{center}
\begin{tabular}{l l l} \toprule
    Выражения                         & Условия             & Новые выражения \\ \midrule
    $T=x$, $T$ не переменная          &                     & $x=T$ \\ \midrule
    $T=T$                             &                     & убрать это уравнение \\ \midrule
    \multirow{2}{*}[-\aboverulesep]{$f(A_1, \ldots A_n) = g(B_1, \ldots B_n)$}
                                      & $f=g$               & $ A_1 = B_1 \ldots A_n = B_n$ \\ \cmidrule{2-3}
                                      & $f \neq g$          & система несовместна \\ \midrule
    \multirow{2}{*}[-\aboverulesep]{$x=T$, $R=S$, $x$ входит в $S$ или $R$}
                                      & $T$ не содержит $x$ & $x=T$,$R\left[x\coloneqq T\right]=S\left[x\coloneqq T\right]$\\ \cmidrule{2-3}
                                      & $T$ содержит $x$    & система несовместна \\ \bottomrule
\end{tabular}
\end{center}

\begin{statement}
    Последовательное применение правил либо за конечное число шагов приведёт систему в разрешённый вид, либо сделает её несовместной.
\end{statement}

\begin{proof}
    Пусть $(n_1, n_2, n_3)$ "--- характеристика системы, где
    $n_1$ "--- количество переменных не входящих систему только слева от знака равенства один раз,
    $n_2$ "--- общее количество вхождений функциональных символов в $S$,
    $n_3$ "--- количество выражений вида $T=x$ или $T=T$.
    Каждое преобразование уменьшает эту тройку (если сравнивать лексикографически).
\end{proof}

\begin{theorem}
    Задача вывода типа в $\lambda$-исчислении разрешима.
\end{theorem}

\begin{proof}
    Опишем алгоритм. \\
    Пусть нам дан $\lambda$-терм $M$. Рекурсивно построим по нему систему уравнений $E_m$:
    \begin{center}
    \begin{tabular}{l l l} \toprule
        $M \equiv x$ & $E_m=\set{}$ & $\tau_m=\alpha$ "--- новая переменная. \\ \midrule
        $M \equiv PR$ & $E_m=E_p \cup E_r \cup \set{\tau_p=\tau_r\rightarrow\pi}$ & $\tau_m=\pi$ "--- новая переменная \\ \midrule
        $M \equiv \lambda x . P$ & $E_m=E_p$ & $\tau_m=\tau_x\rightarrow\tau_p$ \\ \bottomrule
    \end{tabular}
    \end{center}
    Решим построенную систему уравнений.

    Можно показать, что алгоритм корректный.
\end{proof}

\begin{example}
    \todo %todo
\end{example}

%\subsection{\texorpdfstring{Про ложь}{About false}}
%
%\todo\ послушать запись. %todo
