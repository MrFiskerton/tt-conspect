\subsection{\texorpdfstring{Вывод типа}{Type deduction}}
\epigraph{Помните, что в $\lambda$-исчислении нет смысла? Здесь смысл отрицательный, скорее.}{Д.Г.}

В $\lambda$-исчислении выделяют 3 задачи:
\begin{enumerate}[label=(\asbuk*)]
    \item Проверка типа: верно ли $\Gamma \vdash M : \sigma$?
    \item Вывод типа: существуют ли такие $\Gamma$ и $\sigma$, что $\Gamma \vdash M : \sigma$?
    \item Обитаемость типа: существуют ли такие $\Gamma$ и $M$, что $\Gamma \vdash M : \sigma$?
\end{enumerate}%
Первые две задачи возникают в контексте дизайна языков программирования.
Понятно, что задача проверки типа сводится к задаче вывода типа.

Из изоморфизма Карри-Ховарда ясно, что задача обитаемости типа эквивалентна задаче проверки выражения ИИВ на доказуемость.

В этом разделе мы будем рассматривать задачу вывода типа.

\begin{definition}[алгебраический терм]
    \begin{bnf}
    \[
        A ::= x | f\left(A, \ldots, A\right)
    \]
    \end{bnf}%
    Где $x \in X$, $f$ "--- функциональный символ.
\end{definition}

Уравнение в алгебраических термах: $A = A$.

\begin{definition}[$S$-подстановка]
    \[
        S : A \rightarrow A
    \]
    Причём $S$ "--- id почти везде (везде кроме конечного количества).
\end{definition}

\begin{definition}[естественное обобщение]
    Естественное обобщение "--- такая подстановка $S : A \rightarrow A$, получаемая из $S : X \rightarrow A$, что
    $S\left(f\left(A_1 \dots A_n\right)\right) = f\left(S(f_1) \ldots S(f_n)\right)$
\end{definition}

\begin{definition}[унификатор] \label{unificator}
    $S$ "--- унификатор (решение уравнения) $P=Q$, если $S(P)=S(Q)$.
\end{definition}
\begin{example}
    Пусть
    \[
        \sigma = \beta\rightarrow\alpha\rightarrow\beta \qquad \tau = (\gamma\rightarrow\gamma)\rightarrow\delta
    \]
    Их унификатор это
    \[
        S = [\beta \coloneqq \gamma\rightarrow\gamma, \delta \coloneqq \alpha\rightarrow\gamma\rightarrow\gamma]
        \qquad S(\sigma) \equiv S(\tau) = (\gamma\rightarrow\gamma)\rightarrow\alpha\rightarrow\gamma\rightarrow\gamma
    \]
\end{example}
Задача решения уравнение в алгебраических термах "--- унификация.

\begin{definition}[композиция]
    $(S \circ T)(A) = S(T(A))$
\end{definition}

\begin{definition}[частный случай]
    $T$ "--- частный случай $U$, если существует такое $S$, что $T = S \circ U$.
\end{definition}

\begin{definition}[наиболее общий унификатор]
    Наиболее общий унификатор $U$ для уравнения $A=B$ "--- такой унификатор, что:
    \begin{enumerate}
        \item $U(A)=U(B)$.
        \item Любой другой унификатор "--- частный случай $U$.
    \end{enumerate}
\end{definition}

\begin{definition}[несовместная система]
    Назовём систему несовместной, если выполнено одно из условий:
    \begin{enumerate}
        \item В ней есть уравнение вида $f(\ldots)=g(\ldots)$.
        \item В ней есть уравнение вида $x = f(\ldots~x~\ldots)$.
    \end{enumerate}
\end{definition}

\begin{definition}[эквивалентные системы]
    Назовём две системы эквивалентными, если они имеют одинаковые решения.
\end{definition}

\begin{statement}
    Для любой системы
    \[
        \begin{cases}
            A_1 = B_1 \\
            \phantom{A_1} \mathrel{\makebox[\widthof{=}]{\vdots}} \\
            A_n = B_n
        \end{cases}
    \]
    найдётся эквивалентная ей система из одного уравнения:
    \[
        f(A_1 \ldots A_n) = f(B_1 \ldots B_n)
    \]
    где $f$ "--- новый символ.
\end{statement}

\begin{definition}[разрешённая система]
    Назовём систему разрешённой, если:
    \begin{enumerate}
        \item Все уравнения имеют вид $x = A$.
        \item Все переменные в левой части встречаются однократно.
    \end{enumerate}
\end{definition}

По системе в разрешённой форме мы можем построить решение $S$, определив $S(x_i) = A_i$ для каждого $i$.

\begin{statement}
    Построенный по системе в разрешённой форме унификатор $S$ "--- наиболее общий унификатор.
\end{statement}

\begin{statement}
    Несовместная система не имеет решений.
\end{statement}

Рассмотрим следующие преобразования, не меняющие свойств системы:
\begin{center}
\begin{tabular}{l l l} \toprule
    Выражения                         & Условия             & Новые выражения \\ \midrule
    $T=x$, $T$ не переменная          &                     & $x=T$ \\ \midrule
    $T=T$                             &                     & убрать это уравнение \\ \midrule
    \multirow{2}{*}[-\aboverulesep]{$f(A_1, \ldots A_n) = g(B_1, \ldots B_n)$}
                                      & $f=g$               & $ A_1 = B_1 \ldots A_n = B_n$ \\ \cmidrule{2-3}
                                      & $f \neq g$          & система несовместна \\ \midrule
    \multirow{2}{*}[-\aboverulesep]{$x=T$, $R=S$, $x$ входит в $S$ или $R$}
                                      & $T$ не содержит $x$ & $x=T$,
                                        $R\left[x\coloneqq T\right]=S\left[x\coloneqq T\right]$\\ \cmidrule{2-3}
                                      & $T$ содержит $x$    & система несовместна \\ \bottomrule
\end{tabular}
\end{center}

\begin{statement}
    Последовательное применение преобразований либо за конечное число шагов приведёт систему в разрешённый вид,
    либо покажет, что она несовместна.
\end{statement}

\begin{proof}
    Пусть $(n_1, n_2, n_3)$ "--- характеристика системы, где
    $n_1$ "--- количество переменных не входящих систему только слева от знака равенства один раз,
    $n_2$ "--- общее количество вхождений функциональных символов в $S$,
    $n_3$ "--- количество выражений вида $T=x$ или $T=T$.
    Каждое преобразование уменьшает эту тройку (если сравнивать лексикографически).
\end{proof}

\begin{theorem}
    Задача вывода типа в $\lambda$-исчислении разрешима.
\end{theorem}

\begin{proof}
    Опишем алгоритм. Пусть нам дан $\lambda$-терм $M$. Рекурсивно построим по нему систему уравнений $E_m$ и тип выражения $\tau_m$:
    \begin{center}
    \begin{tabular}{l l l} \toprule
        Вид $M$         & $E_m$                                                 & $\tau_m$                         \\ \midrule
        $x$             & $\varnothing$                                         & $\alpha_x$ "--- новая переменная \\ \midrule
        $PR$            & $E_p \cup E_r \cup \set{\tau_p=\tau_r\rightarrow\pi}$ & $\pi$ "--- новая переменная      \\ \midrule
        $\lambda x . P$ & $E_p$                                                 & $\tau_x\rightarrow\tau_p$        \\ \bottomrule
    \end{tabular}
    \end{center}
    Решим построенную систему уравнений.
    Пусть унификатор $S$ "--- решение системы уравнений $E_m$. Тогда тип $M$ это $S(\tau_m)$.

    Можно показать, что алгоритм корректный.
\end{proof}

Отметим важную для реализации деталь. Одной переменной должен соответствовать только один тип.
Однако не всегда переменные, называющиеся одинаково, мы считаем одинаковыми.
В выражении $\lambda f x . f \reduction x \reduction x$ важно, чтобы выделенные переменные получили один тип.
Но в выражении $\lambda x . \reduction x \lambda x . \reduction x \lambda x . \reduction x$ типы выделенных переменных должны быть разными.
Его тип должен быть такой же, как и, например, тип выражения $\lambda x . x \lambda y . y \lambda z . z$.
Можно сформулировать правило: построенная система уравнений должна оставаться корректной для любого выражения,
$\alpha$-эквивалентного исходному.

\begin{example}
    Рассмотрим выражение $\overline 2 = \lambda f . \lambda x . f (f x)$. Сначала построим по нему систему уравнений
    ($M$ "--- аргумент функции, столбики справа "--- рекурсивные вызовы, $E$ и $\tau$ "--- результат функции):
    \begin{center}
        \begin{tabular}{l l l l l} \toprule
            $M = \lambda f x . f (f x)$ & $M = \lambda x . f (f x)$ & $M = f (f x)$ & $M = f$ & \\
            & & & $E = \varnothing$ & \\
            & & & $\tau = \alpha$ \\ \cmidrule{4-5}
            & & & $M = f x$ & $M = f$ \\
            & & & & $E = \varnothing$ \\
            & & & & $\tau = \alpha$ \\ \cmidrule{5-5}
            & & & & $M = x$ \\
            & & & & $E = \varnothing$ \\
            & & & & $\tau = \beta$ \\ \cmidrule{5-5}
            & & & $E = \set{\alpha = \beta \rightarrow \gamma}$ & \\
            & & & $\tau = \gamma$ & \\ \cmidrule{4-5}
            $E = \begin{cases}
              \alpha=\beta\rightarrow\gamma \\
              \alpha=\gamma\rightarrow\delta
            \end{cases}\mkern-18mu$ &
            $E = \begin{cases}
                \alpha=\beta\rightarrow\gamma \\
                \alpha=\gamma\rightarrow\delta
            \end{cases}\mkern-18mu$ &
            $E = \begin{cases}
                \alpha=\beta\rightarrow\gamma \\
                \alpha=\gamma\rightarrow\delta
            \end{cases}\mkern-18mu$ & & \\[1.2em]
            $\tau = \alpha\rightarrow(\beta\rightarrow\delta)$ &
            $\tau = \beta\rightarrow\delta$ &
            $\tau = \delta$ & & \\ \bottomrule
        \end{tabular}
    \end{center}
    Затем решим полученную систему уравнений:
    \[
        \begin{cases}
            \alpha=\beta\rightarrow\gamma \\
            \alpha=\gamma\rightarrow\delta
        \end{cases} \mkern-18mu
        \implies
        \begin{cases}
            \alpha=\beta\rightarrow\gamma \\
            \beta\rightarrow\gamma=\gamma\rightarrow\delta
        \end{cases} \mkern-18mu
        \implies
        \begin{cases}
            \alpha=\beta\rightarrow\gamma \\
            \beta=\gamma \\
            \gamma=\delta
        \end{cases} \mkern-18mu
        \implies
        \begin{cases}
            \alpha=\beta\rightarrow\delta \\
            \beta=\delta \\
            \gamma=\delta
        \end{cases} \mkern-18mu
        \implies
        \begin{cases}
            \alpha=\delta\rightarrow\delta \\
            \beta=\delta \\
            \gamma=\delta
        \end{cases}
    \]
    Получаем подстановку $S = \left[\alpha\coloneqq\delta\rightarrow\delta, \beta\coloneqq\delta, \gamma\coloneqq\delta\right]$.
    Применяя её к $\tau$ получаем тип выражения:
    \begin{gather*}
        S(\tau) = (\delta\rightarrow\delta)\rightarrow(\delta\rightarrow\delta) \\
        \lambda f \lambda x . f (f x) : (\delta\rightarrow\delta)\rightarrow(\delta\rightarrow\delta)
    \end{gather*}
\end{example}

%\subsection{\texorpdfstring{Про ложь}{About false}}
%
%\todo\ послушать запись. %todo
