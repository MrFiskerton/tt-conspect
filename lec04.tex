\section{\texorpdfstring{Задачи в $\lambda$-исчислении}{Link to programming}}

\epigraph{Помните, что в $\lambda$-исчислении нет смысла? Здесь смысл отрицательный, скорее.}{Д.Г.}

В $\lambda$-исчислении выделяют 3 задачи:
\begin{enumerate}[label=(\asbuk*)]
    \item Проверка типа: верно ли $\Gamma \vdash M : \sigma$?
    \item Вывод типа: $? \vdash M : ?$
    \item Обитаемость типа: $? \vdash ? : \sigma$
\end{enumerate}

В этом разделе будем рассматривать задачу вывода типа.

\subsection{\texorpdfstring{Вывод типа}{Type deduction}}

\begin{definition}[Алгебраический терм]
    \begin{gather*}
        \begin{bnf}
            A ::= x | f\left(A, \ldots, A\right)
        \end{bnf} \\
        (x \in X)
    \end{gather*}
\end{definition}

Уравнение в алгебраических термах: $A = A$.

\begin{definition}[$S$-подстановка]
    \[
        S : X \rightarrow A
    \]
    Причём $S$ "--- id почти везде. (везде кроме конечного количества)
\end{definition}

\begin{definition}[Естественное обобщение]
    Естественное обобщение "--- такая подстановка $S : A \rightarrow A$, что
    $S\left(f\left(A_1, \dots, A_n\right)\right) = f\left(S(f_1), \ldots, S(f_n)\right)$
\end{definition}

\begin{definition}[Унификатор]
    $S$ "--- унификатор (решение уравнения) $P=Q$, если $S(P)=S(Q)$.
\end{definition}
Задача решения уравнение в алгебраических термах "--- унификация.

\begin{definition}[Композиция]
    $(S \circ T)(A) = S(T(A))$
\end{definition}

\begin{definition}[Частный случай]
    $T$ "--- частный случай $U$, если существует такое $S$, что $T = S \circ U$.
\end{definition}

\begin{definition}[Наибольший общий унификатор]
    Наибольший общий унификатор $U$ для уравнения $A=B$ "--- такой унификатор, что:
    \begin{enumerate}
        \item $U(A)=U(B)$.
        \item Любой другой унификатор "--- частный случай $U$.
    \end{enumerate}
\end{definition}

\begin{definition}[Несовместная система]
    Назовём систему несовместной, если выполнено одно из условий:
    \begin{enumerate}
        \item в ней есть уравнение вида $f(\ldots)=g(\ldots)$.
        \item в ней есть уравнение вида $x = \ldots x \ldots$.
    \end{enumerate}
\end{definition}


\begin{definition}[Эквивалентные системы]
    Назовём две системы эквивалентными, если они имеют одинаковые решения.
\end{definition}

\begin{statement}
    Для любой системы
    \[
        \begin{cases} A_1 = B_1 \\ \vdots \\ A_n = B_n \end{cases}
    \]
    найдётся эквивалентная ей система из одного уравнения:
    \[
        f(A_1, \ldots, A_n) = f(B_1, \ldots, B_n)\text{,}
    \]
    где $f$ "--- новый символ.
\end{statement}

\begin{definition}[Разрешённая система]
    Назовём систему разрешённой, если:
    \begin{enumerate}
        \item Все уравнения имеют вид $x = A$,
        \item Все переменные в левой части встречаются однократно.
    \end{enumerate}
\end{definition}

Решение по системе в разрешённой форме строится так:
\[
    S(x_i)=A_i
\]
\todo

\begin{statement}
    $S$ "--- наибольший общий унификатор.
\end{statement}

\begin{statement}
    Несовместная система не имеет решений.
\end{statement}

Рассмотрим следующие 4 преобразования:
\begin{enumerate}[label=(\asbuk*)]
    \item $T=x$, где $T$ не переменная $\Rightarrow$ $x=T$
    \item $x=x$ $\Rightarrow$ $\varepsilon$
    \item $s$
\end{enumerate}

бе
