\subsection*{\texorpdfstring{Экзистенциальные типы}{Existential types}}

\todo
Допустим, у нас есть абстрактный тип данных "<Стек">: \\
\begin{tabular}{l l}
    $\comb{empty}$ & $: \alpha$ \\
    $\comb{push}$  & $: \alpha\with\nu\rightarrow\alpha$ \\
    $\comb{pop}$   & $: \alpha\rightarrow\alpha\with\nu$ \\
\end{tabular} \\
Можно попробовать сказать это так: $\mathtt{stack} =
    \alpha \with (\alpha\with\nu\rightarrow\alpha) \with (\alpha\rightarrow\alpha\with\nu)$.
Но проблема в том, что у нас есть только интерфейс стека, а не его реализация. Поэтому лучше будет сказать так:
    $\exists \alpha . \alpha \with (\alpha\with\nu\rightarrow\alpha) \with (\alpha\rightarrow\alpha\with\nu)$.
То есть существует какое-то $\alpha$, реазизовывающее требуемый интерфейс.
\todo 

\ 

По аналогии с правилом удаления квантора существования, можно определить правила вывода для выражений экзистенциальных типов:
\[
    \infer{\Gamma\vdash (\pack{M, \theta}{\exists \alpha . \varphi}) : \exists \alpha.\varphi}
        {\Gamma \vdash M : \varphi[\alpha := \theta]} \qquad
    \infer[(\alpha \notin \FV(\Gamma, \psi))]
        {\Gamma \vdash \abstype{\alpha}{x:\varphi}{M}{N:\psi}}
        {\Gamma \vdash M : \exists \alpha . \varphi && \Gamma, x : \varphi \vdash N : \psi}
\]
Если вспомнить, что квантор существования выразим через квантор всеобщности, то можно попытаться записать типы выражений
\texttt{pack} и \texttt{abstype} через квантор существования и выразить их без расширения языка.
\begin{gather*}
    \pack{M, \theta}{\exists \alpha . \varphi} =
        \Lambda \beta . \lambda x^{\forall \alpha . \varphi \rightarrow \beta} . x \theta M \\
    \abstype{\alpha}{x : \varphi}{M}{N}:\psi =
        M \psi (\Lambda \alpha . \lambda x ^ \varphi . N)
\end{gather*}
Можно показать, что $\abstype{\alpha}{x:\varphi}{(\pack{M,\theta}{\exists\alpha .\varphi})}{N}$
        редуцируется в $N[\alpha\coloneqq\theta][x\coloneqq M]$:
\begin{align*}
    \abstype{\alpha}{x:\varphi}{(\pack{M,\theta}{\exists\alpha .\varphi})}{N}
    &= (\reduction{\Lambda \beta} . \lambda x^{\forall \alpha . \varphi \rightarrow \beta} . x \theta M)
        \reduction{\psi} (\Lambda \alpha . \lambda x ^ \varphi . N) \\
    &\rightarrow_\beta (\reduction{\lambda x^{\forall \alpha . \varphi \rightarrow \psi}} . x \theta M)
        \reduction{(\Lambda \alpha . \lambda x ^ \varphi . N)} \\
    &\rightarrow_\beta (\reduction{\Lambda \alpha} . \lambda x^\varphi . N) \reduction\theta M \\
    &\rightarrow_\beta (\reduction{\lambda x^\varphi} . N)[\alpha\coloneqq\theta] \reduction M \\
    &\rightarrow_\beta N[\alpha\coloneqq\theta][x\coloneqq M]
\end{align*}

\begin{example} \todo там чёрт ногу сломит \textbf{:/}
\end{example}

\begin{statement}
    $F$ сильно нормализуемо.
\end{statement}

\begin{statement}
    $F$ неразрешима.
\end{statement}
Ни одна из задач $\lambda$-исчисления в системе $F$ не разрешима, даже задача проверки типизации.
Доказать это можно через сведение к проблеме останова.

Итак, мы попытались добавить к типизированному $\lambda$-исчислению полиморфизм и абстрактные типы данных и получили слишком сложный язык.
Давайте попробуем немного его упростить, чтобы с ним можно было работать.

\subsection{\texorpdfstring{Типовая система Хиндли-Милнера}{Hindley and Milner’s type system}}

\begin{definition}[ранг типа]
\[
    \rank(\tau) =
    \begin{cases}
        \max(\rank(\sigma)+1, \rank(\rho)) & \tau \equiv \sigma \rightarrow \rho\text{, если }\sigma\text{ содержит }\forall \\
        \rank(\rho) & \tau \equiv \sigma \rightarrow \rho\text{, если }\sigma \text{ не содержит } \forall \\
        0 & \tau \equiv \alpha \\
        \max(\rank(\rho), 1) & \tau \equiv \forall \alpha . \rho
    \end{cases}
\]
\end{definition}

\begin{example} Ранг экзистенциального типа:
\begin{align*}
    \rank(\exists \alpha . \beta) &= \rank(\forall \gamma . (\forall \alpha . \beta \rightarrow \gamma) \rightarrow \gamma) \\
    &= \max(\rank((\forall \alpha . \beta \rightarrow \gamma) \rightarrow \gamma), 1) \\
    &= \max(\max(\rank(\forall \alpha . \beta \rightarrow \gamma) + 1, \rank(\gamma)), 1) \\
    &= \max(\max(2, 0), 1) = 2
\end{align*}
\end{example}

\begin{definition}[тип в системе Хиндли-Милнера] \ \\
    Тип (монотип) "--- выражение в грамматике $ \begin{bnf} \tau ::= \alpha | \tau \rightarrow \tau | (\tau) \end{bnf} $. \\
    Типовая схема (политип) "--- выражение в грамматике $ \begin{bnf} \sigma ::= \tau | \forall \alpha . \sigma \end{bnf} $.
\end{definition}

$\forall\alpha.\alpha\rightarrow\alpha$ "--- политип, $\forall\alpha.\alpha\rightarrow\forall\beta.\beta$ "--- некорректный в системе Хиндли-Милнера тип.

\begin{statement}
    $\rank(\tau) = 0$, $\rank(\sigma) = 1$.
\end{statement}

\begin{definition}[подтип]
    $\sigma_1$ "--- подтип $\sigma_2$, если существует такая подстановка
            $[\alpha_1 \coloneqq \theta_1, \alpha_2 \coloneqq \theta_2 \ldots \alpha_n \coloneqq \theta_n]$, что:
    \begin{enumerate}
        \item $\sigma_1 = \forall \beta_1 \ldots \forall \beta_k . \tau [\alpha_1 \coloneqq \theta_1 \ldots \alpha_n := \theta_n]$,
            $\alpha_i$ не входит свободно в $\theta_j$
        \item $\sigma_2 = \forall \alpha_1 \ldots \forall \alpha_n \tau$
    \end{enumerate}
\end{definition}


\begin{definition}[система Хиндли-Милнера] Грамматика:
\begin{bnf}
\[
    \Lambda ::= x | \lambda x . \Lambda | \Lambda \Lambda | (\Lambda) | \lett{x=\Lambda}{\Lambda}
\]
\end{bnf}%
Правила вывода:
\inferspacing
\begin{gather*}
    \infer[\text{(Тавтология)}]{\Gamma, x : \sigma \vdash x : \sigma}{} \qquad
    \infer[\text{(Уточнение, $\sigma'$ "--- подтип $\sigma$)}]{\Gamma \vdash e : \sigma'}{\Gamma \vdash e : \sigma} \\
    \infer[\text{(Обобщение, $\alpha\notin\FV(\Gamma)$)}]{\Gamma \vdash e : \forall \alpha . \sigma}{\Gamma \vdash e : \sigma} \qquad
    \infer[\text{(Абстракция)}]{\Gamma \vdash \lambda x . e : \sigma \rightarrow \tau}{\Gamma, x : \sigma \vdash e : \tau} \\
    \infer[\text{(Применение)}]
        {\Gamma \vdash e e' : \tau}{\Gamma \vdash e : \sigma \rightarrow \tau && \Gamma \vdash e' : \sigma} \qquad
    \infer[\text{(Let)}]
        {\Gamma \vdash \lett{x=e}{e':\tau}}
        {\Gamma \vdash e : \sigma && \Gamma, x : \sigma \vdash e' : \tau}
\end{gather*}
\end{definition}

Мы существенно ограничили набор возможных типов в нашем языке, однако, он всё ещё вполне сильный.
Например, в нём можно определить тип $\comb Y$: $\forall \alpha . (\alpha\rightarrow\alpha)\rightarrow\alpha$,
и явно определить оператор фиксированной точки: $\mathtt{fix~f~=~f~(fix~f)}$.
Через него можно определять рекурсивные функции и они будут типизироваться.

\subsection{\texorpdfstring{Вывод типа}{Type inference}}
\begin{statement}
    Задача вывода типа в системе Хиндли-Милнера разрешима.
\end{statement}
Приведём алгоритм, решающий эту задачу.
Он будет принимать выражение $e$ в контексте $\Gamma$ и возвращать такую подстановку $S$ и тип $\tau$, что
\[
    S(\Gamma) \vdash e : \tau
\]

В алгоритме будем пользоваться \hyperref[unificator]{унификацией} ($U(\tau_1, \tau_2)$ "--- унификатор $\tau_1$ и $\tau_2$),
будем обозначать контекст $\Gamma$ без типа $x$ как $\Gamma_x$
и определим замыкание всех несвязанных типовых переменных в контексте:
\[
    \overline{\Gamma}(\tau) = \forall \alpha_1 \ldots \forall \alpha_n . \tau
\]
где $\alpha_i \in \FV(\tau)$ и $\alpha_i \notin \FV(\Gamma)$.

Алгоритм описан в таблице \ref{algorithm-w}.
Если какие-то условия не могут быть соблюдены, то тип выражения не может быть выведен.

%\begin{center}
\begin{table}[ht]
\centering
\begin{tabular}{Sl Sl Sl} \toprule
    Вид $e$ & Условия & $W(\Gamma, e)$ \\ \midrule
    $x$
        & $x : \forall \alpha_1 \ldots \alpha_k . \tau' \in \Gamma$ & $S =\mathrm{Id}$ \\
        & $\beta_i$ "--- новые переменные                           & $\tau = \tau'[\alpha_i \coloneqq \beta_i]$ \\
        \midrule
    $e_1 e_2$
            & $W(\Gamma, e_1) = (S_1, \tau_1)$                                       & $S = V \circ S_1 \circ S_2$ \\
            & $W(S_1(\Gamma), e_2) = (S_2, \tau_2)$                                  & $\tau = S(\beta)$ \\
            & $U(S_2(\tau_1), \tau_2 \rightarrow \beta) = V$, $\beta$ "--- новый тип & \\ \midrule
    $\lambda x . e$
        & $W(\Gamma_x \cup \set{x : \beta}, e) = (S_1, \tau_1)$ & $S = S_1$  \\
        & $\beta$ "--- новый тип                                & $\tau = S(\beta) \rightarrow \tau_1$ \\ \midrule
    $\lett{x=e_1}{e_2}$
        & $W(\Gamma, e_1) = (S_1, \tau_1)$                                                     & $S = S_2 \circ S_1$ \\
        & $W(S_1 \Gamma_x \cup \set{x : \overline{S_1 \Gamma} (\tau_1)}, e_2) = (S_2, \tau_2)$ & $\tau = \tau_2$ \\ \bottomrule
\end{tabular}
\caption{Алгоритм $W$.}
\label{algorithm-w}
\end{table}
%\end{center}

\begin{example}
\todo
\end{example}
