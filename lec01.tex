\section{\texorpdfstring{\textlambda-исчисление}{Lambda calculus}}

\subsection{\texorpdfstring{Введение}{Introduction}}
\epigraph{Смысла в этом нет.}{Д.Г.}

\begin{definition}[$\lambda$-выражение]
    $\lambda$-выражение "--- выражение, удовлетворяющее грамматике:
    \begin{bnf}
    \begin{alignat*}{3}
        \Lambda ::= & \lambda{}x.\Lambda{} \qquad && (абстракция) \\
                  | & \Lambda{}\Lambda{}          && (аппликация) \\
                  | & x                                           \\
                  | & \left(\Lambda\right)
    \end{alignat*}
    \end{bnf}
    \begin{enumerate}[label=(\asbuk*)]
        \item Аппликация левоассоциативна.
        \item Абстракция распространяется как можно дальше вправо.
    \end{enumerate}
\end{definition}

\begin{example}
    $((\lambda{} z.(z (y z))) (z x) z) = (\lambda{} z.z (y z)) (z x) z$
\end{example}

Есть понятия связанного и свободного вхождения переменной (аналогично исчислению предикатов).
$\lambda{}x.A$ связывает все свободные вхождения $x$ в $A$.
\begin{definition} Функция $\FV(A)$ "--- множество свободных переменных, входящих в $A$:
    \[
    \FV(A) =
    \begin{cases}
        \set{x}                  & \text{если } A \equiv x \\
        \FV(P) \cup \FV(Q)       & \text{если } A \equiv PQ \\
        \FV(P) \setminus \set{x} & \text{если } A \equiv \lambda x . P
    \end{cases}
    \]
\end{definition}
Договоримся, что:
\begin{enumerate}[label=(\asbuk*)]
    \item Переменные "--- $x$, $a$, $b$, $c$.
    \item Термы (части $\lambda$-выражения) "--- $X$, $A$, $B$, $C$.
    \item Фиксированные переменные обозначаются буквами из начала алфавита, метапеременные "--- из конца.
\end{enumerate}

На самом деле, смысл в этом есть, $\lambda$-выражение можно понимать как функцию.
Абстракция "--- это функция с аргументом, аппликация "--- это передача аргумента.

\begin{definition}[$\alpha$-эквивалентность]
    $A$ и $B$ называются $\alpha$-эквивалентными ($A=_{\alpha}B$), если выполнено одно из следующих условий:
    \begin{enumerate}
        \item $A\equiv{}x$ и $B\equiv{}x$.
        \item $A\equiv{}\lambda{}x.P$, $B\equiv{}\lambda{}y.Q$ и $P [x\coloneqq{}t] =_{\alpha}Q [y\coloneqq{}t]$, где $t$ "--- новая переменная.
        \item $A\equiv{}PQ$, $B\equiv{}RS$ и $P=_{\alpha}R$, $Q=_{\alpha}S$.
    \end{enumerate}
\end{definition}

То есть два выражения $\alpha$-эквивалентны, если они равны с точностью до переименования абстракций.
В C++ между функциями \mintinline{C++}{[](int x){return f(x);}} и \mintinline{C++}{[](int y){return f(y);}} нет разницы.

\begin{example}
    $\lambda{}x.\lambda{}y.xy=_{\alpha}\lambda{}y.\lambda{}x.yx$.
    \begin{proof} Согласно второму правилу следующие утверждения верны:
        \begin{alignat*}{2}
            \lambda{}y.ty=_{\alpha}\lambda{}x.tx &\implies \lambda{}x.\lambda{}y.xy=_{\alpha}\lambda{}y.\lambda{}x.yx \\
            tz=_{\alpha}tz &\implies \lambda{}y.ty=_{\alpha}\lambda{}x.tx
        \end{alignat*}%
        $tz=_{\alpha}tz$ верно по третьему условию.
    \end{proof}
\end{example}

\begin{definition}[$\beta$-редекс]
    Терм вида $\left(\lambda{}a.A\right)B$ называется $\beta$-редексом.
\end{definition}

\begin{example}
    В выражении
    $
        (
            \lambda{}f.
                \underset{A_2}{\underline{
                    (\lambda{}x.\overset{A_1}{\overline{f(xx)}})
                    \overset{B_1}{\overline{(\lambda{}x.f(xx))}}
                }}
        )\underset{B_2}{\underline{g\vphantom{f\left(xx\right)}}}
    $ два $\beta$-редекса.
\end{example}

\begin{definition}
    Множество $\lambda$-термов $\boldsymbol{\Lambda}$ назовём множеством классов эквивалентности $\Lambda$ по $(=_{\alpha})\footnotemark$.
\end{definition}
\footnotetext
{Если $\circ$ "--- двухместный оператор, то $\left(\circ\right)$ "--- это обозначение функции $\left(\circ\right) a b = a \circ b$.
$\left(a \operatorname{\circ}\right)$ и $\left(\operatorname\circ b\right)$ это "<сечения"> оператора, передача ему одного из аргументов.
Например, $\left(\operatorname\div 3\right)$ "--- это функция деления на три,
а $\left(2^\wedge\right)$ "--- это функция возведения двойки в степень.}

\begin{definition}[$\beta$-редукция]
    $A\to_{\beta}B$ ($A$ и $B$ состоят в отношении $\beta$-редукции), если выполняется одно из условий:
    \begin{enumerate}
        \item $A\equiv{}PQ$, $B\equiv{}RS$ и либо $P\to_{\beta}R$ и $Q=_{\alpha}S$,
            либо $P=_{\alpha}R$ и $Q\to_{\beta}S$.
        \item $A\equiv{}\lambda{}x.P$, $B\equiv{}\lambda x.Q$, $P\to_{\beta}Q$ ($x$ из какого-то класса из $\boldsymbol{\Lambda}$).
        \item $A\equiv{}(\lambda{}x.P)Q$, $B\equiv{}P [x\coloneqq{}Q]$, $Q$ свободно для подстановки в $P$ вместо $x$.
    \end{enumerate}
\end{definition}

$\beta$-редукция "--- это вычисление функции, подстановка её аргумента.

\begin{example} $(\lambda x . x y) a \to_\beta a y$.
\end{example}

Будем обозначать транзитивно-рефлексивное замыкание $\left(\to_\beta\right)$ как $\left(\twoheadrightarrow_{\beta}\right)$
и называть его $\beta$-редуцируемостью.
Также для удобства введём сокращение записи:
\[
    \lambda x y . A = \lambda x . \lambda y . A
\]


\subsection{\texorpdfstring{Числа Чёрча}{Church numerals}}
\epigraph{Хотите знать, что такое истина?}{Д.Г.}

\newcommand{\T}{\func{T}}
\newcommand{\F}{\func{F}}
\newcommand{\Not}{\func{Not}}
\begin{gather*}
    \T   = \lambda{}x\lambda{}y.x \qquad
    \F   = \lambda{}x\lambda{}y.y \\
    \Not = \lambda{}a.a\F\T
\end{gather*}

Похоже на тип \mintinline{Pascal}{boolean}, не правда ли?
\begin{example} Отрицание истины "--- это ложь:
    \[
        \Not \T = (\reduction{\lambda{}a}.a\F\T)\reduction\T \to_{\beta}
            \T\F\T = (\reduction{\lambda{}x}.\lambda{}y.x)\reduction\F\T \to_{\beta}
            (\reduction{\lambda{}y}.\F)\reduction\T \to_{\beta}
            \F
    \]
\end{example}
Истина "--- это функция, которая принимает два аргумента и возвращает первый аргумент.
Ложь принимает два аргумента и возвращает второй.
Если в выражении $a \F \T$ терм $a$ является истиной, то результатом будет первый аргумент, $\F$. Если ложью "--- второй аргумент, $\T$.
%Получается, что истиностное выражение также является фукнцией $\mathtt{if}$ от себя самого.

Можно продолжить:
\[
    \func{And} = \lambda{}a.\lambda{}b.ab\F \qquad
    \func{Or}  = \lambda{}a.\lambda{}b.a\T b
\]
Разберём $\func{And}$. Он принимает два аргумента.
Если первый аргумент это истина, то тогда результат $\func{And}$ равен второму аргументу.
Если первый аргумент это ложь, то тогда результат $\func{And}$ это ложь, и от второго аргумента не зависит.
Это и записано в выражении. $\func{Or}$ разбирается аналогично.

Попробуем определить числа:
\begin{definition}[чёрчевский нумерал]
\[
    \overline{n}=\lambda{}f.\lambda{}x.f^{n}x \text{,\quadгде\quad}
    f^{n}x=
    \begin{cases}
        f\left(f^{n-1}x\right) & \text{если } n > 0 \\
        x                      & \text{если } n = 0
    \end{cases}
\]
\end{definition}

\begin{example}
\[
    \overline{3} = \lambda f . \lambda x . f \left(f \left(f x\right)\right)
\]
\end{example}

Несложно определить прибавление единицы к такому нумералу:
\[
    (+1) = \lambda{}n.\lambda{}f.\lambda{}x.f(nfx) \\
\]
\begin{example}
    \[
        (+1) \overline{1} =
        (\reduction{\lambda n} . \lambda f . \lambda x . f (n f x)) \reduction{(\lambda f . \lambda x . f x)} \to_{\beta}
        \lambda f . \lambda x . f ((\reduction{\lambda f . \lambda x} . f x) \reduction{f x}) \twoheadrightarrow_{\beta}
        \lambda f . \lambda x . f (f x) =
        \overline{2}
    \]
\end{example}

\begin{definition}[$\eta{}$-эквивалентность]
    \[
        \lambda x . f x =_{\eta} f
    \]
\end{definition}
Точно так же результаты вычисления \mintinline{C++}{int f(int x)} и \mintinline{C++}{[](int x){return f(x);}} равны.

Арифметические операции:
\begin{align*}
    \func{IsZero} &= \lambda{}n.n\mathinner{(\lambda{}x.\F)}\T &\qquad
    \func{Add}    &= \lambda{}a.\lambda{}b.\lambda{}f.\lambda{}x.a \mathinner{f} (b \mathinner{f} x) &\qquad
    \func{Pow}    &= \lambda{}a.\lambda{}b.b \mathinner{(\func{Mul} a)} \mathinner{\overline{1}} \\
    \func{IsEven} &= \lambda{}n.n \Not \T &\qquad
    \func{Mul}    &= \lambda{}a.\lambda{}b.a \mathinner{(\func{Add} b)} \mathinner{\overline{0}} &\qquad
    \func{Pow'}   &= \lambda{}a.\lambda{}b.b a
\end{align*}
Для того, чтобы определить $(-1)$, сначала определим "<пару">:
\[
    \left<a,b\right> = \lambda f.f \mathinner{a} b \qquad
    \func{First} = \lambda p . p \T \qquad
    \func{Second} = \lambda p . p \F
\]%
Затем $n$ раз применим функцию $f\left(\left<a,b\right>\right) = \left<b,b+1\right>$ и возьмём первый элемент пары:
\[
    (-1) = \lambda n . \func{First}
        (n \mathinner{(\lambda p . \left<\left(\func{Second} p\right), \mathinner{(+1)} (\func{Second} p)\right>)}
        \langle\overline{0},\overline{0}\rangle)
\]

\begin{definition}[нормальная форма]
    Терм $A$ "--- нормальная форма, если в нём нет $\beta$-редексов.
    Нормальной формой $A$ называется такая нормальная форма $B$, что $A \twoheadrightarrow_{\beta} B$.
\end{definition}

\begin{statement}
    Существует $\lambda$-выражение, не имеющее нормальной формы.
\end{statement}

\begin{definition}[комбинатор]
    Комбинатор "--- $\lambda$-выражение без свободных переменных.
\end{definition}

\begin{definition}
\[
    \combl\Omega = \combl\omega \combl\omega \qquad
    \combl\omega = \lambda x . x x
\]
\end{definition}

$\combl\Omega$ не имеет нормальной формы.

\begin{definition}[комбинатор неподвижной точки]
    \[
        \comb Y = \lambda f . (\lambda x . f (x x)) (\lambda x . f (x x))
    \]
\end{definition}

\begin{definition}[$\beta$-эквивалентность]
    $A=_{\beta}B$, если $\exists C : C \twoheadrightarrow_{\beta} A, C \twoheadrightarrow_{\beta}B$
\end{definition}

\begin{statement}
    \[
        \comb Yf =_{\beta} f(\comb Yf)
    \]
\end{statement}

Из-за этого свойста $\comb Y$-комбинатор и получил своё название.

\begin{proof} Наивное $\beta$-редуцирование:
    \begin{align*}
        \comb Yf &=_{\beta} (\reduction{\lambda f} . (\lambda x . f (x x)) (\lambda x . f (x x))) \reduction f \\
                 &=_{\beta} (\reduction{\lambda x} . f (x x)) \reduction{(\lambda x . f (x x))} \\
                 &=_{\beta} f ((\lambda x . f (x x)) (\lambda x . f (x x))) \\
                 &=_{\beta} f (\comb Y f)
    \qedhere
    \end{align*}
\end{proof}

Таким образом, с помощью $\comb Y$-комбинатора можно определять рекурсивные функции.
\begin{example} Определим факториал $\func{Fact}$ с помощью вспомогательной функции $\func{Fact'}$.
\begin{align*}
    \func{Fact'} &= \lambda{} f n . \func{IsZero} n \mathinner{\overline{1}}
                    (\func{Mul} n \mathinner{(f \mathinner{(\mathinner{(-1)} n))}}) \\
    \func{Fact} &= \func Y \func{Fact'} \\
    \func{Fact} \overline 3 &=_\beta \comb Y \func{Fact'} \overline 3 =_\beta \func{Fact'} \mathinner{(\comb Y \func{Fact'})} \overline 3
                \\&=_\beta \func{IsZero} \overline 3 \mathinner{\overline{1}}
                    (\func{Mul} \overline 3 \mathinner{(\comb Y \func{Fact'} \mathinner{(\mathinner{(-1)} \overline 3)})})
                =_\beta \func{Mul} \overline 3 \mathinner{(\comb Y \func{Fact'} \mathinner{(\mathinner{(-1)} \overline 3)})}
                =_\beta \func{Mul} \overline 3 \mathinner{(\comb Y \func{Fact'} \overline 2)} \\&=_\beta \ldots
                =_\beta \func{Mul} \overline 3 \mathinner{(\func{Mul} \overline 2 \mathinner{(\comb Y \func{Fact'} \overline 1)})}
                =_\beta \ldots =_\beta \func{Mul} \overline 3 \mathinner{(\func{Mul} \overline 2 \mathinner{(
                    \func{Mul} \overline 1 \mathinner{(\comb Y \func{Fact'} \overline 0)})})} =_\beta \ldots \\
                &=_\beta \func{Mul} \overline 3 \mathinner{(\func{Mul} \overline 2 \mathinner{(
                    \func{Mul} \mathinner{\overline 1} \overline 1)})}
                =_\beta \overline 6
\end{align*}
\end{example}
