\section{\texorpdfstring{Просто типизированное $\lambda$-исчисление}{Simply typed lambda calculus}}

\begin{definition}[Тип]
    $T = \{\alpha, \beta, \gamma, \ldots\}$ "--- множество типов.
    $\sigma$, $\tau$ "--- метапеременные для типов.
    Если $\tau$, $\sigma$ "--- типы, то $\sigma \rightarrow \tau$ "--- тип.
    \begin{bnf}
    \[
        \Pi ::= T | \Pi \rightarrow \Pi | (\Pi)
    \]
    \end{bnf}
    $\left(\rightarrow\right)$ правоассоциативна.
\end{definition}

\begin{definition}[Контекст] Контекст "--- $\Gamma$.
\begin{gather*}
    \Gamma = \{ \Lambda_{1} : \sigma_{1};\ \Lambda_{2} : \sigma_{2}\ \ldots\ \Lambda_{n} : \sigma_{n} \} \\
    \abs{\Gamma} = \{ \sigma_{1},\ \sigma_{2}\ \ldots\ \sigma_{n} \} \\
    \mathrm{dom}\ \Gamma = \{ \Lambda_{1},\ \Lambda_{2}\ \ldots\ \Lambda_{n} \}
\end{gather*}
\end{definition}

\subsection{\texorpdfstring{Исчисление по Карри}{Curry-style}}

\begin{definition}[Типизируемость по Карри]
    Рассмотрим исчисление со следующими правилами:
    \begin{enumerate}
        \item $\infer[(x \notin \mathrm{dom}(\Gamma))]
            {\Gamma, x:\sigma \vdash x:\sigma}
            {}$
        \item $\infer[]
            {\Gamma \vdash M N : \tau}
            {\Gamma \vdash M:\sigma \rightarrow \tau & \Gamma \vdash N:\sigma}$
        \item $\infer[(x \notin \mathrm{dom}(\Gamma))]
            {\Gamma \vdash \lambda x . M : \sigma \rightarrow \tau}
            {\Gamma, x : \sigma \vdash M : \tau}$
    \end{enumerate}

    %\begin{gather*}
    %\infer[(x\ \cancel{\in}\ \mathrm{dom}(\Gamma))]
    %    {\Gamma, x:\sigma \vdash x:\sigma}
    %    {} \\
    %\infer[]
    %    {\Gamma \vdash M N : \tau}
    %    {\Gamma, M:\sigma \rightarrow \tau & \Gamma \vdash N:\sigma} \qquad
    %\infer[(x\ \cancel{\in}\ \mathrm{dom}(\Gamma))]
    %    {\Gamma \vdash \lambda x . M : \sigma \rightarrow \tau}
    %    {\Gamma, x : \sigma \vdash M : \tau}
    %\end{gather*}

    Если $\lambda$-выражение типизируется этими трёмя правилами, то говорят, что оно типизируется по Карри.
\end{definition}

\begin{lemma}[subject deduction]
    Если $\Gamma \vdash M : \sigma$ и $M \rightarrow_{\beta}N$, то $\Gamma \vdash N : \sigma$.
\end{lemma}

\begin{corollary}
    Если $\Gamma \vdash M : \sigma$ и $M \twoheadrightarrow_{\beta}N$, то $\Gamma \vdash N : \sigma$.
\end{corollary}

\begin{theorem}[Чёрча-Россера]
    Если $\Gamma \vdash M : \sigma$, $M \twoheadrightarrow_{\beta} N$ и $M \twoheadrightarrow_{\beta} P$, тогда найдётся $Q$, что
    $N \twoheadrightarrow_{\beta} Q$, $P \twoheadrightarrow_{\beta} Q$ и $\Gamma \vdash Q : \sigma$.
\end{theorem}

\begin{example} Несколько доказательств:
    \begin{enumerate}
        \item Докажем $\lambda x . x : \alpha \rightarrow \alpha$:
        \[
            \infer[(3)]
                {\vdash \lambda x . x : \alpha \rightarrow \alpha}
                { \infer[(1)]
                    {x : \alpha \vdash x : \alpha}
                    {}
                }
        \]

        \item Докажем $\lambda f . \lambda x . f x : (\alpha \rightarrow \beta) \rightarrow \alpha \rightarrow \beta$:
        \[
            \infer[(3)]
                { \vdash \lambda f . \lambda x . f x : (\sigma \rightarrow \tau) \rightarrow (\sigma \rightarrow \tau) }
                { \infer[(3)]
                    { f : \sigma \rightarrow \tau \vdash \lambda x . f x : \sigma \rightarrow \tau }
                    { \infer[(2)]
                        {f : \sigma \rightarrow \tau; x : \sigma \vdash f x : \tau}
                        {
                            \infer[(1)]{ \Gamma \vdash f : \sigma \rightarrow \tau }{} &
                            \infer[(1)]{ \Gamma \vdash x : \sigma }{}
                        }
                    }
                }
        \]

        \item $\Omega = (\lambda x . x x) (\lambda x . x x)$ не типизируемо:
            \todo % TODO
    \end{enumerate}
\end{example}

\begin{lemma}[Свойство subject expansion]
    Неверно, что если $M \rightarrow_{\beta} N$, $\Gamma \vdash N : \sigma$, то $\Gamma \vdash M : \sigma$.
\end{lemma}
Например, для $Ka\Omega$.

В общем случае тип не уникален, бывает, что одновременно $\vdash \lambda x . x : \alpha \rightarrow \alpha$ и $\vdash \lambda x . x : (\beta \rightarrow \beta) \rightarrow (\beta \rightarrow \beta)$.

\begin{definition}[Сильная нормализация]
    Назовём исчисление сильно-нормализуемым, если любая последовательность редукций неизбежно приводит к нормальной форме (не существует бесконечной последовательности $\beta$-редукций) .
\end{definition}

\begin{definition}[Слабая нормализация]
    Назовём исчисление слабо-нормализуемым, если для любого терма существует последовательность $\beta$-редукций, приводящая его к нормальной форме.
\end{definition}

\begin{theorem}[о сильной нормализации]
    Просто типизируемое $\lambda$-исчисление сильно нормализуемо.
    Любое просто типизируемое $\lambda$-выражение сильно нормализуемо.
\end{theorem}

%\todo\{ это к чему и о чём вообще?
%\begin{theorem}
%    Если $\nu = (\alpha \rightarrow \alpha) \rightarrow (\alpha \rightarrow \alpha)$ и $F : \nu \rightarrow \nu \rightarrow \nu$, то $F$ "--- расширенный полином.
%\end{theorem}
%
%\begin{theorem}
%    Рассмотрим полином $E(m,n) =
%    \begin{cases}
%        \text{Полином}(m,n) & m > 0, n > 0 \\
%        \text{Полином}(m)   & m > 0, n = 0 \\
%        \text{Полином}(n)   & m = 0, n > 0 \\
%        \text{Константа}    & m = 0, n = 0
%    \end{cases}$. \\
%    Если $\nu = (\alpha \rightarrow \alpha) \rightarrow (\alpha \rightarrow \alpha)$ и $F : \nu \rightarrow \nu \rightarrow \nu$, то $F$ "--- рассматриваемый полином.
%\end{theorem}
%\} // \todo

\subsection{\texorpdfstring{Исчисление по Чёрчу}{Church-style}}

\begin{definition}[Типизация по Чёрчу]
    \begin{bnf}
    \[
        \Lambda_{\xx} ::= x | \lambda x^{\sigma}.\Lambda_{\xx} | (\Lambda_{\xx}) | \Lambda_{\xx} \Lambda_{\xx}
    \]
    \end{bnf}
    Правила:
    \begin{enumerate}
        \item $\infer[(x \notin \mathrm{dom}(\Gamma))]
            {\Gamma, x:\sigma \vdash_{\xx} x:\sigma}
            {}$
        \item $\infer[]
            {\Gamma \vdash_{\xx} M N : \tau}
            {\Gamma \vdash_{\xx} M:\sigma \rightarrow \tau & \Gamma \vdash_{\xx} N:\sigma}$
        \item $\infer[(x \notin \mathrm{dom}(\Gamma))]
            {\Gamma \vdash_{\xx} \lambda x^{\sigma} . M : \sigma \rightarrow \tau}
            {\Gamma, x : \sigma \vdash_{\xx} M : \tau}$
    \end{enumerate}

\end{definition}

\begin{definition}
\[
    \abs{\Lambda_{\xx}} =
    \begin{cases}
        x                                   & \Lambda_{\xx} \equiv x \\
        \abs{\Lambda_{1}} \abs{\Lambda_{2}} & \Lambda_{\xx} \equiv \Lambda_{1} \Lambda_{2} \\
        \lambda x . \abs{\Lambda}           & \Lambda_{\xx} \equiv \lambda x^{\sigma} . \Lambda
    \end{cases}
\]
\end{definition}

\begin{lemma}[Subject reduction по Чёрчу]
    Пусть $\Gamma \vdash_{\xx} M : \sigma$ и $\abs{M} \rightarrow_{\beta} N$. \\
    Тогда найдётся такое $H$, что $\abs{H} = N$, $\Gamma \vdash_{\xx} H:\sigma$.
\end{lemma}

\begin{theorem}[Чёрча-Россера]
    Пусть $\Gamma \vdash_{\xx} M : \sigma$, $\abs{M} \twoheadrightarrow_{\beta} N$, $\abs{M} \twoheadrightarrow_{\beta} T$. \\
    Тогда найдётся такое $P$, что $\Gamma \vdash_{\xx} P : \sigma$,
            $N \twoheadrightarrow_{\beta} \abs{P}$ и $T \twoheadrightarrow_{\beta} \abs{P}$.
\end{theorem}

\begin{lemma}[Уникальность типов] \label{uniqueness}
    Если $\Gamma \vdash_\xx M : \gamma$ и $\Gamma \vdash_\xx M : \tau$, то $\sigma = \tau$.
\end{lemma}

Лемма \ref{uniqueness} показывает, чем исчисление по Чёрчу отличается исчислением по Карри.

\begin{theorem}[о стирании] \ 
    \begin{enumerate}
        \item Если $M \rightarrow_{\beta} N$ и $\Gamma \vdash_{\xx} M : \sigma$, то $\abs{M} \rightarrow_{\beta} \abs{N}$.
        \item Если $\Gamma \vdash_{\xx} M : \sigma$, то $\Gamma \vdash_{к} \abs{M} : \sigma$.
    \end{enumerate}
\end{theorem}

\begin{theorem}[о поднятии]
    Пусть $P \in \Lambda_{\xx}$, $M, N \in \Lambda_{\rr}$.
    \begin{enumerate}
        \item Если $M \rightarrow_{\beta} N$, $\abs{P} = M$, то найдётся такое $Q$, что $\abs{Q} = N$, $P \rightarrow_{\beta} Q$.
        \item Если $\Gamma \vdash_{\rr} M : \sigma$, то найдётся такое $P \in \Lambda_{\xx}$, что
                $\Gamma \vdash_{\xx} P : \sigma$, $\abs{P} = M$.
    \end{enumerate}
\end{theorem}

\subsection{\texorpdfstring{Изоморфизм Карри-Ховарда}{Curry-style}}

\todo \ Многими логиками замечалась связь между выражениями типизированного $\lambda$-исчисления и доказательствами выражений ИИВ.

\renewcommand{\arraystretch}{1.5}

\setlength\cellspacetoplimit{5pt}
\setlength\cellspacebottomlimit{5pt}

\begin{center}
\begin{tabular}{Sc|Sc}
    \hline Правило вывода в ИИВ & Правило вывода в типизированном $\lambda$-исчислении \\ \hline

    $\infer{\Gamma \vdash \tau}{\Gamma \vdash \sigma \rightarrow \tau & \Gamma \vdash \sigma}$ &
    $\infer{\Gamma \vdash AB : \tau}{\Gamma \vdash A : \sigma \rightarrow \tau & \Gamma \vdash B : \sigma}$ \\ \hline

    $\infer{\Gamma \vdash \varphi \& \psi}{\Gamma \vdash \varphi & \Gamma \vdash \psi}$ &
    $\infer{\Gamma \vdash \left<A,B\right> : \varphi \& \psi}{\Gamma \vdash A : \varphi & \Gamma \vdash B : \psi}$ \\

    $\infer{\Gamma \vdash \varphi}{\Gamma \vdash \varphi \& \psi}$ &
    $\infer{\Gamma \vdash \pi_1 R : \varphi}{\Gamma \vdash R : \varphi \& \psi}$ \\

    $\infer{\Gamma \vdash \psi}{\Gamma \vdash \varphi \& \psi}$ &
    $\infer{\Gamma \vdash \pi_2 R : \psi}{\Gamma \vdash R : \varphi \& \psi}$ \\ \hline

    $\infer{\Gamma \vdash \varphi \vee \psi}{\Gamma \vdash \varphi}$ &
    $\infer{\Gamma \vdash \inj_1 A : \varphi \vee \psi}{\Gamma \vdash A : \varphi}$ \\

    $\infer{\Gamma \vdash \varphi \vee \psi}{\Gamma \vdash \psi}$ &
    $\infer{\Gamma \vdash \inj_2 A : \varphi \vee \psi}{\Gamma \vdash A : \psi}$ \\

    $\infer{\Gamma \vdash \varphi \vee \psi \rightarrow \pi}
        {\Gamma \vdash \varphi \rightarrow \pi & \Gamma \vdash \psi \rightarrow \pi}$ &
    $\infer{\Gamma \vdash \case{T}{A}{B} : \pi}{\Gamma \vdash T : \varphi \vee \psi &
        \Gamma \vdash A : \varphi \rightarrow \pi & \Gamma \vdash B : \psi \rightarrow \pi}$ \\ \hline
\end{tabular}
\end{center}

\todo\ Пояснения значений инъекций, проекций и case; примеры.

\begin{tabular}{Sc|Sc}
    Интуиционистская логика & $\lambda$-исчисление \\ \hline
    выражение & тип \\
    доказательство & терм (программа) \\
    предположение & свободная переменная \\
    импликация & абстракция (функция)
\end{tabular}

\begin{theorem}[об изоморфизме Карри-Ховарда] \ 
    \begin{enumerate}
        \item Пусть $\Gamma \vdash \sigma$ "--- и.ф.и.и.в., тогда найдётся такой терм M,
            что $\Delta \vdash_{\xx} M : \sigma$, где $\Delta=\left\{ \left(x_\varphi : \varphi \right) \mid \varphi \in \Gamma \right\}$.
        \item Пусть $\Delta \vdash_{\xx} M : \sigma$, тогда $\abs{\Delta} \vdash \sigma$.
    \end{enumerate}
\end{theorem}

%% долг с предыдущей лекции
\begin{proof} \todo
    %есть в Curry-Howard Isomorphism, стр 75

    %Рассмотрим $x_\varphi$, $\varphi \in \Gamma$.
    %$\Gamma = \{\sigma_{1}, \sigma_{2} \ldots\}$,
    %$\uparrow \!\! \Gamma = \Delta = \{x_{1}:\sigma_{1}, x_{2}:\sigma_{2}, \ldots \}$.
    %Индукция по сложности доказательства $\sigma$: \\
    %База: \infer{\Gamma, \varphi \vdash \varphi}{}
    %\begin{enumerate}
    %    \item $\varphi \in \Gamma$. Тогда \infer{\Delta \vdash x_\varphi : \varphi}{}
    %    \item $\varphi \notin \Gamma$. Тогда \infer{\Delta, x_\varphi : \varphi \vdash x_\varphi : \varphi}{}, $x_\varphi$ "--- новая переменная.
    %\end{enumerate}
    %Переход: \infer{\Gamma \vdash \varphi \rightarrow \psi}{\Gamma, \varphi \vdash \psi}.
    %По индукционному предположению $\Delta, y : \varphi \vdash M : \psi$.
    %\begin{enumerate}
    %    \item $\varphi \in \Gamma$. Тогда
    %    \infer{\Delta \setminus \left\{x_\varphi\right\} \vdash \lambda x_\varphi . M : \varphi \rightarrow \psi}{\Delta \vdash M : \psi}.
    %    \item $\varphi \notin \Gamma$. Тогда
    %    \infer{\Delta \vdash \lambda x_\varphi . M : \varphi \rightarrow \psi}{\Delta, x_\varphi : \varphi \vdash M : \psi}.
    %\end{enumerate}
\end{proof}
