\subsection*{\texorpdfstring{Вывод типа с использованием ограничений}{Constraint-based type inference}}

Ой \todo, \todo.

Алгоритм $W$ довольно прост, однако он нагромождённый.
В нём мы одновременно обрабатываем выражение, решаем, каким условиям должны удовлетворять их типы, и выводим эти типы.
Из общих соображений хорошего стиля программирования было бы лучше,
если бы мы разделили логику алгоритма вывода типа на подпрограммы.

Давайте выделим этап построения \emph{ограничений} на тип выражения.
Для начала рассмотрим просто типизированное $\lambda$-исчисление.
Грамматика ограничений:
\begin{bnf}
\[
    C ::= \tau = \tau | C \land C | \exists \alpha . C
\]
\end{bnf}%
Алгоритм построения ограничений:
\begin{align*}
    \restr{\Gamma \vdash x : \tau} \ &=\  \Gamma(x) = \tau \\
    \restr{\Gamma \vdash \lambda x . e : \tau} \ &=\ 
        \exists \alpha_1 \alpha_2 . (\restr{\Gamma, x : \alpha_1\vdash e:\alpha_2} \land \alpha_1\to\alpha_2=\tau) \\
    \restr{\Gamma \vdash e_1 e_2 : \tau} \ &=\ 
        \exists\alpha.(\restr{\Gamma\vdash e_1 :\alpha\to\tau} \land \restr{\Gamma\vdash e_2 : \alpha})
\end{align*}
Где $\alpha_1, \alpha_2, \alpha \notin \FV(\Gamma, \tau)$.

\begin{definition}[эрбановский универсум]
    Пусть $c_i$ "--- константы языка, $f^i$ "--- функциональные символы языка.
    \begin{align*}
        H_0 &= \set{c_1 \ldots c_k \ldots} \\
        H_i &= H_{i-1} \cup \set{f^i(h_1 \ldots h_t) \mid h_t \in H_{i-1}}
    \end{align*}
    Тогда $H = \cup_{i=0}^{\infty} H_i$ "--- эрбановский универсум.
\end{definition}

Пусть $\varphi : T \to H$ ($T$ "--- множество типов).

\begin{theorem}
    $\varphi$ "--- решение $\restr{\Gamma\vdash e : \tau}$ т.и.т.т., когда $(\varphi\Gamma, \varphi\tau)$ "--- типизация $e$.
\end{theorem}

\begin{example} Построим ограничения на тип выражения $\lambda x . x$:
    \begin{align*}
        \restr{\vdash \lambda x . x : \sigma}
        &= \exists \alpha_1 \alpha_2. (\restr{x : \alpha_1 \vdash x : \alpha_2} \land \alpha_1\to\alpha_2 = \sigma) \\
        &= \exists \alpha_1 \alpha_2. (\set{x : \alpha_1}(x) = \alpha_2 \land \alpha_1\to\alpha_2 = \sigma) \\
        &= \exists \alpha_1 \alpha_2. (\alpha_1 = \alpha_2 \land \alpha_1\to\alpha_2 = \sigma)
    \end{align*}
\end{example}

Нам тут не нравится контекст \todo. Давайте избасимся от него.
Добавим в язык ограничений ещё две конструкции:
\begin{bnf}
\[
    C ::= \ldots | x = \tau | \deff{x : \tau}{C}
\]
\end{bnf}%
Теперь решением будет не только $\varphi : T \to H$, но и $\psi : X \to H$ ($X$ "--- множество переменных).
Решение будет удовлетворять ограничению $x = \tau$, если $\psi x = \varphi \tau$.
Решение будет удовлетворять ограничению $\deff{x:\tau}{C}$, если $C$ удовлетворено при $\varphi$ и $\psi[x \coloneqq \varphi(\tau)]$.

Перепишем алгоритм:
\begin{align*}
    \restr{x : \tau} \ &=\  x = \tau \\
    \restr{\lambda x . e : \tau} \ &=\ 
        \exists \alpha_1 \alpha_2 . (\deff{x : \alpha_1}{\restr{e:\alpha_2}} \land \alpha_1\to\alpha_2=\tau) \\
    \restr{e_1 e_2 : \tau} \ &=\ 
        \exists\alpha.(\restr{e_1 :\alpha\to\tau} \land \restr{e_2 : \alpha})
\end{align*}

Теперь расширим наш алгоритм на систему Хиндли-Милнера. Языком ограничений будет
\begin{bnf}
\begin{align*}
    C &::= \tau = \tau | C \land C | \exists \alpha . C | x \preceq \tau | \deff{x : \varsigma}{C} \\
    \varsigma &::= \forall \overline\alpha[C] . \tau
\end{align*}
\end{bnf}%
где $x \preceq \tau$ означает, что $\tau$ "--- подтип $x$, а $\varsigma$ "--- типовая схема.

Сам алгоритм:
\begin{align*}
    \restr{x : \tau} \ &=\  x = \tau \\
    \restr{\lambda x . e : \tau} \ &=\ 
        \exists \alpha_1 \alpha_2 . (\deff{x : \alpha_1}{\restr{e:\alpha_2}} \land \alpha_1\to\alpha_2=\tau) \\
    \restr{e_1 e_2 : \tau} \ &=\ 
        \exists\alpha.(\restr{e_1 :\alpha\to\tau} \land \restr{e_2 : \alpha}) \\
    \restr{\lett{x = e_1}{e_2}} \ &=\ \deflet{x : \forall \alpha[\restr{e_1 : \alpha}].\alpha}
        {\restr{e_2:\tau}}
\end{align*}
где $\deflet{x : \varsigma}{C} \equiv \deff{x : \varsigma}{\exists \alpha . x \preceq \alpha \land C}$.
\\ \todo

\subsection{\texorpdfstring{Рекурсивные типы}{Recursive types}}
Допустим, мы хотим представить список. Все знают, что список это
\begin{minted}{Haskell}
data List a = Nil | Cons a (List a)
\end{minted}
Хотелось бы написать $\operatorname{\mathtt{list}}\alpha=\varnothing\lor(\alpha\with\operatorname{\mathtt{list}}\alpha)$,
однако так нельзя.
Текущими выразительными средствами мы это сделать не можем.

Есть два способа решить эту проблему.

\subsection*{\texorpdfstring{Эквирекурсивные типы}{Equirecursive types}}
Введём аналог комбинатора неподвижной точки для типов, $\mu$-оператор.
Список можно будет записать так: $\operatorname{\mathtt{list}} \alpha = \mu \beta . \varnothing \lor \alpha \land \beta$.
Будем считать, что типы $\mu \alpha . T$ и $T[\alpha \coloneqq \mu \alpha.T]$ равны.
На основе такого равенства сделаем отношение эквивалентности $(=_\mu)$.
Введём правило:
\begin{bnf}
\[
    \infer[(\tau_1=_\mu \tau_2)]
        {\Gamma \vdash e : \tau_2}
        {\Gamma \vdash e : \tau_1}
\]%
\end{bnf}%

Такое решение было применено в Java. Там можно писать \mintinline{Java}{Enum<E extends Enum<E>>}.

\subsection*{\texorpdfstring{Изорекурсивные типы}{Isorecursive types}}
Будем считать типы $\mu \alpha . T$ и $T[\alpha \coloneqq \mu \alpha.T]$ изоморфными.
Введём две операции:
\begin{align*}
    \func{fold}   &: T[\alpha \coloneqq \mu \alpha.T] \to \mu \alpha . T \\
    \func{unfold} &: \mu \alpha . T \to T[\alpha \coloneqq \mu \alpha.T]
\end{align*}
Разница в том, что изорекурсивные типы нужно сворачивать и разворачивать вручную, мы не считаем их равными.

Типы и указатели на них в Си "--- пример изорекурсивных типов.
