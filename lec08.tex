\section{\texorpdfstring{\textlambda-куб}{Lambda cube}}
\epigraph{А где все?}{Д.Г.}

\subsection{\texorpdfstring{Зависимые типы}{Depended types}}

Рассмотрим такой код на Си:
\begin{minted}{C}
int n;
scanf("%d", &n);
int a[n];
\end{minted}
Возникает вопрос, какой тип у \mintinline{C}{a}.
Во-первых, это массив. Во-вторых, это массив значений тип \mintinline{C}{int}. В треьтих, он размера \mintinline{C}{n}.
Хотелось бы это как-то формализовать.

Вспомним логику первого порядка.
В ней есть конструкции вида $\forall x . \varphi$ и $\exists x . \varphi$, и аксиомы для них: %
\begin{@empty}
\inferspacing
\begin{gather*}
    \infer[(x \notin \FV(\Gamma))]{\Gamma \vdash \forall x . \varphi}{\Gamma \vdash \varphi} \qquad
    \infer{\Gamma \vdash \varphi \left[x \coloneqq \sigma\right]}{\Gamma \vdash \forall x . \varphi} \\
    \infer{\Gamma \vdash \exists x . \varphi}{\Gamma \vdash \varphi \left[x \coloneqq \psi\right]} \qquad
    \infer[(x \notin \FV(\Gamma, \psi))]{\Gamma \vdash \psi}{\Gamma \vdash \exists x . \varphi && \Gamma, \varphi \vdash \psi}
\end{gather*}
\end{@empty}%
В логике второго порядка переменные под кванторами могли принимать значения любых выражений, что давало там выразительную силу.
Выражения зависели от типов.

Теперь же давайте скажем, что типы могут зависеть от значений объектов.
Будем обозначать неизвестный тип символом $*$.
Тогда можно записать "<тип"> массива так: $\mintinline{C}{[]}: * \to \mintinline{C}{int} \to *$.
Однако, называть это типом нельзя, иначе мы бы могли писать противоречивые конструкции вида $* : * \to *$.
Это \emph{род}, обозначать мы его будем символом $\Box$. Это "<тип"> типового конструктора.
Например, $* \to \mintinline{C}{int} \to * : \Box$.
Вместе типы и рода это \emph{сорта}.

Сейчас мы определим исчисление, в котором типы могут зависить от значений.

\begin{definition}[обобщённая типовая система]
Грамматика выражения:
\begin{bnf}
\[
    T ::= x | c | T T | \lambda x : T . T | \Pi x : T . T
\]
\end{bnf}%
где под выражение $x$ подходят все переменные, под $c$ водходят все константы.

Есть две выделенные константы для сортов: $*$ и $\Box$.
Пусть $s, s_1, s_2 \in \set{*, \Box}$. Общие правила вывода:
\inferspacing
\begin{gather*}
    \infer[(\text{Аксиома})]{\vdash * : \Box}{} \\
    \infer[(\text{Введение}, x \notin \Gamma)]{\Gamma, x : A \vdash x : A}{\Gamma \vdash A : s} \qquad
    \infer[(\text{Ослабление}, x \notin \Gamma)]{\Gamma, x : C \vdash A : B}{\Gamma \vdash A : B && \Gamma \vdash C : s} \\
    \infer[(\text{Применение})]
        {\Gamma \vdash (Fa) : B [x \coloneqq a]}{\Gamma \vdash F : \left(\Pi x : A . B\right) && \Gamma \vdash a : A} \qquad
    \infer[(\text{Конверсия}, B =_\beta B')]
        {\Gamma \vdash A : B'}{\Gamma \vdash A : B \vphantom{(} && \Gamma \vdash B' : s} %todo убрать костыль
\end{gather*}%
Следующие специальные правила параметризуются парами сортов $(s_1, s_2)$:
\begin{@empty} \inferspacing
\begin{gather*}
    \infer[(\Pi\text{-правило})]{\Gamma \vdash (\Pi x : A . B) : s_2}{\Gamma \vdash A : s_1 && \Gamma, x : A \vdash B : s_2} \\
    \infer[(\lambda\text{-правило})]{\Gamma \vdash (\lambda x : A . b) : (\Pi x : A . B)}
            {\Gamma \vdash A : s_1 && \Gamma, x : A \vdash b : B && \Gamma, x : A \vdash B : s_2}
\end{gather*}
\end{@empty}%
\end{definition}

$\Pi$ "--- это обобщение $\to$. $\Pi x : a . b$, где $b$ зависит от $x$, это множество пар,
каждая из которых сопоставляет типу $a$ тип $b$.
В частности, если $\varphi = \Pi x : \psi . \sigma$, и $x \notin \FV(\sigma)$,
то аналогичной записью будет $\varphi = \psi \to \sigma$.

Рассмотрение обобщённой типовой системы мы оставим на потом, а пока рассмотрим ту же систему,
в которой специальные правила ограничены парами $(s_1, s_2) \in \set{(*, *), (*, \Box)}$.
Такая система называется $\lambda P$.
Правила с $(s_1, s_2) = (*, *)$ это обычные правила для $\lambda$-выражений.
Правила с $(s_1, s_2) = (*, \Box)$ это правила для типов, зависящих от значений ($\lambda$-выражений).

В системе $\lambda P$ тип массива из \mintinline{C}{int} можно записать так:
$\mintinline{C}{int[]} \equiv \Pi \mintinline{C}{x} : \mintinline{C}{int}.\mintinline{C}{int[x]}$.
Мы явно указали, что тип зависит от значения типа \mintinline{C}{int}.
То есть тип \mintinline{C}{int[]} это набор всех возможных пар из размера массива и массива соответствующей длины.
Тогда если \mintinline{C}{int a[5]}, то $\mintinline{C}{a}:\mintinline{C}{int[5]}$.

\begin{example}
Типизируем $\lambda x . x$ (опуская $\vdash * : \Box$):
\[
    \infer{a : * \vdash (\lambda x : a . x) : (\Pi x : a . a)}
        { \infer{a : * \vdash a : *}{}
        &&\infer{a : *, x : a \vdash x : a}{\infer{x : a \vdash x : a}{}}
        &&\infer{a : *, x : a \vdash a : *}{\infer{a : * \vdash a : *}{}}
        }
\]
\todo ещё надо
\end{example}

\subsection{\texorpdfstring{Обобщённая типовая система и \textlambda-куб}{Generalized type system and lambda cube}}

Ранее мы рассматривали систему $F$, в которой был полиморфизм за счёт выражений, зависящих от типов.
Нам ничего не мешает совместить полиморфизм с зависимыми типами.
Кроме того, для формализации массивов из Си нам всё ещё не хватает типов, которые зависили бы от других типов.
Их мы также можем добавить.

$\lambda$-куб "--- это систематизация этих трёх не зависящих друг от друга расширений
просто типизированного $\lambda$-исчисления.
Позволяя или запрещая различные пары сортов $(s_1, s_2)$ можно получать разные типовые системы:
\begin{center}
\begin{tabular}{l c c c c} \toprule
    $\lambda{\to}$         & $(*, *)$ &             &                &             \\
    $\lambda 2$                    & $(*, *)$ & $(\Box, *)$ &                &             \\
    $\lambda \underline \omega$    & $(*, *)$ &             & $(\Box, \Box)$ &             \\
    $\lambda \omega$               & $(*, *)$ & $(\Box, *)$ & $(\Box, \Box)$ &             \\
    $\lambda P$                    & $(*, *)$ &             &                & $(*, \Box)$ \\
    $\lambda P2$                   & $(*, *)$ & $(\Box, *)$ &                & $(*, \Box)$ \\
    $\lambda P\underline \omega$   & $(*, *)$ &             & $(\Box, \Box)$ & $(*, \Box)$ \\
    $\lambda P \omega = \lambda C$ & $(*, *)$ & $(\Box, *)$ & $(\Box, \Box)$ & $(*, \Box)$ \\ \bottomrule
\end{tabular}
\end{center}%
Эти пары имеют следующий смысл:
\begin{center}
\begin{tabular}{r l@{\ }l} \toprule
    $(*, *)$       & разрешены правила для выражений, &зависящих от выражений \\
    $(\Box, *)$    & разрешены правила для выражений, &зависящих от типов     \\
    $(*, \Box)$    & разрешены правила для типов,     &зависящих от выражений \\
    $(\Box, \Box)$ & разрешены правила для типов,     &зависящих от типов     \\ \bottomrule
\end{tabular}
\end{center}

Система $F$, например, в таблице обозначена как $\lambda 2$.
Системы, в которых мы можем писать нечто вроде
$\mintinline{C}{[]}: * \to \mathtt{int} \to *$ это $\lambda P \underline \omega$ и $\lambda C$.

Наглядно такую классификацию обычно представляют в виде куба (рисунок \ref{cube}).
\begin{figure}[ht]
\centering
\begin{tikzpicture}
\matrix (m) [matrix of math nodes,
row sep=3em, column sep=3em,
text height=1.5ex,
text depth=0.25ex]{
\phantom{\lambda P \underline{\omega}} & \lambda\omega & \phantom{\lambda P \underline{\omega}} & \lambda P \omega             \\
\lambda 2   &   \phantom{\lambda P \underline{\omega}} & \lambda P 2                                                           \\
            & \lambda\underline{\omega}                &                                        & \lambda P \underline{\omega} \\
\lambda{\to}&                                          & \lambda P                                                             \\
};
\path[-{Latex[length=2.5mm, width=1.5mm]}]
(m-1-2) edge (m-1-4)
(m-2-1) edge (m-2-3)
        edge (m-1-2)
(m-3-2) edge (m-1-2)
        edge (m-3-4)
(m-4-1) edge (m-2-1)
        edge (m-3-2)
        edge (m-4-3)
(m-3-4) edge (m-1-4)
(m-2-3) edge (m-1-4)
(m-4-3) edge (m-3-4)
        edge (m-2-3);
\end{tikzpicture}
\caption{$\lambda$-куб.}
\label{cube}
\end{figure}

\begin{example}
В $\lambda{\to}$ выводимо $A : * \vdash \Pi x : A.A : *$ (или $A \to A : *$):
\[
    \infer[(\Pi\text{-правило})]{A : * \vdash \Pi x : A . A : *}
    {   \infer[(\text{Введение})]{A : * \vdash A : *}{\infer[(\text{Аксиома})]{\vdash * : \Box}{}}
    &&  \infer[(\text{Ослабление})]{A : *, x : A \vdash A : *}
        {   \infer[(\text{Введение})]{A : * \vdash A : *}{\infer[(\text{Аксиома})]{\vdash * : \Box}{}}
        &&  \infer[(\text{Введение})]{A : * \vdash A : *}{\infer[(\text{Аксиома})]{\vdash * : \Box}{}}
        }
    }
\]
Выведем $A : *, b : A \vdash ((\lambda a : A . a) b) : A$ (из-за большого объёма две ветки вывода опущены):
\begin{@empty}
\inferspacing
\begin{gather}
    \infer[(\text{Введение})]{A : *, b : A , a : A \vdash a : A}
    {   \infer[(\text{Ослабление})]{A : *, b : A \vdash A : *}
        {   \infer[(\text{Введение})]{A : * \vdash A : *}{\infer[(\text{Аксиома})]{\vdash * : \Box}{}}
        &&  \infer[(\text{Введение})]{A : *, b : A \vdash b : A}
            {   \infer[(\text{Введение})]{A : * \vdash A : *}{\infer[(\text{Аксиома})]{\vdash * : \Box}{}}
            }
        }
    } \label{q1} \\
    \infer{A : *, b : A \vdash ((\lambda a : A . a) b) : A}
    {   \infer{A : *, b : A \vdash (\lambda a : A . a) : (\Pi x : A . A)}
        {   \infer{A : *, b : A \vdash A : *}{(\text{ослабление})}
        &&  \infer{A : *, b : A , a : A \vdash a : A}{(\ref{q1})}
        &&  \infer{A : *, b : A , a : A \vdash A : *}{(\text{два ослабления})}
        }
    &&  \infer{A : *, b : A \vdash b : A}{(\text{ослабление})}
    } \nonumber
\end{gather}
\end{@empty}%
Два неуказаных правила это применение и $\lambda$-правило.
\\ \todo
\end{example}
