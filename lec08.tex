\section{\texorpdfstring{Зависимые типы и обобщённая типовая система}{Dependent types and generalized type systems}}

\subsection{\texorpdfstring{Зависимые типы}{Dependent types}}

Рассмотрим такой код на Си:
\begin{minted}{C}
int n;
scanf("%d", &n);
int a[n];
\end{minted}
Возникает вопрос, какой тип у \mintinline{C}{a}.
Во-первых, это массив. Во-вторых, это массив значений тип \mintinline{C}{int}. В треьтих, он размера \mintinline{C}{n}.
Хотелось бы это как-то формализовать.

\begin{@empty}
Вспомним логику первого порядка.
В ней есть конструкции вида $\forall x . \varphi$ и $\exists x . \varphi$, и аксиомы для них: %
\inferspacing
\begin{gather*}
    \infer[(x \notin \FV(\Gamma))]{\Gamma \vdash \forall x . \varphi}{\Gamma \vdash \varphi} \qquad
    \infer{\Gamma \vdash \varphi \left[x \coloneqq \sigma\right]}{\Gamma \vdash \forall x . \varphi} \\
    \infer{\Gamma \vdash \exists x . \varphi}{\Gamma \vdash \varphi \left[x \coloneqq \psi\right]} \qquad
    \infer[(x \notin \FV(\Gamma, \psi))]{\Gamma \vdash \psi}{\Gamma \vdash \exists x . \varphi && \Gamma, \varphi \vdash \psi}
\end{gather*}
В логике второго порядка переменные под кванторами могли принимать значения любых выражений, что давало там выразительную силу.
Типы зависели от других типов.
Теперь же давайте скажем, что типы могут зависеть не только от типов, а ещё от значений объектов.
\end{@empty}

Можно было бы записать "<тип"> массива так: $\mintinline{C}{[]}: * \rightarrow \mathtt{int} \rightarrow *$.
Однако, назвать это типом нельзя. Это \emph{род}. А вместе типы и рода это \emph{сорта}.

\subsection*{\texorpdfstring{Система $\lambda P$}{Lambda P}}
\begin{definition}[грамматика выражения в системе $\lambda P$]
\begin{bnf}
\begin{align*}
    \kappa &::= * | (\Pi x : \tau) \kappa \\
    \tau   &::= \alpha | (\forall x . \tau) \tau | \tau M \\
    M      &::= x | MM | \lambda x : \tau . M | (M)
\end{align*}
\end{bnf}%
$\kappa$ это род, $\tau$ это сорт.
Контекст в новом исчислении будет состоять из выражений вида $x : \tau$ и $\alpha : \kappa$.
\end{definition}

Теперь мы можем записать, например
$\mintinline{C}{[]} : (\Pi x : \mintinline{C}{int}) \mathinner{*} \vdash a : \mathinner{\mintinline{C}{[]}} 5$.
Или $5 : \mintinline{C}{int} : * : \Box$, то есть $5$ имеет тип $\mintinline{C}{int}$, $\mintinline{C}{int}$ имеет род $*$, а $*$ это сорт.

\subsection{\texorpdfstring{Обобщённая типовая система}{Generalized type system}}
