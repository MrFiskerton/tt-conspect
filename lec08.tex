\section{\texorpdfstring{$\lambda$-куб}{Lambda cube}}
\epigraph{А где все?}{Д.Г.}

\subsection{\texorpdfstring{Зависимые типы}{Depended types}}

Рассмотрим такой код на Си:
\begin{minted}{C}
int n;
scanf("%d", &n);
int a[n];
\end{minted}
Возникает вопрос, какой тип у \mintinline{C}{a}.
Во-первых, это массив. Во-вторых, это массив значений тип \mintinline{C}{int}. В треьтих, он размера \mintinline{C}{n}.
Хотелось бы это как-то формализовать.

\begin{@empty}
Вспомним логику первого порядка.
В ней есть конструкции вида $\forall x . \varphi$ и $\exists x . \varphi$, и аксиомы для них: %
\inferspacing
\begin{gather*}
    \infer[(x \notin \FV(\Gamma))]{\Gamma \vdash \forall x . \varphi}{\Gamma \vdash \varphi} \qquad
    \infer{\Gamma \vdash \varphi \left[x \coloneqq \sigma\right]}{\Gamma \vdash \forall x . \varphi} \\
    \infer{\Gamma \vdash \exists x . \varphi}{\Gamma \vdash \varphi \left[x \coloneqq \psi\right]} \qquad
    \infer[(x \notin \FV(\Gamma, \psi))]{\Gamma \vdash \psi}{\Gamma \vdash \exists x . \varphi && \Gamma, \varphi \vdash \psi}
\end{gather*}
В логике второго порядка переменные под кванторами могли принимать значения любых выражений, что давало там выразительную силу.
Типы зависели от других типов.
\end{@empty}

Теперь же давайте скажем, что типы могут зависеть не только от типов, а ещё от значений объектов.
Можно было бы записать "<тип"> массива так: $\mintinline{C}{[]}: * \rightarrow \mathtt{int} \rightarrow *$.
Однако, назвать это типом нельзя. Это \emph{род}. А вместе типы и рода это \emph{сорта}.

Сейчас мы определим исчисление, в котором типы могут зависить от значений.

\begin{definition}[обобщённая типовая система]
Грамматика выражения:
\begin{bnf}
\[
    T ::= x | c | T T | \lambda x : T . T | \lambda \Pi : T . T
\]
\end{bnf}%
где под выражение $x$ подходят все переменные, под $c$ водходят все константы.

Есть две выделенные константы для сортов: $*$ и $\Box$.
Пусть $s, s_1, s_2 \in \set{*, \Box}$. Общие правила вывода:
\inferspacing
\begin{gather*}
    \infer{\vdash * : \Box}{} \qquad
    \infer[(x \notin \Gamma)]{\Gamma, x : A \vdash x : A}{\Gamma \vdash x : s} \qquad
    \infer[(x \notin \Gamma)]{\Gamma, x : C \vdash A : B}{\Gamma \vdash A : B && \Gamma \vdash C : s} \\
    \infer{\Gamma \vdash (Fa) : B [x \coloneqq a]}{\Gamma \vdash F : \left(\Pi x : A . B\right) && \Gamma \vdash a : A} \qquad
    \infer[(B =_\beta B')]{\Gamma \vdash A : B'}{\Gamma \vdash A : B \vphantom{(} && \Gamma \vdash B' : s} %todo убрать костыль
\end{gather*}%
Следующие специальные правила параметризуются парами $(s_1, s_2)$:
\begin{gather*}
    \infer{\Gamma \vdash (\Pi x : A . B) : s_2}{\Gamma \vdash A : s_1 && \Gamma, x : A \vdash B : s_2} \qquad
    \infer{\Gamma \vdash (\lambda x : A . b) : (\Pi x : A . B)}
            {\Gamma \vdash A : s_1 && \Gamma, x : A \vdash b : B && \Gamma, x : A \vdash B : s_2}
\end{gather*}
\end{definition}

$\Pi$ "--- это обобщение $\rightarrow$. $\Pi x : a . b$, где $b$ зависит от $x$, это множество пар,
каждая из которых сопоставляет типу $a$ тип $b$.
В частности, если $\varphi = \Pi x : \psi . \sigma$, и $x \notin \FV(\sigma)$,
то аналогичной записью для $\varphi$ будет $\varphi = \psi \rightarrow \sigma$.

Рассмотрение обобщённой типовой системы мы оставим на потом, а пока рассмотрим ту же систему,
в которой специальные правила ограничены парами $(s_1, s_2) \in \set{(*, *), (*, \Box)}$.
Такая система называется $\lambda P$.
Правила с $(s_1, s_2) = (*, *)$ это обычные правила для $\lambda$-выражений.
Правила с $(s_1, s_2) = (*, \Box)$ это правила для типов, зависящих от значений ($\lambda$-выражений).

В системе $F$ тип массива из \mintinline{C}{int} можно записать так:
$\mintinline{C}{int[]} \equiv \Pi \mintinline{C}{x} : \mintinline{C}{int}.\mintinline{C}{int[x]}$.
Мы явно указали, что тип зависит от значения типа \mintinline{C}{int}.
Тогда если \mintinline{C}{int a[5]}, то $\mintinline{C}{a}:\mintinline{C}{int[5]}$.

\begin{example} Типизируем $\lambda x . x$ (опуская $\vdash * : \Box$):
\[
    \infer{a : * \vdash \lambda x : a . x : \Pi x : a . a}
        { \infer{a : * \vdash a : *}{}
        &&\infer{a : *, x : a \vdash x : a}{\infer{x : a \vdash x : a}{}}
        &&\infer{a : *, x : a \vdash a : *}{\infer{a : * \vdash a : *}{}}
        }
\]
\todo ещё надо
\end{example}

\subsection{\texorpdfstring{Обобщённая типовая система}{Generalized type system}}

\subsection{\texorpdfstring{$\lambda$-куб}{Lambda cube}}
